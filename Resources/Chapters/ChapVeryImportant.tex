\documentclass[debug]{phd}

\begin{document}
%%
\ensurepagenumbering{arabic}
%%	
	%
	\chapter{Generalised Calibrations in \texorpdfstring{$\mathrm{AdS}$}{AdS} backgrounds}
	\label{chapbrane}
			
			\section{Introduction and Motivations}
			%
			In this chapter, based mainly on~\cite{oscar2}, we are interested in investigating the relation between the Exceptional Sasaki-Einstein structures~\cite{AshmoreESE} presented in~\cref{sec:ESE} and generalised brane calibrations, in $\mathrm{AdS}_5 \times M_5$ backgrounds in type IIB and in $\mathrm{AdS}_5 \times M_6$ and $\mathrm{AdS}_4 \times M_7$ compactifications in M-theory.
			
			Also in this case it is useful to analyse the problem through the lens of $G$-structures.
			We have mentioned how requiring the AdS background to be supersymmetric is equivalent to put integrability conditions on HV structures, that in that case take the name of \emph{Exceptional Sasaki-Einstein structures}~\cite{AshmoreESE}, or of \emph{Exceptional Calabi-Yau spaces (ECY)} for compactifications to Minkowski spacetimes~\cite{AshmoreECY}.
			
			G-structures also appear naturally in defining calibration forms on the compactification manifolds. 
			
			A $p$-form $\phi$ on a $d$-dimensional manifold $M$ ($d> p$) is a \emph{calibration} if and only if it is closed, \emph{i.e.} $\dd  \phi =0$, and its pull-back to any tangent $p$-plane $\mathcal{S}$ satisfies the inequality
				%	
					\begin{equation}\label{eq:def_cal}
						P_{\mathcal{S}}[\phi] \leq \vol_{\mathcal{S}}\, ,
					\end{equation}
				%
			where $\vol_{\mathcal{S}}$ is the volume form on the plane $\mathcal{S}$ induced from the metric on $M$~\cite{Cal_Geo}.
			
			The ordering relation in~\eqref{eq:def_cal} has to be read as $P_{\mathcal{S}}[\phi] = \alpha  \vol_{\mathcal{S}}$, with $\alpha  \in \mathbb{R}^+$ and $\alpha  \leq 1$.
			For the supersymmetric backgrounds relevant in string and M-theory the calibration forms can be written as bilinears in the supersymmetry Killing spinors. 
			For instance, on Calabi-Yau manifolds there are two types of calibration forms, which corresponds to products of the K\"ahler form and to the real part of the holomorphic form on the Calabi-Yau. 

			A $p$-dimensional submanifold $\Sigma_p$ is called \emph{calibrated} if it saturates the condition~\eqref{eq:def_cal} at each point: $P_{\Sigma_p} [\phi] = \vol_{\Sigma_p}$. 
			One can show that a calibrated submanifold minimises the volume in its homology class. 
			Indeed, given another submanifold $\Sigma'$, such that $\Sigma - \Sigma' =\partial B$ is the boundary of a $p+1$-dimensional manifold $B$, one has (see e.g.~\cite{Cal_Geo})
%
	\begin{equation*}
		\mathrm{Vol}\left( \Sigma' \right) = \int_{\Sigma'} \vol_{\Sigma'} \geq \int_{\Sigma} P_{\Sigma}[ \phi] + \int_{\partial B} P_{\partial B}[\phi] 
			= \int_{\Sigma} \vol_{\Sigma} + \int_{B} \dd  P_B[ \phi] = \mathrm{Vol} \left( \Sigma\right)\, , 
	\end{equation*}
%
where we used the definition of calibration form, Stokes theorem and the fact that $\Sigma$ saturates the inequality~\eqref{eq:def_cal}.
For a nice review on these arguments, one can refer, for instance, to~\cite{Joyce:2001nm}. 

Calibrations are useful tools in string theory because they provide a classification of supersymmetric branes in a given background. 
In a purely geometric background (no fluxes) supersymmetric branes wrap calibrated submanifolds, so that they minimise their volume~\cite{Becker:1995kb, Becker:1996ay, Gibbons:1998hm, Gauntlett:1998vk}. 
The calibration form is constructed as a bilinear in the Killing spinors of the background geometry, and its closure follows from the Killing spinor equations of the background. 

In the more general case of a background with non-trivial fluxes supersymmetric branes are associated with \emph{generalised calibrations}. 
Since the branes couple with the background fluxes, they do not correspond to minimal volume submanifolds but to configurations that minimise the energy. 
As in the fluxless case the generalised calibration form is related to the Killing spinors of the background~\cite{Gutowski:1999iu, Gutowski:1999tu, Townsend:1999nf, Gauntlett:2001ur, Gauntlett:2002sc, Gauntlett:2003cy, MS03, Cascales:2004qp, HPS03, HPS04}. 
Also in these cases the calibration forms can be written as bilinears in the Killing spinors and the closure of the generalised calibration form can then be deduced from the Killing spinor equations of the supersymmetric background~\cite{Cascales:2004qp, Gutowski:1999tu, Martucci:2005ht, MS03}.

In~\cref{sec:ESE} we have seen how the exceptional HV structure contains a generalised vector $K$ that generalises the Reeb vector field and the contact structure of usual Sasakian geometry.
For this reason, it is believed to encode information on brane configurations and conformal dimensions of chiral operators, as the contact structure does in~\cite{Martelli:2006yb}. 
In particular, in~\cite{AshmoreESE} the form parts of the generalised vector $K$ were conjectured to describe generalised calibrations for brane configurations dual to barionic operators in the dual gauge theory.
The aim of this chapter is to prove this conjecture and to show that the vector structure is indeed associated to generalised calibrations.

In the following, we focus on the calibrations forms associated to branes wrapping cycles in the internal manifolds and that are point-like in the AdS space. 
We show that for these configurations the general expression for the calibration forms that can be constructed using $\kappa$-symmetry can be expressed in terms of the generalised Killing vector $K$ defining the Exceptional Sasaki-Einstein structure and that the closure of the calibration forms is given by the integrability (more precisely the $L_K$ condition) of the ESE structure. 
Our results proves the conjecture appeared in~\cite{AshmoreESE}, that the generalised Killing vector is a generalised calibration.
We also partially discuss other brane configurations that are calibrated by the vector $K$.

The analysis in this chapter is far from being complete. 
For instance we did not fully study the calibration forms for branes with world-volumes spanning different directions in the AdS space. 
These should be related to components of the hypermultiplet structures and their closure to the moment map conditions. 
It would also be interesting to perform a similar analysis for compactifications to Minkowski space where the relevant structures are Generalised Calabi-Yau's~\cite{AshmoreECY}. 
We leave this analysis for future work.

Conventions for Clifford algebras and bilinears of spinor notations are relegated to~\cref{app:notation}.

\section{Generalised calibrations in M-theory}
\label{sec:Mcal}
%

The aim of this section is to study calibrations for supersymmetric brane configurations of M-theory on AdS backgrounds of the form
		%
		\begin{equation*}
			\dd s^2 = e^{2 \Delta} \dd s^2(X_D) + \dd s^2(M_d)\, ,
		\end{equation*}		
		%
 in terms of exceptional geometry.
AdS calibrations have been thoroughly discussed in the literature~\cite{Gutowski:1999iu, Gutowski:1999tu, MS03, Koerber:2007jb} and led to the notion of generalised calibration. 
In this section we will interpret these calibrations in terms of the Exceptional Sasaki-Einstein structures describing the AdS background.

Supersymmetry static M-theory backgrounds have been studied in~\cite{GauntlettGeomKill}. 
As we have seen in~\cref{chapComp}, a supersymmetric background admits a Majorana Killing spinor $\varepsilon$ satisfying,
	%
		\begin{equation}
			\nabla_M \varepsilon + \frac{1}{288} \left[ \Gamma_M^{\phantom{M}NPQR} - 8 \delta_M^{\phantom{M}N} \Gamma^{PQR} \right] G_{NPQR}\ \varepsilon =0 \, ,
		\end{equation}
	%
where $M,N, \ldots = 0,1, \dots, 10$, $G= \dd  A $ is the four-form field strength and the Gamma matrices are the Clifford algebra elements in $11$ dimensions. 
The four-form $G$ and the metric $g$ satisfy the relative equations of motion
	%
		\begin{align}
			R_{MN} - \dfrac{1}{12}\left( G_{MPQR}G_{N}^{\phantom{N}PQR} - \dfrac{1}{12}g_{MN} G^2 \right) &= 0 \, ,\\
			\dd  \star G + \dfrac{1}{2} G \wedge G &= 0\, .
		\end{align}
	%
	
	The Killing spinor can then be used to build one-, two- and five-forms
	%
	\begin{subequations}
		\label{11df}
			\begin{align}
			%
				\label{11d1f}
				\mathcal{K}_M &= \bar{\varepsilon} \Gamma_M \varepsilon \, , \\
			%
				\label{11d2f}
				\omega_{M N} &= \bar{\varepsilon} \Gamma_{M N} \varepsilon \, , \\
			%
				\label{11d5f}
				\Sigma_{MNPQR} & = \bar{\varepsilon} \Gamma_{M N PQ R} \varepsilon \, , 
			%
			\end{align}
	\end{subequations}
	%
and the supersymmetry conditions imply that
%
	\begin{eqnarray}
		& \dd  \mathcal{K} = \frac{2}{3} \iota_\omega G + \frac{1}{3} \iota_\Sigma \star G \, , \\
		& \dd  \omega = \iota_\mathcal{K} G \, , \\
		& \dd  \Sigma = \iota_\mathcal{K} \star G - \omega \wedge G \, . 
	\end{eqnarray}

Supersymmetry also implies that the vector $\hat{\mathcal{K}}^M$ dual to the one-form~\eqref{11d1f} is a Killing vector, \emph{i.e.}
	%
		\begin{equation}
			\begin{array}{ccc}
				\mathcal{L}_{\hat{\mathcal{K}}} g = 0\, , & \phantom{and} & \mathcal{L}_{\hat{\mathcal{K}}} G = 0 \, . 
			\end{array}
		\end{equation}	
	%
	The vector $\hat{\mathcal{K}}^M$ can be either null or time-like, and for the backgrounds of interest here it is time-like%
		\footnote{%
			In this case the forms $\mathcal{K}$, $\omega$ and $\Sigma$ define an $SU(5)$ structure in 11 dimensions.%
		}.%

	Now let us focus on AdS backgrounds,
	%
		\begin{equation}
		\label{eq:metric_mtheory_ads}
			\dd  s^2 = e^{2\Delta}\dd  s^2(\mathrm{AdS}) + \dd  s^2(M)\, ,
		\end{equation}
	%
	where $\Delta$ is a real function on $M$, the warp factor.
	
	As usual, to construct the generalised calibrations for M-branes we can make use of $\kappa$-symmetry. 
	A supersymmetric brane satisfies the bound
	%
		\begin{equation}
		\label{kprojector}
			\hat{\Gamma} \varepsilon = \varepsilon \, ,
		\end{equation}
	%
	where $\varepsilon$ is the background Killing spinor and the $\kappa$-symmetry operator $\hat \Gamma$ depends on the type of brane. For an $\mathrm{M}5$-brane this is defined as~\cite{MS03,Gabella:2012rc},
	%
		\begin{equation}
		\label{KopMth}
			\hat{\Gamma} = \frac{1}{L_{DBI}}\Gamma_0 \left[\frac{1}{4}\Gamma^\alpha  (\tilde{H}\lrcorner H)_\alpha  + \frac{1}{2} \Gamma^{\alpha \beta} \tilde{H}_{\alpha \beta}+ \frac{1}{5!}\Gamma^{\alpha _1 \ldots \alpha _5} \epsilon_{\alpha _1 \ldots \alpha _5} \right]\, , 
		\end{equation} 
	%
	where $H = d B + P[A]$ is the world-volume three-form, $\tilde{H}$ is its world-space dual~\cite{Pasti:1997gx,Pasti:1995tn,Bergshoeff:1998vx}%
		\footnote{%
			The field $\tilde{H}$ is defined in terms of an auxiliary scalar field $a$, which is needed to ensure the Lorentz covariance of the world-volume Lagrangian~\cite{Pasti:1997gx,Pasti:1995tn},
			%
				\begin{equation}
					\tilde{H}_{\mu\nu} = \frac{1}{\sqrt{\vert\partial a \vert^2}} \left(\star H\right)_{\mu\nu\alpha }\partial^{\alpha } a (\sigma)\, .
				\end{equation}
			%
			The scalar $a$ is subject to a gauge transformation and one usually fixes it by going to the \emph{temporal gauge}, \emph{i.e.} $a= \sigma^0 = t $. 
			This gauge fixing procedure breaks the Lorentz invariance $SO(1,5)$ down to $SO(5)$ and sets $\tilde{H}$ equal to the world-space dual of $H$.%
			}%
		and $L_{DBI}$ is the Dirac-Born-Infeld Lagrangian for the $\mathrm{M}5$ brane,
	%
		\begin{equation}
		\label{DBIlagr}
			L_{DBI} = - \sqrt{-\det(P[g] + \tilde{H})}\, .
		\end{equation} 
	%
	Per usual, $P[\bullet]$ denotes the pull-back on the $\mathrm{M}5$ world-volume and we defined
	%
		\begin{equation}
			\Gamma_{\alpha _1 \ldots \alpha _s} = \Gamma_{M_1 \ldots M_s} \partial_{\alpha _1}X^{M_1} \ldots \partial_{\alpha _s}X^{M_s}\, .
		\end{equation}
	%
	
	As discussed in~\cite{MS03, Gabella:2012rc}, the $\kappa$-symmetry condition~\eqref{kprojector} can be used to derive the following bound~\cite{Barwald:1999hx},
	%
		\begin{equation}
		\label{M5ksym}
			\lVert \varepsilon \rVert^2 L_{DBI}\ \vol_5 \geq \left[ \frac{1}{2} P[ \iota_{\hat{\mathcal{K}}} H ] \wedge H + P[\omega] \wedge H + P[ \Sigma] \right]\, ,
		\end{equation}
	%
	where $K$, $\Sigma$ and $\omega$ are defined in~\eqref{11df}.
	To satisfy the bound one has to take into account that the space is Anti-de Sitter. 
	As discussed in~\cite{Gabella:2012rc}, the norm $\varepsilon^\dagger \varepsilon$ depends on the AdS coordinates and the bound is saturated when the $\mathrm{M}5$ brane sits at the center of AdS.
	Explicitly, the metric of $\mathrm{AdS}_n$ in global coordinates can be written as,
	%
		\begin{equation}
			\dd  s^2 = R^2\left(-\cosh^2 \varrho\, \dd  t^2 + \dd  \varrho^2 + \sinh^2 \varrho\, \dd  \Omega_{n-2} \right)\, ,
		\end{equation}
	%
	and $\varepsilon^\dagger \varepsilon \propto \cosh \varrho$, thus, the~\eqref{M5ksym} can be saturated only for $\varrho = 0$, \emph{i.e.} in the center of AdS.

	Further, the bound~\eqref{M5ksym} can be used to derive a bound on the energy of the $\mathrm{M}5$ brane \cite{MS03,Gabella:2012rc}. The energy of the an $\mathrm{M}5$-brane is given by 
	%
		\begin{equation}
		\label{energyM5}
			E_{\mathrm{M}5} =- \int_{\mathcal{S}} \dd ^5 \sigma\ g(\hat{P},\hat{\mathcal{K}}) \, , 
		\end{equation}
	%
	where $\mathcal{S}$ denotes the $5$-dimensional world-space of the brane, $\hat{P}_M$ is the conjugate momentum%
		\footnote{%
		To write this expression, we have chosen again the static gauge $X^M = ( t, \sigma^\alpha )$.%
		}~\cite{Bergshoeff:1998vx},
	%
		\begin{equation}
			\begin{split}
				\hat{P}_M =& \frac{\partial L_{M 5}}{\partial (\partial_\tau X^M)} = P_M + \dfrac{1}{4} \dfrac{1}{5!} \epsilon^{\tau \alpha _1\ldots\alpha _5} H_{\alpha _1 \alpha _2\alpha _3}H_{\alpha _4\alpha _5\alpha _6} \partial^{\alpha _1} X_M \\
						&\phantom{= P_M + + } - \dfrac{\tau_5}{5!} \epsilon^{\tau \alpha _1\ldots\alpha _5}\left[ \iota_M \tilde{A} - \tfrac{1}{2} \iota_M A \wedge (A - 2 H)\right]_{ \alpha _1\ldots\alpha _5} \, ,
			\end{split}
		\end{equation}
	%
	where $X^M$ are the embedding coordinates of the brane. 
	The quantity $g(\hat{P},\hat{\mathcal{K}}) = \hat{P}^M \hat{\mathcal{K}}^N g_{MN}$ can be interpreted as a Noether charge density of the symmetry generated by $\hat{\mathcal{K}}$~\cite{Martucci:2011dn}. 
	Then the inequality~\eqref{M5ksym} gives a bound on the energy of the brane,
	%
		\begin{equation}
		\label{BPScon}
			E_{\mathrm{M}5} \geq E^{BPS}_{\mathrm{M}5}\, ,
		\end{equation}
	%
	where
	%
		\begin{equation}
			E^{BPS}_{\mathrm{M}5} = \int_{\mathcal{S}} P[\Sigma] + P[\iota_{\hat{\mathcal{K}}} \tilde{A}] + P[\omega] \wedge H + \tfrac{1}{2} P[\iota_{\hat{\mathcal{K}}} H] \wedge (A - 2H)\, .
		\end{equation}
	%
	As shown in~\cite{Gabella:2012rc}, the form
	%
		\begin{equation}
		\label{genkcal}
			\Phi_{\mathrm{M}5} = \Sigma + \iota_{\hat{\mathcal{K}}} \tilde{A}+\omega \wedge H + \tfrac{1}{2} \iota_{\hat{\mathcal{K}}} H \wedge (A - 2H)\, ,
		\end{equation}
	%
	is a generalised calibration, namely is closed by supersymmetry and it minimise the energy in its homology class being a topological quantity~\cite{MS03}.

	The discussion for $\mathrm{M}2$ works analogously, and the calibration form is 
	%
		\begin{equation}
		\label{genkcalM2}
			\Phi_{\mathrm{M}2} = \omega + \iota_{\hat{\mathcal{K}}} H\, .
		\end{equation}
	%
	
As final comment, we want just to point out that the construction above can also be derived by the supersymmetry algebra. The same calibration forms emerge in the supersymmetry algebra with the central extensions due to the presence of BPS extended objects, and one can prove their closure by using the Killing spinor equations~\cite{HPS03,Gutowski:1999tu, Cascales:2004qp}.

%%
%
\subsection{\texorpdfstring{Calibrations on $\mathrm{AdS}_5 \times M_6$}{Calibration on AdS5 x M6}}
%
Even if the formalism described in the previous section is completely general, in what follows we will focus on static M-branes in backgrounds of the type~\eqref{eq:metric_mtheory_ads} and we will show how the calibration forms~\eqref{genkcal} and~\eqref{genkcalM2} are naturally encoded in the generalised Sasaki-Einstein structure.

We consider first the case of compactifications to $5$-dimensional AdS. 
The supersymmetry conditions for backgrounds of this type are give in \cite{Gauntlett:2004zh} while the corresponding exceptional generalised geometry is given in~\cite{AshmoreESE,Grana_Ntokos}, and we briefly review it below. 
We refer to these works also for notation and conventions, and we collect, for convenience, again the relevant conventions used here in~\cref{app:mtheoryconv}.

The metric takes the form
\begin{equation}
\label{eq:metric_mtheory_ads5}
	\dd  s^2 = e^{2\Delta}\dd  s^2_{\mathrm{AdS}_5} + \dd  s^2_{M_6}\, ,
\end{equation}
where we denote the inverse AdS radius as $m$. As shown in~\cite{Gauntlett:2004zh}, supersymmetry constrains the geometry of the six-dimensional internal manifold: $M_6$ has a local $SU(2)$ structure and topologically is a two-sphere bundle over a four-dimensional base%
		\footnote{%
			The four dimensional base can be either a K\"ahler-Einstein manifold with positive curvature or a product of two constant curvature Riemannian surfaces. The latter case in non-Einstein~\cite{Gauntlett:2004zh}.%
			}. 

There is a non-trivial four-form field strength $\mathcal{F}$ with non-zero components along the internal manifold $M_6$, 
%
	\begin{equation}
		F_{m_1 \ldots m_4} = \left(\mathcal{F}\right)_{m_1 \ldots m_4}\, ,
	\end{equation}
%
while the external components are set to zero, $\mathcal{F}_{\mu_1 \ldots \mu_4} = 0$.

The internal flux $F$ satisfies the equations of motion and the Bianchi identity
%
	\begin{equation}
		\begin{array}{ccccc}
			\dd  F = 0\, , & & & & \dd  (e^\Delta \star_6 F) = 0\, , 
		\end{array}
	\end{equation}
%
with $\star_6$ the Hodge star on $M_6$, while the dual form $\tilde{F}_{m_1 \ldots m_7} = \left(\star_{11}\mathcal{F}\right)_{m_1 \ldots m_7}$ identically vanishes on the six-dimensional internal manifold $M_6$.

The Clifford algebra $\mathrm{Cliff}(1,10)$ decomposes in $\mathrm{Cliff}(1,4)$ and $\mathrm{Cliff}(0,6)$:
%
	\begin{equation}
	\label{eq:dec_gammas_m6}
		\begin{array}{ccc}
			\hat{\Gamma}^\mu = e^{\Delta}\,\rho^\mu \otimes \gamma_7\, , & \phantom{and} &\hat{\Gamma}^{m+4} = \mathds{1}_4 \otimes \gamma^m\, ,
		\end{array}
	\end{equation}
%
with $\gamma_7 = - i \gamma^1 \ldots \gamma^6$ the chiral operator in $6$ dimensions, and $\rho$ and $\gamma$ matrices satisfying
%
	\begin{equation}
	\label{eq:cliff}
		\begin{array}{ccc}
			\left\{\rho_\alpha , \rho_\beta \right\} = 2 \eta_{\alpha \beta} \mathds{1}\, , & \phantom{and} &\left\{\gamma_{a} , \gamma_b \right\} = 2 \delta_{ab} \mathds{1}\, ,
		\end{array}
	\end{equation}
%
in terms of the frame indices $\alpha ,\beta = 0,\ldots,4$ on $\mathrm{AdS}_5$ and $a,b = 1, \ldots, 6$ on $M_6$. We collect further conventions about spinors and Clifford algebras in~\cref{app:mtheoryconv}.
%

To have an $\mathcal{N}=2$ supersymmetric background we decompose the $11$-dimensional spinor as
%
	\begin{equation}
	\label{eq:decomp_mtheory_spinors}
		\varepsilon = \psi \otimes \chi + \psi^c \otimes \chi^c\, ,
	\end{equation}
	%
where $\psi$ is an element of $\mathrm{Cliff}(1,4)$. 
Notice that, in order to have an AdS backgrounds, the internal spinor $\chi$ cannot be a chirality eigenstate~\cite{Gauntlett:2004zh}. 
Hence, it can be written as,
%
	\begin{equation}
	\label{chiredef}
		\chi = \sqrt{2} \left(\cos \alpha  \chi_1 + \sin \alpha  \chi_2^* \right)\, ,
	\end{equation}
%
where $\alpha $ is a parameter chosen to get the unit norm for the spinor, as in~\cite{Gauntlett:2004zh}.\\


The exceptional geometry for these backgrounds is given in~\cite{AshmoreESE,Grana_Ntokos}.
%\subsubsection{Exceptional Sasaki-Einstein structure}
The exceptional bundles to consider are again~\eqref{MthExBun} and~\eqref{eq:Ggeom-M}. 
The vector structure $K\in E$ and the hypermultiplet structure $J_a \in \mathrm{ad}F$ can be expressed in terms of the $SU(2)$ structure of~\cite{Gauntlett:2004zh}. 

In this discussion we are mostly concerned with the generalised Killing vector $K$. 
Its untwisted version is given by
%
	\begin{equation}
	\label{vecK}
		\begin{array}{cc}
			\tilde{K}=\xi -e^{\Delta} Y' + e^{\Delta} Z \equiv \xi + \tilde{\omega} + \tilde{\sigma} & \in \tilde{E} 
		\end{array}
	\end{equation}
%
with the vector $\xi$, the two-form $Y'$ and five-form $Z$ defined in terms of spinor bilinears as in~\cite{AshmoreESE}, 
%
	\begin{align}
		\label{vecbilM6}
		\xi &= -i \left(\bar{\chi}_1+\chi_2^T\right)\gamma^{(1)}\left(\chi_1 - \chi_2^* \right)\, ,\\[1mm]
		%
		\label{YbilM6}
		Y' &= -i \left(\bar{\chi}_1+\chi_2^T\right)\gamma_{(2)}\left(\chi_1 - \chi_2^* \right)\, ,\\[1mm]
		%
		\label{ZbilM6}
		Z &= -i \left(\bar{\chi}_1+\chi_2^T\right) \gamma_{(5)} \left(\chi_1 - \chi_2^* \right)\, .
		%
	\end{align}
%
The twisted version of the V structure is obtained by the (exponentiated) adjoint action
%
\begin{equation}
	K = e^{A+\tilde{A}} \tilde{K}\, ,
\end{equation}
%
where $A$ is the three-form and $\tilde{A}$ the six-form potential of $M$-theory. Using the expressions for the commutator and the adjoint action from~\cite{AshmoreECY}, one obtains~\cite{AshmoreESE} 
%
	\begin{equation}
	\label{eq:twisted_K_M}
		K= \xi + \left( \iota_{\xi}A - e^{\Delta}Y'\right) + \left( e^{\Delta} Z - e^{\Delta} A \wedge Y' + \tfrac{1}{2} \iota_{\xi} A \wedge A\right)\, .
	\end{equation}
%
As discussed above, the tensor $\tilde{R}$ must vanish for the generalised Lie derivative to reduce to the usual one, and this is equivalent to~\cite{AshmoreESE}
%
	\begin{align} \label{isom5}
		& &\dd  \tilde{\omega} = \iota_{\xi} F\, , & & \dd  \tilde{\sigma} = \iota_{\xi} \tilde{F} - \tilde{\omega} \wedge F\, . & &
	\end{align}
%
On $M_6$, this yields the differential conditions
%
	\begin{subequations}
	\label{eq:m-theory_structure_eqs}
		\begin{align}
			\dd  \left( e^{\Delta} Y'\right) &= - \iota_{\xi} F\, , \\[1mm]
			\dd  \left( e^{\Delta}Z\right) &= e^{\Delta} Y' \wedge F\, ,
		\end{align}
	\end{subequations}
%
which we refer to as $L_K$ conditions in the language of exceptional generalised geometry.
%

Supersymmetry gives also the Killing vector condition
%
	\begin{equation}
		\mathcal{L}_\xi F = \mathcal{L}_\xi \Delta = \mathcal{L}_\xi g = 0\, . 
	\end{equation}
% 
%

We can now discuss how the generalised Killing vector $K$ is related to the calibration forms for supersymmetric branes.
The general calibrations for $\mathrm{M}5$ and $\mathrm{M}2$ branes are given by the~\eqref{genkcal} and~\eqref{genkcalM2}. 
With an appropriate choice of the $\mathrm{AdS}_5$ gamma matrices (see~\cref{app:mtheoryconv}), the 11-dimensional Killing vector $\mathcal{K}_M$ in~\eqref{11d1f} has only two non-zero components,
\begin{subequations}
\begin{align}
& \mathcal{K}_0 = \bar{\psi} \rho_0 \psi \\
& \mathcal{K}_m = -i \left(\bar{\chi}_1+\chi_2^T\right)\gamma_{m}\left(\chi_1 - \chi_2^* \right) = \xi_m \, ,
\end{align}
\end{subequations}
where we fixed the norms of the spinors to $\bar \chi \chi = 1$ and $(\bar{\psi}\psi) = i/2$ and $\xi_m$ is the Reeb vector on $M_6$. Consistently, we also fixed the value of the angular parameter to $\alpha  = \pi/4$ in~\eqref{chiredef}.

The specific expression of the calibration forms $\Phi_{\mathrm{M}5}$ and $\Phi_{\mathrm{M}2}$ in~\eqref{genkcal} and~\eqref{genkcalM2} depends on how many AdS directions are spanned by the world-volume of the branes. 


Consider an $\mathrm{M}5$ wrapping a $5$-cycle in $M_6$. 
We choose again the static gauge for the brane embedding and we set to zero the world-volume gauge field (so $H = A$). 
The the relevant components of the $\Sigma$ and $\omega$ in~\eqref{11d2f} and~\eqref{11d5f} are the internal ones,
%
	\begin{subequations}
		\begin{align}
			\omega_{m_1 m_2} &= e^\Delta \, \bar{\chi}\gamma_7 \gamma_{m_1 m_2} \chi = e^\Delta Y^\prime \, , \\
			\Sigma_{m_1 \ldots m_5} &= e^\Delta \, \bar{\chi}\gamma_7 \gamma_{m_1 \ldots m_5} \chi = e^\Delta Z \, , 
		\end{align}
	\end{subequations}
%
and the calibration form in~\eqref{genkcal} reads (recall that the pull-back of $\tilde A$ is zero),
%
	\begin{equation}
		\Phi_{\mathrm{M}5} = e^{\Delta} Z -e^{\Delta} A \wedge Y' + \tfrac{1}{2}\iota_{\xi} A\wedge A \, . 
	\end{equation}
%
Note that this is exactly the pull-back on the brane of the twisted generalised vector $K$ in~\eqref{eq:twisted_K_M}. 
A similar computation for an $\mathrm{M}2$-brane wrapping a $2$-cycle in $M_6$ gives
%
	\begin{equation}
		\Phi_{\mathrm{M}2} = e^{\Delta} Y' - \iota_{\xi} A \, ,
	\end{equation}
%
which is again the pull-back on the $\mathrm{M}2$-brane of the twisted generalised vector $K$.
Using the $L_K$ conditions~\eqref{eq:m-theory_structure_eqs} and choosing the a gauge for $A$ such that 
$\mathcal{L}_\xi A =0$, it is straightforward to check that $\Phi_{\mathrm{M}5}$ and $\Phi_{\mathrm{M}2}$ are closed.
Explicitly, for instance for $\mathrm{M}5$, one has,
%
	\begin{equation}
		\begin{split}
			\dd  \Phi_{\mathrm{M}5} &= \dd  (e^{\Delta} Z ) - \dd  (e^{\Delta} A \wedge Y') + \tfrac{1}{2}\dd (\iota_{\xi} A\wedge A) = \\
					&= e^{\Delta} Y' \wedge F - F \wedge e^\Delta Y' + \iota_\xi F \wedge A + \tfrac{1}{2}\dd (\iota_\xi A)\wedge A + \tfrac{1}{2}\iota_\xi A \wedge F = \\
					&= \iota_\xi F \wedge A + \tfrac{1}{2} \mathcal{L}_\xi A \wedge A - \tfrac{1}{2}\iota_\xi F \wedge A - \tfrac{1}{2} A \wedge \iota_\xi F = 0\, .
		\end{split}
	\end{equation}
%
Analogously, one can verify that $\Phi_{\mathrm{M}2}$ is also closed, showing that the purely internal configuration of the membrane is supersymmetric.
%

The generalised vector $K$ is also related to the calibration forms for other types of brane probes. 
Here we focus on branes with one one leg in the external space-time, that is a string moving in AdS. 
We leave the study of other membrane configurations to future work.
The calibration forms for $\mathrm{M}2$ and $\mathrm{M}5$-branes of this kind are given by~\eqref{genkcal} and~\eqref{genkcalM2} in this case, take the following form
%
	\begin{align}
		& \Phi_{\mathrm{M}2} = e^{2\Delta} \tilde{\zeta}_1 \\
		& \Phi_{\mathrm{M}5} = e^{2\Delta} \star Y' + e^{2\Delta} \tilde{\zeta}_1 \wedge A 
	\end{align}
%
where $Z = \star \tilde{\zeta}_1$. 
The two calibrations are components of the (poly)-form 
%
	\begin{equation}
	\label{eq:cal_M_q1}
		\Phi= e^{2\Delta}\tilde{\zeta}_1 +e^{2\Delta}\star_6 Y'\, + e^{2\Delta} \tilde{\zeta}_1 \wedge A\, .
	\end{equation}
%
which is the Hodge dual of the vector structure ~\eqref{eq:twisted_K_M} . 
We want now to study it's closure and its relation to the integrability conditions.
In this case, the closure follows from the moment map condition $\mu_3 \equiv 0$, rather than from the $L_{K}$ condition. 
In~\cite{AshmoreESE}, it is shown that this moment map condition requires
%
\begin{equation}
	\dd  \bigl( e^{2\Delta} \tilde{\zeta}_1\bigr) = 0\, ,
\end{equation}
%
so that this form calibrates a $\mathrm{M}2$-brane. 
Again, combining the two conditions, we get
%
\begin{equation}
	\begin{array}{lcr}
		\dd  \left( e^{3\Delta} \sin\Theta \right) = 2 m e^{2\Delta} \tilde{\zeta}_1 & \mbox{and} & \dd  \left( e^{3\Delta} V\right) = e^{3\Delta} \sin \Theta F + 2m e^{2\Delta} \star Y'\, .
	 \end{array}
\end{equation}
%
From the vanishing of $\mu_3$ in~\cite{AshmoreESE}, it is easy to verify that the form $e^{2\Delta} \star_6 Y' + e^{2\Delta}\tilde{\zeta}_1 \wedge A$ is closed (for non-vanishing $m$).
%
%
%%%%%%
%
%%%%%%%%%%%%%%%%%%%%%% M theory on AdS_4 %%%%%%%%%%%%%%%%%%%%%%%%%%%%%%%%%%%%%%%%%%%
\subsection{\texorpdfstring{Calibrations in $\mathrm{AdS}_4 \times M_7$}{Calibrations on AdS4 x M7}}
\label{M-thAdS4}
%
In this section, we discuss M-theory calibrations on $\mathrm{AdS}_4$ backgrounds. 
Again, we first review the exceptional generalised geometry~\cite{AshmoreESE} and then relate it to generalised calibrations.
Conventions for the spinor bilinears and the supersymmetry equations for the internal forms can be found in~\cite{Gabella:2012rc} and the relevant ones for this work are collected in appendix~\ref{app:conv}. 

The background metric has the following form 
%
	\begin{equation}
	\label{eq:metric_mtheory_ads4}
		\dd  s^2 = e^{2\Delta} \dd  s^2_{\mathrm{AdS}_4} + \dd  s^2_{M_7} \, .
	\end{equation}
%
We set the inverse $\mathrm{AdS}_4$ radius to $m=2$. In addition, there is a non trivial four-form flux 
%
	\begin{equation}
		G = m \vol_4 + F \, ,
	\end{equation}
%
where $F = \dd  A$ is the flux component on $M_7$ and it satisfies the following Bianchi identity and equations of motion 
%
	\begin{equation}
		\begin{array}{ccccc}
			\dd  F = 0\, , & & & &\dd ( e^{2\Delta} \star_7 F ) = - m F \, , 
	\end{array}
	\end{equation}
%
with $\star_7$ the Hodge star on $M_7$. 
We will also need its dual $\tilde{F} = \dd  \tilde{A} - \tfrac{1}{2} A \wedge F$. 

%
%
%
%
	
The $11$-dimensional gamma matrices split as 
%
	\begin{equation}
	\label{eq:dec_gammas_m7}
		\begin{array}{ccc}
			\Gamma_{\mu} = e^{\Delta} \rho_{\mu} \otimes \id & \mbox{and} & \Gamma_m = e^{\Delta} \rho_5 \otimes \gamma_m\, ,
		\end{array}
	\end{equation}
%
with $\{\rho_{\mu}, \rho_{\nu}\}=2 g_{\mu \nu}$ and $\{\gamma_m, \gamma_n\}= g_{mn}$. 
The matrix $\rho_5 = i \rho_{0123}$ is the chirality operator in four dimensions, and $\gamma_{1 \ldots 7}=i \id$.
For further details about Clifford algebra conventions we refer to the~\cref{app:conv}.

The spinor ansatz preserving eight supercharges reads~\cite{Gabella:2012rc, Gabella:2011sg}
%
	\begin{equation}
	\label{eq:spinor_m7}
		\begin{split}
			\varepsilon &= \sum_{i=1,2} \psi_i \otimes e^{\Delta/2} \chi_i + \psi^c_i \otimes e^{\Delta/2} \chi_i^c\\
					&= e^{\Delta/2} \psi_+ \otimes \chi_- + e^{\Delta/2} \psi_- \otimes \chi_+ + \mathrm{c.c.} 
		\end{split}
	\end{equation}
%
where $\chi_{\pm} \coloneqq \tfrac{1}{\sqrt{2}} \left( \chi_1 \pm i \chi_2\right)$ and $\psi_{\pm} \coloneqq \tfrac{1}{\sqrt{2}} \left( \psi_1 \pm \psi_2\right)$. 
In addition, we take the $\mathrm{AdS}_4$ spinors $\psi_i$ to have positive chirality, \emph{i.e.} $\rho_5 \psi_i = \psi_i$.\\

Combining the supersymmetry conditions and equations of motion for the fluxes one can express the internal fluxes in terms of spinor bilinears by~\cite{Gabella:2012rc},
%
	\begin{align}
		F &= \dfrac{3 m}{\tilde{f}}\ \dd  (e^{6 \Delta} i( \bar{\chi}_+^c \gamma_{(3)} \chi_-))\, , \\
		\tilde{F} &= - \tilde{f}\ \vol_7 \, . 
	\end{align}
%

The features of the solutions depend on the \emph{electric} charge $m$. 
When $m=0$ the solutions correspond to near horizon geometries of $\mathrm{M}5$-branes wrapped on internal cycles (no $\mathrm{M}2$ charge). 
The geometries with $m\neq0$ correspond to the presence of a non-vanishing $\mathrm{M}2$ charge. 
For $m \neq 0$ the internal manifolds always admit a canonical contact structure, as shown in~\cite{Gabella:2012rc}. \\

%%%
%%
%
%\subsubsection{Exceptional Sasaki-Einstein structure}
%\label{ESEAdS4}
%
	The generalised geometry relevant for backgrounds of this kind is discussed in~\cite{AshmoreESE}. 
	The $HV$ structure is given by a generalised vector $X$ in the fundamental of $E_{7(7)}$ and a triplet $J_a$ in the adjoint representation. 
	The untwisted vector reads
	%
		\begin{equation}
			\tilde{X}= \xi + e^{3\Delta} Y + e^{6\Delta} Z - i e^{9\Delta} \tau\, ,
		\end{equation}	
	%	
where the forms are bilinears in the internal background spinors 
	%
		\begin{equation}
		\label{bilrel}
			\begin{array}{lcccccr}
				\sigma = i \bar{\chi}_+^c \gamma_{(1)} \chi_- \, , &\phantom{and}& Y = i \bar{\chi}_+^c \gamma_{(2)} \chi_- \, ,&\phantom{and}& Z=\star_7 Y \, ,& \phantom{and}& \tau = \sigma \otimes \vol_7\, ,
			\end{array}
		\end{equation}
	%
and $\xi$ is the vector dual to the one-form $\sigma$. 
Notice that the vector structure has the same form in both cases of a Sasaki-Einsten internal manifold and of a generic flux background~\cite{AshmoreESE}. 
Indeed the seven-dimensional manifolds giving $\mathcal{N}=2$ supersymmetry always admit a local $SU(2)$ structure. 
Moreover, the Killing vector constructed by spinor bilinears in~\eqref{bilrel} (or equivalently its dual one-form $\sigma$) defines a contact structure.
This allows us to write the $M_7$ metric as a Reeb foliation, analogously to the case of a Sasaki-Einstein manifold~\cite{Gabella:2012rc}. 
As a consequence, the volume form can be written making use of the contact structure,
	%
		\begin{equation}
			\dfrac{1}{3!} \sigma \wedge \dd  \sigma \wedge \dd  \sigma \wedge \dd  \sigma = \left(\dfrac{3m^2}{\tilde{f}}\right)^3 e^{9\Delta} \vol_7 = 2\left(\dfrac{3m^2}{\tilde{f}}\right)^3 \sqrt{q(K)}\, ,
		\end{equation}
	%
where $q(K)$ is the $E_{7(7)}$ invariant and $K$ is the real part of the twisted vector structure $X$
	%
		\begin{equation}
		\label{Kads4}
			K = \xi - \frac{1}{2} \sigma \wedge \omega \wedge \omega + \iota_\xi \tilde{A}\, .
		\end{equation}
	%
As already mentioned in~\cref{sec:ESE}, supersymmetry implies that $X$ is a generalised vector and its vector part, $\xi$, is a Killing vector. 
Through AdS/CFT, $\xi$ is the dual of the R-symmetry of the conformal $\mathcal{N}=2$ gauge theory in three dimensions. 
Then, as discussed in~\cref{sec:ESE}, the generalised Lie derivative along $X$ must reduce to $\mathcal{L}_\xi$, which implies the vanishing of the tensor $\tilde{R}$, or more explicitly
	%		
		\begin{equation}
		\label{M7Rvan}
			\begin{array}{l}
				\dd  (e^{3\Delta}Y)=\iota_{\xi} F\, , \\
				\dd  (e^{6\Delta}Z)= \iota_{\xi} \tilde{F} -e^{3\Delta} Y \wedge F\, .
			\end{array}
		\end{equation}
	%
As expected, these reproduce part of the supersymmetry equations in~\cite{Gabella:2012rc}.	
	
%

One can choose the gamma matrices and spinors in such a way that the Killing vector $\mathcal{K}$ has components~\cite{Gabella:2012rc} 
%
	\begin{equation}
		\begin{split}
			\mathcal{K}_0 & = \sum_i \bar{\psi}_i\rho_0 \psi_i\, , \\
			\mathcal{K}_m & = - \tfrac{i}{2} e^{2\Delta}\, \bar{\chi}^c_+ \gamma_m \chi_- \, .
		\end{split}
	\end{equation}
%


As in the previous section, the form of the generalised calibrations for $\mathrm{M}5$ and $\mathrm{M}2$ branes,~\eqref{genkcal} and~\eqref{genkcalM2}, depends on the direction spanned by the branes. 
Again, we considered first an $\mathrm{M}5$ wrapping a 5-cycle in $M_7$, with a zero world-volume gauge field ($H = A$) and in the static gauge. 

In this case, the relevant components of the forms~\eqref{11d2f} and~\eqref{11d5f} are 
%
	\begin{equation*}
		\begin{aligned}
			& \omega = \tfrac{i}{2} e^{3\Delta} \bar{\chi}_{+}^c \gamma_{(2)} \chi_{-} = e^{3\Delta} Y \, ,\\[1mm]
			& \Sigma = -e^{6\Delta}(\bar{\chi}_+ \gamma_{(5)} \chi_+^c + \bar{\chi}_-^c \gamma_{(5)} \chi_- ) = e^{6\Delta} Z\, ,
		\end{aligned}
	\end{equation*}
%
and the calibration $\Phi_{\mathrm{M}5}$ gives
%
	\begin{equation}
		\Phi_{\mathrm{M}5} = (e^{6\Delta} Z + A \wedge e^{3\Delta} Y + \tfrac{1}{2}\iota_\xi A \wedge A + \iota_{\xi}\tilde{A})\, .
	\end{equation}	
%
One can also add an $\mathrm{M}2$ completely arranged along the internal directions. 
The corresponding calibration form is given by,
%
	\begin{equation}
		\Phi_{\mathrm{M}2} = (e^{3\Delta} Y + \iota_{\xi} A )\, ,
	\end{equation}
%
which together with $\Phi_{\mathrm{M}5}$ gives rise to a poly-form,
%
	\begin{equation}
		\Phi = (e^{3\Delta} Y + \iota_{\xi} A ) + (e^{6\Delta} Z + A \wedge e^{3\Delta} Y + \tfrac{1}{2}\iota_\xi A \wedge A + \iota_{\xi}\tilde{A})\, ,
	\end{equation}
%
and this, again, corresponds to the vector structure. 
The closure of $\Phi$, follows from supersymmetry%
		\footnote{%
			Note that $L_K$ conditions imply that $\Phi_{\mathrm{M}5}$ and $\Phi_{\mathrm{M}2}$ are separately closed.%
			}%
			, more precisely, from the $L_K$ conditions~\eqref{M7Rvan},
%
		\begin{equation}
			\begin{split}
				\dd  \Phi &= \dd  (e^{3\Delta} Y + \iota_{\xi} A ) + \dd  (e^{6\Delta} Z + A \wedge e^{3\Delta} Y + \tfrac{1}{2}\iota_\xi A \wedge A + \iota_{\xi}\tilde{A}) \\[1mm]
						& = \iota_\xi F + \dd  (\iota_\xi A) + \iota_\xi \tilde{F} - e^{3\Delta} Y \wedge F + F \wedge e^{3\Delta} Y - A \wedge \iota_\xi F \\
						& \phantom{=} + \tfrac{1}{2} \dd  (\iota_\xi A) \wedge A + \tfrac{1}{2} \iota_\xi A \wedge F + \dd  (\iota_\xi \tilde{A}) \\[1mm]
						&= \mathcal{L}_\xi A + \mathcal{L}_\xi \tilde{A} + \tfrac{1}{2} \mathcal{L}_\xi A \wedge F + \tfrac{1}{2} A \wedge \iota_\xi F - A \wedge \iota_\xi F + \tfrac{1}{2}\iota_\xi A \wedge F = 0\, ,
			\end{split}	
		\end{equation}
%
where in the last line we made a gauge choice, such that,
%
	\begin{align}
		& & & \mathcal{L}_\xi A= 0\, , & & \mathcal{L}_\xi \tilde{A}= 0\, . & 
	\end{align}
%
%
	
One can consider not only branes wrapping internal cycles. 
For instance, for $\mathrm{M}5$-brane spanning two spatial directions, we can show that -- also in this case -- the related calibration form comes from the vector $K$. 
The relevant form is given by
	%
		\begin{equation}
			\Phi = (e^{3\Delta} Y + \iota_{\xi} A) \, .
		\end{equation}
	%
The closure of this form comes directly from~\eqref{M7Rvan}.

It is also easy to see that, in this case, branes with one leg aligned with an external space direction 
are not supersymmetric, as indeed already discussed in~\cite{SanchezLoureda:2005ap}.

Indeed, for instance in the case of an $\mathrm{M}2$ wrapping an internal cycle and with one external leg, the candidate calibration form is proportional to $\sigma$ in~\eqref{bilrel}, which is not closed -- \emph{i.e.} $\dd  \sigma \sim \omega$. 
This condition, in the language of Exceptional generalised geometry, is a part of $ L_K J_a = \epsilon_{abc}  \lambda_b J_c $. 
It is interesting to note that, on the other hand, this configuration is supersymmetric in the case of a Minkowski background, since the analogous condition reads $L_K J_a = 0$,~\cite{AshmoreECY}.

To conclude the analysis, let us focus on space filling brane configurations. 
This case corresponds to the $J_a$ components of the Exceptional Sasaki-Einstein structure. 
For instance, for an $\mathrm{M}5$-brane we find
	%
		\begin{equation}
		\label{q3cal}
			\Phi = -e^{4\Delta}{V}_-\, ,
		\end{equation}
	%
where $V_{-}$ is the two-form defined from spinor bilinears as follows,
	%
		\begin{equation}
		\label{Vbil}
			V_{\pm} \coloneqq \dfrac{1}{2i} \left( \bar{\chi}_+ \gamma_{(2)}\chi_+ \pm \bar{\chi}_- \gamma_{(2)}\chi_-\right)\, ,
		\end{equation}
	%
	which gives the $TM \otimes T^*M$-component of $J_a$ by raising one index. 
	In particular, in the limit of a Sasaki-Einstein manifold (the only one for which the expression of $J_a$ is given explicitly in~\cite{AshmoreESE}), the calibration form~\eqref{q3cal} corresponds to $J_3$, and we have good reasons to trust this result also for the cases where generic fluxes are turned on. 
	We leave the complete discussion of these cases for future work.

\section{Supersymmetric branes in type IIB}
\label{sec:IIBcal}
%
In this section, we want to discuss the analogous conditions to have supersymmetric extended objects in a type IIB supergravity AdS background and their connections to Exceptional Sasaki-Einstein structures defining such backgrounds.

We are interested in backgrounds with non trivial fluxes. The NS three-form is $H = \dd  B$ and the RR fields are 
%
	\begin{equation} 
	 F_1=\dd  C_0 \qquad 
	 F_3=\dd  C_2 \qquad 
	 F_5=\dd  C_4 - \frac{1}{2} H \wedge C_2 + \frac{1}{2} F_3 \wedge B
	\end{equation} 
%	
The field strengths $F$ satisfy the duality condition
%
	\begin{equation} 
	\label{fdual}
		F_p = (-1)^{\left[\tfrac{p}{2}\right]} \star F_{10-p}\, , 
	\end{equation} 
%
while the S-duality of type IIB is reflected in the fact that $B$ with $C_2$ form an $SL(2)$ doublet. It could be useful to define a complexified version of the three-form flux~\cite{Schwarz:1983qr},
%
	\begin{equation} 
		G = F_3 + i H_3\, .
	\end{equation} 
%
The Bianchi identities are written as
%
	\begin{align}
		& &	\dd  F_5= \frac{1}{8}\, \IIm\, G\wedge G^*\, ,
			& & 
		\dd  G= 0\, . & & 
	\end{align}
%

The generalised calibrations for backgrounds with non-trivial NS-NS flux have been constructed in~\cite{Martucci:2011dn} (see also~\cite{Cascales:2004qp, HPS04} for an equivalent derivation in terms of the supersymmetry algebra). 
%
The two Majorana-Weyl supersymmetry parameters $\varepsilon_1$ and $\varepsilon_2$ can be used to construct the following bilinears~\cite{BJRT99},
%
	\begin{align}
	%
		\label{K10}
		\mathcal{K} &= \dfrac{1}{2} ( \bar{\varepsilon}_1 \Gamma_M \varepsilon_1 + \bar{\varepsilon}_2 \Gamma_M \varepsilon_2)\ \dd  x^M\, , \\
		%
		\label{omega10}
		\omega &= \dfrac{1}{2} ( \bar{\varepsilon}_1 \Gamma_M \varepsilon_1 - \bar{\varepsilon}_2 \Gamma_M \varepsilon_2)\ \dd  x^M\, , \\
		%
		\label{psiIIB}
		\Psi &= \sum_{k=0}^2 \dfrac{1}{(2 k+ 1)!} \bar{\varepsilon}_1 \Gamma_{M_1 \ldots M_{2k +1} } \varepsilon_2 \ \dd  x^{M_1} \wedge \ldots \wedge \dd  x^{M_{2k +1}} \, . 
	%
	\end{align}
%
Using the Killing spinor equations for type IIB, one can show that the vector $\hat{\mathcal{K}}$ dual to the one form $\mathcal{K}$ is a Killing vector~\cite{Tomasiello:2011eb},
%
	\begin{equation} 
		\begin{array}{ccc}
			\mathcal{L}_{\hat{\mathcal{K}}} g = 0\, ,& \phantom{and}& \mathcal{L}_{\hat{\mathcal{K}}} F = 0 \, . 
		\end{array}
	\end{equation} 
%
Notice that also the spinor bilinears~\eqref{K10}--\eqref{psiIIB} are invariant under the transformation generated by~$\hat{K}$
%
	\begin{equation} 
		\begin{array}{ccc}
			\mathcal{L}_{\hat{\mathcal{K}}} \omega = 0\, , & \phantom{and} & \mathcal{L}_{\hat{\mathcal{K}}} \Psi = 0 \, .
		\end{array}
	\end{equation} 
%

As discussed in~\cite{Martucci:2011dn}, we may write the $\kappa$-symmetry condition to have a supersymmetric $\mathrm{D}p$-brane 
%
	\begin{equation} 
		\hat{\Gamma}_{\mathrm{D}p}\ \varepsilon_2 =\varepsilon_1\, ,
	\end{equation} 
%
where, the $\kappa$-symmetry operator is defined as~\cite{BT97, Marolf:2003vf} 
%
	\begin{equation} 
	\label{eq:brane_kappa}
		\hat{\Gamma}_{\mathrm{D}p}=\frac{1}{\sqrt{-\det\left( P[G]+\mathcal{F} \right)}} \sum_{2l+s=p+1}\frac{\epsilon^{\alpha _1 \ldots \alpha _{2l}\beta_1 \ldots \beta_s}}{l!s!2^l} \mathcal{F}_{\alpha _1\alpha _2} \ldots \mathcal{F}_{\alpha _{2l-1}\alpha _{2l}} \Gamma_{\beta_1 \ldots \beta_s}\, ,
	\end{equation} 
%
and $P[\bullet]$ denotes the pullback to the $(p+1)$-dimensional brane world-volume and $\mathcal{F}= F + P[B]$, with $B$ the NS two-form and $F$ the world-volume gauge field-strength.

The energy of the brane (the charge associated to the transformation generated by $\hat{\mathcal{K}}$) is
%
	\begin{equation} 
		E = - \int_{\mathcal{S}}\ \dd ^p \sigma\ \hat{P}_M \hat{\mathcal{K}}^M \, ,
	\end{equation} 
%
where $\mathcal{S}$ is the brane world-space and $\hat{P} = \tfrac{\partial L_{\mathrm{D}p}}{\partial (\partial_\tau X^M)}$ takes the form
%
	\begin{equation} 
		\hat{P}_M = - \mu_{\mathrm{D}p} e^{- \phi} \sqrt{- \det \mathcal{M}} (\mathcal{M}^{-1})^{( \alpha  \tau)} B_{MN} \partial_\alpha  X^N 
			+ \frac{\mu_{\mathrm{D}p}}{p!} \epsilon^{\tau \alpha _1 \ldots \alpha _p} \left[ \iota_M (C \wedge e^\mathcal{F}) \right]_{\alpha _1 \ldots \alpha_p} \, ,
	\end{equation} 
%
where we denoted $\mathcal{M} = P[g] + \mathcal{F}$. 
Note that again we are in the temporal gauge in adapted coordinates, such that the world-volume of the brane is $\mathbb{R}\times \mathcal{S}$.
%
%
%
One has the usual BPS bound,
%
	\begin{equation} 
		E \geq E_{BPS}\, ,
	\end{equation} 
%
with
%
	\begin{equation} 
		\label{BPScal}
			\begin{split}
				E_{BPS} =\mu_{\mathrm{D}p} \int_\mathcal{S} \dd ^p \sigma\, &P\left[e^{- \phi} \Psi - \iota_{\hat{\mathcal{K}}} C - \omega \wedge C \right] \wedge e^{\mathcal{F}} \\
					&+ \mu_{\mathrm{D}p}\int_{\mathcal{S}}\dd ^p \sigma\, P\left[\omega-\iota_{\hat{\mathcal{K}}}B\right] \wedge \left.\left(C \wedge e^{\mathcal{F}}\right)\right\vert_{p-1} \, .
			\end{split}
	\end{equation} 
%
Thus, one can read the generalised calibration form from the last expression,
%
	\begin{equation} 
	\label{BPScal2}
		\Phi_{\mathrm{D}p} = e^{- \phi} \Psi - \iota_{\hat{\mathcal{K}}} C - \omega \wedge C \wedge e^{\mathcal{F}} + \omega-\iota_{\hat{\mathcal{K}}}B \wedge \left.\left(C \wedge e^{\mathcal{F}}\right)\right\vert_{p-1}\, .
	\end{equation} 
%
One can show~\cite{Martucci:2011dn, Evslin:2007ti} that this is a topological quantity.
In addition, one can also show that this form is closed, making use of potential configurations preserving the symmetry generated by $\hat{\mathcal{K}}$, \emph{i.e.}
%
	\begin{align}
	\label{vanpot}
		&&\mathcal{L}_{\hat{\mathcal{K}}} B = 0\, , & &\mathcal{L}_{\hat{\mathcal{K}}} C = 0\, ,&&
	\end{align}
%
analogously to what has been done in the previous section for M-theory. 
As a final observation, we would like to point out that the same conclusions about calibration forms can be obtained by supertranslation algebra, as done for example in~\cite{Cascales:2004qp, Gutowski:1999tu,HPS04}. \\

Let us now focus on type IIB compactifications to $\mathrm{AdS}_5$-backgrounds. 
As for the discussion of supersymmetric extended objects in M-theory above, we now apply the supersymmetry conditions and the aforementioned approach to branes in type IIB string theory on 
%
	\begin{equation} 
	\label{eq:metric_IIB_ads5}
		\dd  s^2= e^{2\Delta} \dd  s^2_{\mathrm{AdS}_5} + \dd  s^2_{M_5}\, ,
	\end{equation} 
%
and relate the calibration forms to the geometric description by the vector and hypermultiplet structures. 
%
%
The exceptional geometry of this setup is discussed in~\cite{AshmoreESE,Grana_Ntokos}, based on the geometric description in~\cite{Gauntlett:2005ww}. 
For $\mathcal{N}=2$ backgrounds of the form~\eqref{eq:metric_IIB_ads5} with generic fluxes, the internal manifold $m_5$ admits a (local) identity structure~\cite{Gauntlett:2005ww,GGPSW09, GGPSW09_02}. 

%
%
%
%\subsection{Exceptional Sasaki-Einstein structure}
%\label{sect:IIB_ESE}
%
The two ten-dimensional Majorana-Weyl spinors of the same chirality which describe a IIB background of the form~\eqref{eq:metric_IIB_ads5} can be decomposed as in~\cite{Grana_Ntokos}%
		\footnote{%
		We follow the conventions given in the appendix of~\cite{Grana_Ntokos}.
		We collect them in~\cref{sect:IIB_notation}.
		},
%	
	\begin{equation} 
	\label{eq:splitting_IIB_ads5}
		\varepsilon_i= \psi \otimes \chi_i \otimes u + \psi^c \otimes \chi_i^c \otimes u\, . 
	\end{equation} 
	%
Here $\psi$ denotes the external $\mathrm{Spin(4,1)}$ spinor, $\chi_i$ are the internal $\mathrm{Spin}(5)$ spinors and $u$ a two-component spinor. 
It might be convenient to define the complex spinors $\zeta_1 = \chi_1 + i \chi_2$ and $\zeta_2^c = \chi_1^c + i \chi_2^c$.
As for the previous cases, one can construct the relevant bilinears defining a local identity structure on the internal manifold~\cite{Gauntlett:2005ww}. 
One introduces the vectors
%
\begin{equation} 
\label{IIBbil}
	\begin{aligned}
 		K_0^m &:= \bar{\zeta}_1^c\gamma^m\zeta_2\, , \\
 		K^m_3 &:= \bar{\zeta}_2\gamma^m\zeta_1 \, ,\\
 		K^m_4 &:= \tfrac{1}{2}\left(\bar{\zeta}_1\gamma^m\zeta_1 - \bar{\zeta}_2\gamma^m\zeta_2\right)\, , \\
 		K^m_5 &:= \tfrac{1}{2}\left(\bar{\zeta}_1\gamma^m\zeta_1 + \bar{\zeta}_2\gamma^m\zeta_2\right)\, ,
\end{aligned}
\end{equation} 
%
that are not all linear independent. 
Relations between these forms comes from supersymmetry, as shown in~\cite{Gauntlett:2005ww}. 
Then, one can use the following scalars to parametrise the norms of the spinors
%
\begin{equation} 
\label{scalarbilIIB}
	\begin{aligned}
 		A &:= \tfrac{1}{2}\left(\bar{\zeta}_1\zeta_1 + \bar{\zeta}_2\zeta_2\right)\, , \\
		A\sin\Theta &:= \tfrac{1}{2}\left(\bar{\zeta}_1\zeta_1 - \bar{\zeta}_2\zeta_2\right)\, , \\
		S &:= \bar{\zeta}^c_2\zeta_1\, , \\
		Z &:= \bar{\zeta}_2\zeta_1\, .
	\end{aligned}
\end{equation} 
%
Finally, one considers the two-forms
%
\begin{equation} 
\label{IIB2formsbil}
	\begin{aligned}
		U_{mn} &:= -\tfrac{i}{2}\left(\bar{\zeta}_1\gamma_{mn}\zeta_1 + \bar{\zeta}_2\gamma_{mn}\zeta_2\right)\, , \\
		V_{mn} &:= -\tfrac{i}{2}\left(\bar{\zeta}_1\gamma_{mn}\zeta_1 - \bar{\zeta}_2\gamma_{mn}\zeta_2\right)\, , \\
		W_{mn} &:= -\bar{\zeta}_2\gamma_{mn}\zeta_1\, .
	\end{aligned}
\end{equation} 
%

The HV structure for these backgrounds can be found in \cite{AshmoreESE, Grana_Ntokos}. 
The untwisted generalised vector structure 
$K \in \Gamma(\tilde{E})$ in \eqref{genvecs} is given in terms of the identity structure above by 
%
	\begin{equation} 
	\label{eq:K_tilde_IIB}
		\begin{aligned}
			\tilde{K}= \tilde{\xi} + \tilde{\lambda}^i +\tilde{\rho} + \tilde{\sigma}^i = K_5^{\sharp} +e^{2\Delta - \frac{\phi}{2} } \begin{pmatrix} \RRe K_3 \\ \IIm K_3 \end{pmatrix} - e^{4\Delta - \phi} \star V \, ,
		\end{aligned}
	\end{equation} 
%
where $\phi$ is the dilaton and $\Delta$ the warp factor.
Notice that for these backgrounds the five-forms vanish $\sigma^i =0$. 
The twisted vector structure is obtained by acting on $\tilde K$ with the adjoint element as in appendix E of~\cite{AshmoreECY}
%	
	\begin{equation} 
	\label{eq:IIB_twisted_explicit}
	K = \xi + \lambda^i + \rho \, ,
	\end{equation} 
%
where the twisted quantities are~\cite{AshmoreESE,AshmoreECY}
%
	\begin{subequations}
	\label{IIbtwist}
		\begin{align}
			\xi &= \tilde{\xi}\, , \label{IIbtwistv} \\[1mm]
			\lambda^i & = \tilde{\lambda}^i + \iota_{\xi}B^i\, , \label{IIbtwist1} \\[1mm]
			\rho & = \tilde{\rho} + \iota_\xi C + \epsilon_{ij}\tilde{\lambda}^i \wedge B^j + \tfrac{1}{2}\epsilon_{ij}\left( \iota_\xi B^i \right)\wedge B^j\, . \label{IIbtwist3}%\\[1mm]
		\end{align}
	\end{subequations}
%
and we defined $B^1 = B$, $B^2 = C_2$, $C = C_4$, $F^1= H$, $F^2 = F_3$ and $F=F_5$.

As already discussed the condition that the generalised Lie derivative $L$ along the Reeb vector $K$ has to reduce to the conventional one, $\mathcal{L}_{\xi}$ implies some differential equations on the elements of the vector structure that reproduce some of the supersymmetry conditions on the identity structure derived in~\cite{Gauntlett:2005ww},
%
	\begin{equation} 
	\label{eq:IIB_ESE_structure_eqs}
		\begin{split}
			\dd  \tilde{\lambda}^i&= \iota_{\xi} F^i\, ,\\
			\dd  \tilde{\rho} &= \iota_{\xi} F +\epsilon_{ij} \tilde{\lambda}^i \wedge F^j\, .
		\end{split}
	\end{equation} 
%
%======================================================================
%======================================================================
%++++++++++++++++++++++++++++++++++++++++++++++++++++++++++++++++++++++
%++++++++++++++++++++++++++++++++++++++++++++++++++++++++++++++++++++++
%======================================================================
%======================================================================
%
%\subsection{\texorpdfstring{$\kappa$-symmetry and brane calibrations}{k-symmetry and brane calibrations}}
%%
%\label{sec:kappa_symmetry_IIB}

Analogously to the M-theory case, we want now to express the calibration conditions for a D$p$ probe in these backgrounds in terms of the generalised structure and check that their closure in implied by differential conditions on Exceptional Sasaki-Einstein structures. To this purpose we have to specialise the calibrations \eqref{K10}, \eqref{omega10} and \eqref{psiIIB} to the various brane configurations.
The $\mathrm{AdS}_5$ geometry and, in particular the products of external spinors, is the same as in the previous section. Thus, in our conventions, the Killing vector $\mathcal{K}$ has the following components,
%
	\begin{equation} 
		\begin{split}
			\mathcal{K}_0 &= e^{-2\Delta} \bar{\psi} \rho_0 \psi \otimes A \, , \\
			\mathcal{K}_m & = \tfrac{1}{2}\left(\bar{\zeta}_1\gamma^m\zeta_1 + \bar{\zeta}_2\gamma^m\zeta_2\right) = \xi_m \, ,
		\end{split}
	\end{equation} 
%
where $\xi$ is the Reeb vector. 
We also fix the norm of the internal spinors such that $A=1$.

As in the previous sections, we focus on the cases of point-like AdS particles and space-filling branes where the calibrations are related to the generalised vector $K$. 
Consider first a D$1$ wrapping an internal one-cycle. 
The relevant terms is \eqref{BPScal2} are 
%
	\begin{equation} 
		\begin{aligned}
			&\omega =- e^{2\Delta-\phi/2} \tfrac{1}{2} \left(\bar{\zeta}_2 \gamma_m \zeta_1 + \zeta_2^T\gamma_m \zeta_1^* \right) = - e^{2\Delta-\phi/2} \RRe K_3\, , \\
			&\Psi =- e^{2\Delta+\phi/2}\tfrac{1}{2i} \left(\bar{\zeta}_2 \gamma_m \zeta_1 - \zeta_2^T\gamma_m \zeta_1^* \right) = - e^{2\Delta+\phi/2} \IIm K_3\, , \\
			&\iota_{\hat{\mathcal{K}}} B = \iota_\xi B^1 =: \iota_\xi B\, , \\
			&\iota_{\hat{\mathcal{K}}} C = \iota_\xi B^2\, .
		\end{aligned}
	\end{equation} 
%
and it is immediate to see that the calibration form is given by the generalised vector $K$
%
	\begin{equation} 
		\Phi_{\mathrm{D}1} = -\tilde{\lambda}^2 -\iota_\xi B^2 = -e^{2\Delta-\phi/2} \IIm K_3 -\iota_\xi B^2\, .
	\end{equation} 
	%
Using equation~\eqref{eq:IIB_ESE_structure_eqs} one can show that $\Phi_{\mathrm{D}1}$ is closed. 
Using again the properties of the $\mathrm{AdS}_5$ spinors, it is easy to show that a space-filling D$5$-brane is also calibrated by the same form,
%
	\begin{equation} 
		\Phi_{\mathrm{D}5} = ( -\tilde{\lambda}^2 -\iota_\xi B^2) \otimes \star 1 =( -\tilde{\lambda}^2 -\iota_\xi B^2) \otimes \vol_{\mathrm{AdS}_5} \, .
	\end{equation} 
%

Similarly, one can find the calibration form for a D$3$-brane wrapping purely internal cycles,
%
	\begin{equation} 
	\label{D3cal}
		\Phi_{\mathrm{D}3}= \tilde{\rho} +\iota_\xi C +\epsilon_{ij}\tilde{\lambda}^i \wedge B^j + \tfrac{1}{2}\epsilon_{ij}\iota_\xi B^i \wedge B^j\, .
	\end{equation} 
%
Its closure follows from the $L_K$ conditions under the gauge choice%
		\footnote{%
		We also need to use $\tilde{\rho} \propto \star V$.%
		} that the potential are invariant under $K$.
Again, this form provides also the calibration for a space-filling D$7$-brane.

In the particular case where the only non-trivial background flux is the five-form, the generalised Sasaki-Einstein structure reduces to the standard one and the internal manifold is Sasaki-Einstein.
In this case the spinor ansatz~\eqref{eq:splitting_IIB_ads5} simplifies since the two internal spinors are proportional to each other, \emph{i.e.} $\chi_2=i \chi_1$, and, consequently, the one-form part vanishes.
The twisteed vector~\eqref{eq:IIB_twisted_explicit} simplifies to
%
	\begin{equation} 
		K = \xi - \sigma \wedge \omega + \iota_{\xi}C\, , 
	\end{equation}  
%
and the (untwisted) hypermultiplet structure is~\cite{AshmoreESE}
%
	\begin{align}
	\label{Jstruct}
		\tilde{J}_+ &= \tfrac{1}{2}\kappa u^i \Omega - \tfrac{i}{2} \kappa u^i \Omega^{\sharp}\, ,\\[1mm]
		\tilde{J}_3 &= \tfrac{1}{2}\kappa I + \tfrac{1}{2} \kappa \hat{\tau}^{i}_{\phantom{i}j} + \tfrac{1}{8}\kappa \Omega^{\sharp}\wedge \bar{\Omega}^{\sharp} - \tfrac{1}{8}\kappa \Omega \wedge \bar{\Omega} \, , 
	\end{align}
	%
where $u^i = (-i, 1)^i$ and $I$, $\omega$ and $\Omega$ are the complex structure, the symplectic and the holomorphic two-forms on the K\"ahler-Einstein basis of $M_5$. 

Note that in the vector structure, the three-form is $\sigma \wedge \omega$ and by the structure equation~\eqref{eq:IIB_ESE_structure_eqs} one immediately sees that adding the potential part~$\iota_{\xi}C$ yields to a closed form (up to a gauge choice),
%
	\begin{equation} 
		\dd  \left( \tilde{\rho}+\iota_{\xi}C \right)= \iota_{\xi} \dd  C + \mathcal{L}_{\xi} C - \iota_{\xi} \dd  C = \mathcal{L}_{\xi} C = 0\, .
	\end{equation} 
%

Since in this case, the form of the hypermultiplet structure is simple, we can also study calibrations that are not associated to the vector $K$. 
We will do it in the simplest Sasaki-Einstein background, namely $\mathrm{AdS}_5 \times S^5$, where $S^5$ is the five-dimensional sphere. 
The background is given by
%
	\begin{equation} 
	\label{ads5bkgrd}
		\begin{split}
			\dd  s^2 &= \dfrac{R^2}{r^2} \dd  r^2 + \dfrac{r^2}{R^2} \eta_{\mu\nu} \dd  x^\mu \dd  x^\nu + \dd  s^2(S^5)\, , \\[1mm]
			C_4 &= \left(\dfrac{r^4}{R^4} - 1\right) \dd  x^0 \wedge \ldots \wedge \dd  x^3\, .
		\end{split}
	\end{equation} 
%
while all other fluxes, dilaton and warp factors vanish.
The $S^5$ can be written as a $U(1)$ fibration over $\mathbb{CP}^2$,
%
	\begin{equation} 
		\dd  s^2(S^5) = \dd  \Sigma_{4}^2 + \sigma \otimes \sigma\, ,
	\end{equation} 
%
where $\dd  \Sigma_{4}^2$ is the \emph{Fubini-Study metric} over $\mathbb{CP}^2$. The form $\sigma$ is given by $\sigma = \dd  \psi + A$, where $A$ is a connection such that $\mathcal{F} = \dd  A = 2 \omega$, and $\psi$ is the periodic coordinate on the circle $U(1)$ with period $6\pi$.

Explicitly, the sphere $S^5$ takes the form~\cite{Gauntlett:2004yd}
%
	\begin{equation} 
		\begin{split}
			\dd  s^2(S^5) &= \dd  \alpha ^2 + \dfrac{1}{4} \sin^2 \alpha  (\dd  \theta^2 + \sin^2 \theta \dd  \phi^2) + \dfrac{1}{4} \cos^2 \alpha  \sin^2\alpha  (\dd  \beta + \cos\theta \dd  \phi)^2 \\
				 & \phantom{=} + \dfrac{1}{9}\left[\dd  \psi - \dfrac{3}{2}\sin^2\alpha  (\dd  \beta + \cos\theta \dd  \phi)\right]^2\, ,
		\end{split}
	\end{equation} 
%
with $\psi \in [0,6\pi]$, $\beta \in [0,4\pi]$, $\alpha  \in [0,\pi/2]$, $\theta \in [0,\pi]$ and $\phi \in [0,2\pi]$.
In these coordinates the holomorphic form has the following expression,
%
	\begin{equation} 
		\Omega= - \frac{1+i}{\cos \theta}\ \dd  \beta \wedge \dd  \theta + \frac{1+i}{8} \cos \sigma \cos \theta \sin \theta \sin^3 \sigma\ \dd  \sigma \wedge \dd  \phi + (1+i)\ \dd  \theta \wedge \dd  \phi\, .
	\end{equation} 
%

First, consider a D$5$-brane spanning the directions $0,1,2, r$ in $\mathrm{AdS}_5$. 
The world-volume of the brane is $\mathrm{AdS}_4 \times S^2$, where $S^2$ is the sphere parametrized by the angles $(\theta, \phi)$. 
Then the expression~\eqref{BPScal2} reduces to 
%
	\begin{equation} 
		\begin{split}
			\Phi_{\mathrm{D}5} &= -\ \dd  x^0 \wedge \dd  x^1 \wedge \dd  x^2 \wedge \dd  x^4 \wedge \tfrac{(1-i)}{2} e^{4\Delta}\Omega \\
				& = -\ \dd  x^0 \wedge \dd  x^1 \wedge \dd  x^2 \wedge \dd  r \wedge \vol_{S^2}\, ,
		\end{split}
	\end{equation} 
%
and we see that it corresponds to the two-form part of $\tilde{J}_+$ in~\eqref{Jstruct}.
Modulo choice of coordinates,%
		\footnote{%
		Here $\dd  x^4 \propto \sin \theta\ \dd  r + \dd  \sigma + \dd  \beta$.%
		} it agrees with the analogous form in~\cite{Cascales:2004qp}.

We can also consider a D$3$-brane probe spanning the directions $0,1,r $ of $AdS_5$. 
The world-volume is now $\mathrm{AdS}_3 \times S^1$ and the calibration is given by the Hodge dual of the $4$-form part of $\tilde{J_3}$,
%
	\begin{equation} 
		\Phi_{\mathrm{D}3} = \dd  x^0 \wedge \dd  x^1 \wedge \dd  r \wedge \tfrac{1}{8} e^{4\Delta} \star( \Omega \wedge \bar{\Omega})\, .
	\end{equation} 
%
One can prove the closure of this form by the $L_K J$ relations. In particular,
%
	\begin{equation} 
		\dd  (e^{4\Delta} \star ( \Omega \wedge \bar{\Omega})) = - m\ \iota_\xi \vol_5 = 0\, ,
	\end{equation} 
%
where the first equality comes from the conditions to have a vanishing $\tilde{R}$-tensor~\cite{AshmoreECY,AshmoreESE}. 
In other words, it is the rewriting of the~\eqref{eq:IIB_ESE_structure_eqs} in the Sasaki-Einstein case.


					%
				%
			%
		%
\end{document}