\documentclass[debug]{phd}

\begin{document}
	%
	\chapter{Exceptional Generalised Geometry}
	\label{app:EGG}
	%	%
		%
		\section{Generalised geometry for M-theory}\label{appsec:EGGMth}
			%
			We review the construction of $\E_{d(d)} \times \RR^+$ given in~\cite{hull1, waldram2, waldram4}.
			This will also be useful since by a dimensional reduction we will build the appropriate geometry for type IIA.
			We use the same notation and conventions as~\cite{waldram2, waldram4}. 
			
			In M-theory compactified on a seven-dimensional manifold $M_7$, the fibres of the generalised tangent bundle $E$ transform in the fundamental $\mathbf{56}_1$ representation of the $E_{7(7)}\times\mathbb R^+$ structure group. 
			Under $\GL(7)$, $E$ decomposes as
				%
					\begin{equation}\label{gense7}
 						E \cong TM_7 \oplus \Lambda^2T^*M_7 \oplus \Lambda^5T^*M_7 \oplus (T^*M_7\otimes\Lambda^7T^*M_7)\, .
					\end{equation} 
				%
			A section can be written as 
				%
					\begin{equation}
 						V = v + \omega + \sigma + \tau\, ,
					\end{equation}
				%
			where at each point on $M_7$, $v \in \Gamma(TM_7)$ is an ordinary vector field, $\omega\in \Gamma(T^*M_7)$, $\sigma \in \Gamma(\Lambda^5T^*M_7)$ and $\tau \in \Gamma((T^*\otimes \Lambda^7 T^*)M_7)$.
			
			The adjoint bundle $\ad$ decomposes under $\GL(7)$ as
				%
					\begin{equation}
						\adj = \RR \oplus (TM_7\otimes T^*M_7) \oplus \Lambda^3 T^*M_7 \oplus \Lambda^6 T^*M_7 \oplus \Lambda^3 TM_7 \oplus \Lambda^6 TM_7 \, ,
					\end{equation}
				%
			with sections transforming in the $\mathbf{133}_0+ \mathbf{1}_0$ representation of $\E_{7(7)}\times \RR^+$ given by
				%
					\begin{equation}\label{eq:Radj}
						R = l + r + a + \tilde a + \alpha + \tilde\alpha \ ,
					\end{equation}
				%
			where $l \in \RR$ gives the shift of the warp factor, $r \in \mathrm{End}(T M_7)$, $a \in \Lambda^3 T^*M_7$ is related to the three-form potential of M-theory, $\tilde a \in \Lambda^6T^*M_7$ to its dual, while $\alpha \in \Lambda^3 TM_7$ and $ \tilde\alpha \in \Lambda^6 TM_7$ are a three- and a six-vector.
				
			The adjoint action of the $\mathfrak{e}_{7(7)}\times \RR^+$ algebra on a generalised vector is denoted as $V' = R \cdot V$ and reads:
				%
					\begin{equation}\label{Mth_adjoint_act}
						\begin{split}
							v' &= l\ v + r \cdot v + \alpha \lrcorner \omega - \tilde{\alpha} \lrcorner \sigma \, ,\\
							\omega' &= l \, \omega + r \cdot \omega +v \lrcorner a+ \alpha \lrcorner \sigma + \tilde{\alpha} \lrcorner \tau \, \\
							\sigma' &= l \,\sigma + r \cdot \sigma +v \lrcorner \tilde a + a \wedge \omega+ \alpha \lrcorner \tau \, , \\
							\tau'&= l\, \tau+ r \cdot \tau - j \tilde a \wedge \omega+ j a \wedge \sigma \, .
						\end{split}
					\end{equation}
			The $\mathfrak{e}_{7(7)}$ subalgebra is given by imposing $ \frac{1}{2} \tr(r) = l$.
			
			The adjoint commutator $R''= [R , R']$ is
				%
					\begin{equation}\label{comm_Mth_adj}
						\begin{split}
							l'' &= \tfrac{1}{3} (\alpha \lrcorner a' - \alpha' \lrcorner a) + \tfrac{2}{3} (\tilde \alpha' \lrcorner \tilde a - \tilde \alpha \lrcorner \tilde a') \, , \\
							r'' &= [ r, r'] + j \alpha \lrcorner j a' - j \alpha' \lrcorner j a - \tfrac{1}{3} (\alpha \lrcorner a' - \alpha' \lrcorner a) \id \\
								& \phantom{=} + j \tilde \alpha' \lrcorner j \tilde a - j \tilde \alpha \lrcorner j \tilde a' - \tfrac{2}{3} (\tilde \alpha' \lrcorner \tilde a - \tilde \alpha \lrcorner \tilde a')\id \, ,\\
							a'' &= r \cdot a' - r' \cdot a + \alpha' \lrcorner \tilde a - \alpha \lrcorner \tilde a' \, , \\
							\tilde a'' &= r \cdot \tilde a' - r' \cdot \tilde a - a \wedge a' \, , \\
							\alpha'' &= r \cdot \alpha' - r' \cdot \alpha +\tilde \alpha' \lrcorner a - \tilde \alpha \lrcorner a' \, , \\
							\tilde \alpha'' &= r \cdot \tilde \alpha' - r' \cdot \tilde \alpha - \alpha \wedge \alpha' \, .
						\end{split}
					\end{equation}
				%

			 As seen in the main text, the generalised tangent bundle $E$ is actually twisted to take into account the non-trivial gauge potentials of M-theory, and this is why it is only isomorphic to the sum of bundles in~\eqref{gense7}.  The twist is implemented by an action by adjoint elements.
			 Given a section $\tilde{V}$ of the untwisted tangent bundle $\tilde{E}$, a section $V$ of $E$ is defined as
				%
					\begin{equation}\label{twist_Mth}
						V = e^{A + \tilde A}  \cdot \tilde{V} \, , 
					\end{equation}
				%
			where $A + \tilde A$ is an element of the adjoint bundle.
			The patching condition on the overlaps $U_{\alpha} \cap U_{\beta}$ is 
				%
					\begin{equation}
						V_{(\alpha)} = e^{\dd \Lambda_{(\alpha \beta)} + \dd \tilde \Lambda_ {(\alpha \beta)}} \cdot V_{(\beta)} \, , 
					\end{equation}
				%
			where $\Lambda_{(\alpha \beta)}$ and $ \tilde\Lambda_ {(\alpha \beta)}$ are a two- and five-form, respectively. 
			This corresponds to gauge-transforming the three- and six-form potentials in~\eqref{twist_Mth} as 
				%
					\begin{equation}
						\begin{split}
							A_{(\alpha)} &= A_{(\beta)} + \dd \Lambda_{(\alpha \beta)} \, ,  \\
							\tilde A_{(\alpha)} &= \tilde A_{(\beta)} + \dd \tilde \Lambda_{(\alpha \beta)}  -\frac{1}{2}   \dd \Lambda_{(\alpha \beta)}  \wedge A_{(\beta)} \, .
						\end{split}
					\end{equation}
				%
			Then, the respective gauge-invariant field-strengths reproduce the supergravity ones are
				%
					\begin{equation}
						\begin{split}
								F &= \dd A \, ,  \\
							\tilde F &= \dd \tilde A - \frac{1}{2} A \wedge F \, .
						\end{split}
					\end{equation}
				%
				
			Finally, we discuss the relevant representations of $\E_{d(d)}\times \RR^+$, defining generalised tensors.
			Further than the vector one, there are other tensor bundles which will be of particular importance in the construction of consistent truncations, we collect some of them in~\cref{tab:EddRep}.
			
			Amongst the interesting bundle representations, an object we will need is the bundle $N$ first introduced in~\cite{Coimbra:2011ky}. 
			This is a sub-bundle of the symmetric product of two generalised tangent bundles, $N \subset S^2 E$, and can be expressed as
				%
				\begin{equation}\label{MthNbundle}
					\begin{split}
						N \cong & T^*M_7 \oplus \Lambda^4 T^*M_7 \oplus (T^*M_7 \otimes \Lambda^6T^*M_7) \\[1mm]
						& \oplus (\Lambda^3T^*M_7\otimes \Lambda^7T^*M)\oplus (\Lambda^6T^*M_7 \otimes \Lambda^7T^*M_7) \,.
					\end{split}
				\end{equation}
				%
Formally, $N$ can be described via a series of exact sequences
%
\begin{equation}
\label{eq:N-sequences}
\begin{tikzcd}[row sep=tiny]
										0 \arrow{r} &\Lambda^4 T^*M \arrow{r} &N' \arrow{r} &T^*M \arrow{r} &0 	\, , \\
										0 \arrow{r} & T^*M\otimes \Lambda^6T^*M  \arrow{r} &N'' \arrow{r} &N' \arrow{r} &0 	\, , \\
										0 \arrow{r} &\Lambda^7 T^*M \otimes \Lambda^3 T^*M \arrow{r} &N''' \arrow{r} &N'' \arrow{r} &0 \, . \\
										0 \arrow{r} &\Lambda^7 T^*M \otimes \Lambda^6 T^*M \arrow{r} &N \arrow{r} &N''' \arrow{r} &0 \, .
									\end{tikzcd}
\end{equation}
%
Under $E_{7(7)}\times \mathbb{R}^+$, sections of $N$ transform in the $\mathbf{133}_{2}$ representation. 
Their expression in terms of the symmetric product of generalised vectors can be found in~\cite{Coimbra:2011ky}. 

The simplest of the intermediate bundles appearing in~\eqref{eq:N-sequences} is $N'$, whose type IIA counterpart will be relevant for the scopes of this paper. This can be expressed as
\begin{equation}\label{MthN'}
N' \simeq  T^*M_7 \oplus \Lambda^4 T^*M_7\,.
\end{equation}
Given a basis $\{\hat E_A\}$, $A = 1,\ldots, 56$, for the generalised tangent bundle $E$, a section $S$ of $N'$ has the form
\begin{equation}
S \,=\, S^{AB}\hat E_A \otimes_{N'} \hat E_B \,,
\end{equation}
where $S^{AB}$ are functions on the manifold and the map $\otimes_{N'}: E \otimes E \to N'$ is defined by
\begin{equation}\label{N'prod_Mth}
V \otimes_{N'} V' \,=\, (v \,\lrcorner\, \omega' + v'\,\lrcorner\,\omega) + (v \,\lrcorner\, \sigma' + v' \,\lrcorner\, \sigma - \omega \wedge \omega')\,.
\end{equation}
We make this definition as it is the result of taking the $E_{7(7)}\times\RR^+$ covariant projection of $V\otimes V'$ onto $N$ (from~\cite{Coimbra:2011ky}) and then projecting onto $N'$ using the natural mappings in~\eqref{eq:N-sequences}. We stress that the sections of $N'$ themselves do not transform in a definite representation of $\E_{7(7)}\times \mathbb{R}^+$.

			We refer to~\cite{waldram4} for a detailed discussion.
				%
					\begin{table}[h!]
						\centering
						\begin{tabular}{l l l l l }
%							\toprule
							$d$	&	$E^*$					&	$\adj F$													& 	$N \subset S^2 E$			&$K \subset E^* \otimes \adj F$	\\
							\midrule
							7	& 	$\mathbf{56}_{-1}$			&	$\mathbf{133}_0 + \mathbf{1}_0$								&	$\mathbf{133}_2$			&$\mathbf{912}_{-1}$		\\[1.2mm]
							6 	&	$\mathbf{27}_{-1}$			&	$\mathbf{78}_0 + \mathbf{1}_0$								&	$\mathbf{27'}_2$			&$\mathbf{351'}_{-1}$		\\[1.2mm]
							5  	&	$\mathbf{16}^c_{-1}$			&	$\mathbf{45}_0 + \mathbf{1}_0$								&	$\mathbf{10}_2$			&$\mathbf{144}^c_{-1}$		\\[1.2mm]
							4	&	$\mathbf{10}_{-1}$			&	$\mathbf{24}_0 + \mathbf{1}_0$								&	$\mathbf{5'}_2$				&$\mathbf{40}_{-1} + \mathbf{15}'_{-1}$		\\[1.2mm]
							3	&	$(\mathbf{3},\mathbf{2})_{-1}$	&	$(\mathbf{8},\mathbf{1})_0 + (\mathbf{1}, \mathbf{3})_0 + \mathbf{1}_0$	&	$(\mathbf{3'},\mathbf{1})_{2}$	&$(\mathbf{3'},\mathbf{2})_{-1} + (\mathbf{6},\mathbf{2})_{-1}$		\\[1.2mm]
							\bottomrule
						\end{tabular}
						\caption{Some generalised tensor bundles.}
						\label{tab:EddRep}
					\end{table}
				%
			Note that these bundle representations also appear in the tensor hierarchy formulation of gauged supergravity~\cite{Ciceri:2014wya, TensorHier1} and in $E_{11}$ dimensional reduction~\cite{E11PetW, E11Berg}.
				%
			\subsection{Generalised Lie derivative}
				%
				The Dorfman derivative is constructed as a generalisation of the Lie derivative,
					%
						\begin{equation}
							(L_V V')^M =  V^N \partial_N  V^{\prime M} - (\partial \times_{\ad} V)^M_{\phantom{M}N} V^{\prime N} \, , 
						\end{equation}
					%
				Its expression for two generalised vectors is given in~\eqref{dorfM}.
				
				The action of the Dorfman derivative acting on an element of the adjoint can also be constructed~\cite{waldram4} and reads,
					%
						\begin{equation}\label{eq:M_Dorf_adjoint}
							\begin{split}
								L_{V} R & =(\mathcal{L}_{v}r+ j\alpha \lrcorner j\dd\omega-\tfrac{1}{3}\id\alpha\lrcorner\dd\omega-j\tilde{\alpha}\lrcorner j\dd\sigma+\tfrac{2}{3}\id\tilde{\alpha}\lrcorner\dd\sigma) +(\mathcal{L}_{v}\tilde{\alpha}) \\
 & \phantom{=} + (\mathcal{L}_{v}a+r\cdot\dd\omega-\alpha\lrcorner\dd\sigma) +(\mathcal{L}_{v}\tilde{a}+r\cdot\dd\sigma+\dd\omega\wedge a)+(\mathcal{L}_{v}\alpha-\tilde{\alpha}\lrcorner\dd\omega)\, .
							\end{split}
						\end{equation}
					%
				
				The expression for the twisted Lie derivative is
					%
						\begin{equation}
							\mathbb{L}_{\tilde{V}} \tilde{\mathcal{A}} = \mathcal{L}_{\tilde{v}} \tilde{\mathcal{A}} + R_{ \mathbb{L}_{\tilde{V}} } \cdot \tilde{\mathcal{A}} \, ,
						\end{equation}
					%
				where $R_{ \mathbb{L}_{\tilde{V}} }$ is a element in the adjoint representation of $\mathfrak{e}_{7(7)}\times \RR^+$ explicitly given by
					%
						\begin{equation}
							R_{ \mathbb{L}_{\tilde{V}} } = \dd \tilde{\omega} - \iota_{v} F + \dd \tilde{\sigma} - \iota_{\tilde{v}} \tilde{F} +\tilde{\omega} \wedge F \, .
						\end{equation}
					%
				This coincides with the replacements rules given in~\eqref{repruleTwDorf}.
				%
			%
		\section{Generalised geometry for type IIA from M-theory reduction}\label{appsec:EGGIIA}
			%
			We can now proceed and reduce the structures above to type IIA supergravity (in string frame) on a six-dimensional manifold $M_6$.
			Consider M-theory exceptional generalised geometry on a seven dimensional manifold $M_7$, then decomposing the $E_{7(7)}\times \RR^+$ generalised tangent bundle $E$ under the $\GL(6, \RR)$ structure group of $M_6$, we get 
				%
					\begin{equation}\label{app:gentb}
						E \simeq T \oplus T^* \oplus \Lambda^5 T^*\oplus (T^* \otimes \Lambda^6 T^*) \oplus \Lambda^\mathrm{even}T^* \, ,
					\end{equation}
				%
			where $\Lambda^{\mathrm{even}}T^*=\mathbb{R} \oplus \Lambda^2 T^* \oplus \Lambda^4 T^* \oplus \Lambda^6 T^*$ and each term in the direct sum is now on $M_6$. 
			A section, transforming again in the fundamental of $E_{7(7)}\times \RR^+$, can be written as
				%
					\begin{equation}\label{app:genvec}	
						V = v + \lambda + \sigma + \tau + \omega\ ,
					\end{equation}
				%
			where $\omega=\omega_0 + \omega_2 + \omega_4 + \omega_6$ is a poly-form in $ \Lambda^{\mathrm{even}} T^*$.

			The $\GL(6)$ decomposition of the adjoint bundle is
				%
					\begin{equation}
					\adj F = \RR_\Delta \oplus \RR_\phi \oplus (T\otimes T^*) \oplus \Lambda^2 T \oplus \Lambda^2 T^*	\oplus \Lambda^6 T \oplus \Lambda^6 T^* \oplus \Lambda^{\mathrm{odd}} T \oplus 												\Lambda^{\mathrm{odd}} T^*\ ,
					\end{equation}
				%
			with generic section
				%
					\begin{equation}\label{Edd_adjoint}
						R = l + \varphi + r + \beta + b + \tilde \beta + \tilde b + \alpha + a \, ,
					\end{equation}
				%
			where $\alpha = \alpha_1 + \alpha_3 + \alpha_5 \in \Lambda^{\mathrm{odd}} T$ and $a= a_1 + a_3 + a_5 \in \Lambda^{\mathrm{odd}} T^*$ are antisymmetric poly-vectors and poly-forms, respectively. 

			Let us now derive the action of the adjoint of $E_{7(7)}\times \RR^+$ on a generalised vector and the commutators of two adjoints in type IIA language. 
			Denoting by $z$ the coordinate along the seventh direction, a type IIA generalised vector is related to an M-theory one as
				%
					\begin{equation}\label{redrules}
						\begin{split}
							v_{\mathrm{M}} &= v + \omega_0 \partial_z\, , \\
							\omega_{\mathrm{M}} &= \omega_2 - \lambda \wedge \dd z \, , \\
							\sigma_{\mathrm{M}} &= \sigma + \omega_4 \wedge \dd z\, , \\
							\tau_{\mathrm{M}} &= \tau \wedge \dd z + \dd z \otimes (\omega_6 \wedge \dd z)\, ,
						\end{split}
					\end{equation}
				%
			where, as in the main text, $\tau = \tau_1 \otimes \tau_6$ and the subscript M denotes the M-theory quantities defined in section~\ref{sec:MthExcGeom}. 
			Similarly, the M-theory adjoint~\eqref{eq:Ggeom-M} decomposes as
				%
					\begin{equation}\label{redrulesad}
						\begin{array}{l}
							l_{\mathrm{M}} \ =\ l - \tfrac13 \varphi\\
							a_{\mathrm{M}} \ =\ a_3 + b \wedge \dd z \\
							\tilde a_{\mathrm{M}} \ = \ \tilde b + a_5 \wedge \dd z \\
							\alpha_{\mathrm{M}} \ =\ \alpha_3 + \beta \wedge \partial_z \\ 
							\tilde \alpha_{\mathrm{M}} \ =\ \tilde \beta + \alpha_5 \wedge \partial_z 
						\end{array}
							\qquad \qquad 
						r_{\mathrm{M}} = 	\begin{pmatrix} 
										r + \tfrac13 \varphi \,\id \,&\, - \alpha_1 \\ 
										a_1 \,&\, -\tfrac23 \varphi
									\end{pmatrix} \, ,
					\end{equation}
				%
			where the identification $l_{\mathrm{M}} = l - \tfrac13 \varphi$ follows from the relation between the M-theory and IIA warp factors $\Delta_{\mathrm{M}} = \Delta_{\mathrm{IIA}} - \tfrac13 \phi$.

			Decomposing the M-theory adjoint action given in~\eqref{Mth_adjoint_act} yields the IIA adjoint action on a generalised vector. Denoting this by $V' = R\cdot V$, we have
				%
					\begin{align}\label{IIAadjvecCompact}
						v' &= l v + r \cdot v - [ \alpha \lrcorner s(\omega)]_{-1} - \beta \lrcorner \lambda - \tilde \beta \lrcorner \sigma \, , \\
						\lambda' &= l \lambda + r \cdot \lambda - v \lrcorner b - [\alpha\lrcorner s(\omega)]_1 - \tilde \beta \lrcorner \tau \, , \\
						\sigma' &= (l-2\varphi) \sigma + r \cdot \sigma + v \lrcorner \tilde b - [\omega\wedge s(a)]_5 - \beta \lrcorner \tau \, , \\
						\tau' &= (l-2\varphi) \tau + r \cdot \tau + j a \wedge s(\omega) + j \tilde b \wedge \lambda - j b \wedge \sigma \, , \\
						\omega' &= (l-\varphi) \omega + r \cdot \omega + b\wedge \omega + v \lrcorner a + \lambda \wedge a + \beta \lrcorner \omega + \alpha \lrcorner \sigma +\alpha\lrcorner \tau \, ,
					\end{align}
				%
			where $s$ is the sign operator $s(\omega_n) = (-1)^{[n/2]} \omega_n$ for $\omega_n \in \Lambda^n T^*$, and $[\ldots]_p$ denotes the form of degree $p$ in the formal sum inside the parenthesis (by $-1$ we mean we pick the vector component). 
			The $E_{7(7)}$ subalgebra is specified by $\frac 12{\rm tr}(r) = l - \varphi$. 
			In particular, the $\rmO(6,6) \subset \E_{7}$ action is generated by $r, b$ and~$\beta$, also setting $\varphi = -\tfrac{1}{2}\mathrm{tr}(r)$ and all other generators to zero. 

			Reducing the M-theory commutator~\eqref{comm_Mth_adj} with the decomposition~\eqref{redrulesad} we find that the IIA adjoint commutator $R'' = [R,R']$ reads 
				%
				\begin{subequations}\label{eq:commAdjIIA}
				%
					\begin{align}
							%
								\begin{split}
									l'' &= -\tfrac{1}{2} (\alpha_1 \lrcorner a_1' - \alpha_1' \lrcorner a_1)+\tfrac{1}{2} (\alpha_3 \lrcorner a_3' - \alpha_3' \lrcorner a_3) \\
									&\phantom{=} - \tfrac{1}{2} (\alpha_5 \lrcorner a_5' - \alpha_5' \lrcorner a_5) + (\tilde \beta' \lrcorner \tilde b - \tilde \beta \lrcorner \tilde b') \, ,
								\end{split}
							%
								\\
							%
								\begin{split}
									\phi'' &= \tfrac{3}{2} (\alpha_1' \lrcorner a_1 - \alpha_1 \lrcorner a_1') + \tfrac{1}{2} (\alpha_3 \lrcorner a_3' - \alpha_3' \lrcorner a_3) - \tfrac{1}{2}(\alpha_5' \lrcorner a_5 - \alpha_5 \lrcorner a_5') \\
										&\phantom{=} - ( \beta \lrcorner b' - \beta' \lrcorner b) + (\tilde \beta' \lrcorner \tilde b - \tilde \beta \lrcorner \tilde b') \, ,
								\end{split}
							%
							\\
							%
								\begin{split}
									r'' 	&= [ r, r'] + j \alpha_1' \lrcorner j a_1 - j \alpha_1\lrcorner j a_1' + j\alpha_3 \lrcorner j a_3' - j\alpha_3' \lrcorner ja_3 - j\alpha_5 \lrcorner j a_5' +j \alpha_5' \lrcorner j a_5 \\ 
										&\phantom{=} + j \beta \lrcorner j b' - j \beta' \lrcorner j b - j \tilde \beta \lrcorner j \tilde b' + j \tilde \beta' \lrcorner j \tilde b + \tfrac{1}{2}\id (\alpha_1' \lrcorner a_1 - \alpha_1 \lrcorner a_1') \\
										&\phantom{=} + \tfrac{1}{2} \id (\alpha_3' \lrcorner a_3 - \alpha_3 \lrcorner a_3')+ \tfrac{1}{2} \id (\alpha_5' \lrcorner a_5 - \alpha_5 \lrcorner a_5') + \id(\tilde \beta \lrcorner \tilde b' - \tilde\beta' \lrcorner \tilde b) \, ,
								\end{split}
							%
							\\
							b'' &= r \cdot b'- r' \cdot b +\alpha_1 \lrcorner a_3' - \alpha_1' \lrcorner a_3 - \alpha_3 \lrcorner a_5' + \alpha_3' \lrcorner a_5 \, , \\
							\tilde b'' &= r \cdot \tilde b' - r' \cdot \tilde b -2\varphi \tilde b' + 2\varphi'\tilde b + a_1 \wedge a_5' - a_1' \wedge a_5 - a_3 \wedge a_3' \, ,\\
							a'' &= r \cdot a' - r' \cdot a - \varphi a' + \varphi' a + b\wedge a' - b'\wedge a + \beta \lrcorner a' - \beta' \lrcorner a -\alpha \lrcorner \tilde b' + \alpha' \lrcorner \tilde b \, ,\\
							\beta'' &= r \cdot \beta' - r' \cdot \beta + \alpha_3' \lrcorner a_1 - \alpha_3 \lrcorner a_1' - \alpha_5' \lrcorner a_3 + \alpha_5 \lrcorner a_3' \, , \\
							\tilde \beta'' &= r \cdot \tilde \beta' - r' \cdot \tilde \beta + 2\varphi \tilde \beta' - 2\varphi' \tilde \beta + \alpha_1 \wedge \alpha_5' - \alpha_3 \wedge \alpha_3' + \alpha_5\wedge \alpha_1' \, ,\\
							%
								\begin{split}
									\alpha'' &= r \cdot \alpha' - r' \cdot \alpha +\varphi \alpha' - \varphi' \alpha + \beta\wedge\alpha' -\beta'\wedge\alpha - \alpha \lrcorner b' + \alpha' \lrcorner b \\
										&\phantom{=} - \tilde \beta \lrcorner a' + \tilde \beta' \lrcorner a \, .
								\end{split}
							%
					\end{align}
				%
				\end{subequations}
				%
			Next, we obtain the explicit expression for the Dorfman derivative between two type IIA generalised vectors $V$ and $V'$. 
			By plugging~\eqref{redrules} into~\eqref{dorfM} we find:
				%
					\begin{equation*}
						%
						\begin{split}
							L_V V' =& \mathcal{L}_v v' + \left(\mathcal{L}_v \lambda' - \iota_{v^\prime} \dd \lambda\right) + \left(\iota_v \dd \omega_0' - \iota_{v^\prime} \dd \omega_0\right) \\
								& + \left(\mathcal{L}_v \omega_2^\prime - \iota_{v^\prime}\dd \omega_2 - \lambda' \wedge \dd \omega_0 + \omega_0' \dd \lambda\right) \\
								& + \left( \mathcal{L}_v \omega_4^\prime - \iota_{v^\prime}\dd \omega_4 - \lambda' \wedge \dd \omega_2 + \omega_2^\prime \wedge \dd \lambda\right) \\
								& + \left(\mathcal{L}_v \omega_6^\prime - \lambda' \wedge \dd \omega_4 + \omega_4' \wedge \dd \lambda\right) \\
								& + \left( \mathcal{L}_v \sigma' - \iota_{v^\prime}\dd \sigma + \omega_0^\prime \dd \omega_4 - \omega_2^\prime \wedge \dd \omega_2 + \omega_4' \wedge \dd \omega_0\right) \\
								& + \left(\mathcal{L}_v \tau' + j \sigma' \wedge \dd \lambda + \lambda^\prime \otimes \dd \sigma + \dd \omega_0 \otimes \omega_6^\prime + j \omega_4^\prime \wedge \dd \omega_2 - j \omega_2^\prime \wedge \dd \omega_4 \right) \, . 
						\end{split}
						%
					\end{equation*}
				%
			This expression can be cast in the more compact form given in~\eqref{dorfIIA}.
			
			Also the Dorfman derivative acting on a section of the adjoint bundle can be get by the M-theory one using the reduction rules~\eqref{redrules}.
			This reads,
				%
					\begin{equation}
						\begin{split}
							L_V R  =& \left(-\frac{2}{3}\mathcal{L}_{v}\varphi+\frac{1}{2} \alpha_1 \lrcorner \mathrm{d} \omega_0	-\frac{2}{3}\beta \lrcorner \mathrm{d} \lambda+\frac{2}{3}\tilde \beta \lrcorner \mathrm{d} \sigma-\frac{1}{3} \alpha_3 \lrcorner \mathrm{d} \omega_2-\frac{1}{3} \alpha_5 \lrcorner \mathrm{d} \omega_4 \right) \\[1mm]
								&+ \left(\mathcal{L}_{v}\varphi-\frac{3}{2} \alpha_1 \lrcorner \mathrm{d} \omega_0+\beta \lrcorner \mathrm{d} \lambda-\tilde \beta \lrcorner \mathrm{d} \sigma+\frac{1}{2} \alpha_3 \lrcorner \mathrm{d} \omega_2+\frac{1}{2} \alpha_5 \lrcorner \mathrm{d} \omega_4	\right ) \\[1mm] 
								&+ \left( \mathcal{L}_{v} r  -[j\alpha \lrcorner j s(\mathrm{d} \omega)]_{-1\otimes 1} +j \beta \lrcorner j \mathrm{d} \lambda -j \tilde \beta \lrcorner j \mathrm{d} \sigma \right. \\
								& \phantom{++} \left. + \mathds{1} \left( -\frac{1}{3} \mathcal{L}_{v} \varphi	 +\frac{1}{2}[\alpha \lrcorner s(\mathrm{d}\omega)]_0  + \tilde \beta \lrcorner \mathrm{d} \sigma \right)\right ) \\[1mm]
								& + \left(\mathcal{L}_{v}\beta - \alpha_3 \lrcorner \mathrm{d} \omega_0 + \alpha_5 \lrcorner \mathrm{d} \omega_2\right ) \\[1mm]
				& + \left(\mathcal{L}_{v}b-r\cdot  \mathrm{d} \lambda	- \frac{1}{3}\varphi \mathrm{d} \lambda+\alpha_1 \lrcorner \mathrm{d} \omega_2+\alpha_3 \lrcorner \mathrm{d} \omega_4	\right ) \\[1mm]
				& + \left(\mathcal{L}_{v} \tilde \beta \right )  + \left(\mathcal{L}_{v} \tilde b 	+r\cdot \mathrm{d} \sigma	+\frac{1}{3} \varphi \mathrm{d} \sigma+[ \alpha \wedge s( \mathrm{d} \omega ) ]_6\right ) \\[1mm]
				& + \left( \mathcal{L}_{v}\alpha  -\tilde \beta \lrcorner \mathrm{d} \omega  +(\alpha_3 + \alpha_5 ) \lrcorner \mathrm{d} \lambda	\right ) \\[1mm]
				& + \left(\mathcal{L}_{v} \alpha	+r\cdot \mathrm{d} \omega -\alpha \lrcorner \mathrm{d} \sigma +(\alpha_1 +\alpha_3) \wedge \mathrm{d} \lambda+b \wedge ( \mathrm{d} \omega_0 + \mathrm{d} \omega_2) \right. \\
				& \phantom{++} \left. + \beta \lrcorner ( \mathrm{d} \omega_2 + \mathrm{d} \omega_4) -\varphi \mathrm{d} \omega_0 -\frac{1}{3} \varphi \mathrm{d} \omega_2+\frac{1}{3} \varphi \mathrm{d} \omega_4 \right) \, .
						\end{split}
					\end{equation}
				%

			As in M-theory, the presence in type IIA of non-trivial gauge potentials leads to the definition of a twisted generalised tangent bundle whose sections are related to~\eqref{app:genvec} by the twist~\eqref{eq:twistC_short}. 
			In order to derive the explicit form of the twist we need to exponentiate the $\E_{7(7)}$ adjoint action on a generalised vector~\eqref{IIAadjvecCompact} with $l=\varphi = r = \beta =\tilde\beta= \alpha= 0$. 
			This corresponds to exponentiating a nilpotent sub-algebra of the $\mathfrak{e}_{7(7)}$ algebra, comprising precisely the form potentials of type IIA supergravity. 
			We find that the series expansion
				%
					\begin{equation}
						V' = e^{R}\cdot V \ \equiv\ V + R \cdot V + \tfrac{1}{2} R \cdot(R \cdot V) + \ldots 
					\end{equation}
				%
			truncates at fifth order, and is given by
				%
					\begin{align*}
							v' 		&= v \, , \\
							\lambda' 	&= \lambda - \iota_v b \, , \\
							\sigma' 	&= \sigma + \iota_v \tilde b -\left[\mathcal{B}^{(1)} \wedge s(a)\wedge \omega + \mathcal{B}^{(2)}\wedge s(a) \wedge \iota_v a \right]_5 + a_1 \wedge a_3 \wedge \left(\lambda -\tfrac 13 \iota_v b \right) \, ,\\ 
							%
							\begin{split}
							\tau' 		&= \tau + j \tilde b \wedge \left(\lambda - \tfrac 12 \iota_v b \right) - j s(a) \wedge \left(\mathcal{B}^{(1)}\wedge\omega + \mathcal{B}^{(2)}\wedge (\iota_v a + \lambda \wedge a)+ \mathcal{B}^{(3)}\wedge a\wedge \iota_v b \right) \\
									& \phantom{=} - j b \wedge \left(\sigma + \tfrac 12 \iota_v \tilde b - \mathcal{B}^{(2)}\wedge s(a) \wedge \omega - \mathcal{B}^{(3)}\wedge s(a)\wedge \iota_v a + \tfrac 13 a_1 \wedge a_3 \wedge \left(\lambda-\tfrac 14 \iota_v b\right)\right)\, , 
							\end{split}
							%
							\\
							\omega' 	&= e^b \wedge\omega + \mathcal{B}^{(1)}\wedge (\iota_v a + \lambda \wedge a) + \mathcal{B}^{(2)}\wedge a\wedge \iota_v b \ ,
					\end{align*}
				%
			where we introduced the shorthand notation,
				%
					\begin{align}
						\mathcal{B}^{(1)} &= \frac{e^b-1}{b} = 1 + \tfrac{1}{2} b + \tfrac{1}{3!}b \wedge b + \ldots \, , \\
						\mathcal{B}^{(2)} &= \frac{e^b-1-b}{b\wedge b} = \tfrac{1}{2} + \tfrac{1}{3!} b + \tfrac{1}{4!}b\wedge b + \ldots \, , \\
						\mathcal{B}^{(3)} &= \frac{e^b-1-b- \tfrac 12 b \wedge b}{b\wedge b\wedge b} = \tfrac{1}{3!} + \tfrac{1}{4!}b + \tfrac{1}{5!}b \wedge b + \ldots\, .
					\end{align}
				%

			We can also reduce to type IIA the bundle $N \subset S^2 E$ given in~\eqref{MthNbundle}. 
			In terms of bundles on $M_6$, we obtain 
				%
					\begin{equation}\label{Nbundle}
						N \simeq \mathbb{R} \oplus \Lambda^4T^* \oplus \Lambda^{\mathrm{odd}}T^* \oplus \Lambda^6T^* \oplus (T^* \otimes \Lambda^5 T^*) \oplus (\Lambda^2T^* \oplus \Lambda^6T^* \oplus \Lambda^\mathrm{odd}T^*)\otimes \Lambda^6T^* \, .
					\end{equation}
				%
			The full $N$ bundle in type IIA is described as a similar set of exact sequences to those in M-theory~\eqref{eq:N-sequences}. 
			Again, these provide us with a natural projection onto a smaller bundle $N'$, which is isomorphic to
				%
					\begin{equation}\label{NprimeBundle}
						N' \simeq \mathbb{R} \oplus \Lambda^4T^* \oplus \Lambda^{\rm odd} T^* \, ,
					\end{equation}
				%
			(note that this also includes $\Lambda^5 T^*$ and thus it is not just the reduction of the M-theory $N'$ bundle given in~\eqref{MthN'}).
			Given a basis $\{\hat E_A\}$, $A = 1,\ldots, 56$, for the generalised tangent bundle $E$, a section $S$ of $N'$ has the form
				%
					\begin{equation}
						S = S^{AB}\hat E_A \otimes_{N'} \hat E_B \, ,
					\end{equation}
				%
			where $S^{AB}$ are functions on the manifold and the map 
				%
					\begin{equation}
						\otimes_{N'}: E \otimes E \longrightarrow N'
					\end{equation} 
				%
			is defined as
				%
					\begin{equation}\label{N'prod_IIA}
						\begin{split}
							V \otimes_{N'} V' =& v\lrcorner \lambda' + v' \lrcorner \lambda \\
										& + v \lrcorner \sigma' + v' \lrcorner \sigma + [\omega \wedge s(\omega')]_4 \\
										& + v \lrcorner \omega' + \lambda\wedge \omega' + v' \lrcorner \omega + \lambda' \wedge \omega \, .
						\end{split}
					\end{equation}
				%
			As for~\eqref{N'prod_Mth}, this is the $\E_{7(7)}\times\RR^+$ covariant projection to $N$ further projected onto $N'$.
			%
			%%
			%
			\subsection{The split frame}\label{splitfr_MtoIIA}
			%
				As discussed in section~\ref{gen_frame_metric}, a convenient way to compute the generalised metric is starting from the conformal split frame, namely a specific choice of frame on the generalised tangent bundle~\eqref{IIAtangbung}. 
				Here we derive the type IIA conformal split frame by reducing the M-theory one given in~\cite{Coimbra:2011ky}. 
				The latter reads
					%
						\begin{equation}\label{eq:geom-basis}
							\begin{split}
							\mathcal{E}_{{\mathrm{M}} \, \hat{a}} &= e^{\Delta_{\mathrm{M}}} \Big( \hat{e}_{\hat{a}} + i_{\hat{e}_{\hat{a}}} A + i_{\hat{e}_{\hat{a}}} \tilde A + \tfrac{1}{2} A \wedge i_{\hat{e}_{\hat{a}}} A \\
														&\phantom{= e^{\Delta_{\mathrm{M}}} \Big( \hat{e}_{\hat{a}}} + j A \wedge i_{\hat{e}_{\hat{a}}}\tilde{A} + \tfrac{1}{6} j A \wedge A \wedge \iota_{\hat{e}_{\hat{a}}} A \Big) \, , \\
 							\mathcal{E}_{\mathrm{M}}^{\hat{a}\hat{b}} &= e^{\Delta_{\mathrm{M}}} \left( e^{\hat{a}\hat{b}} + A \wedge e^{\hat{a}\hat{b}} - j \tilde{A}\wedge e^{\hat{a}\hat{b}} + \tfrac{1}{2}j A \wedge A \wedge e^{\hat{a}\hat{b}} \right)\, , \\
							\mathcal{E}_{\mathrm{M}}^{\hat{a}_1\dots \hat{a}_5} &= e^{\Delta_{\mathrm{M}}} \left( e^{\hat{a}_1\dots \hat{a}_5} + j A \wedge e^{\hat{a}_1\dots \hat{a}_5} \right)\, , \\
 							\mathcal{E}_{\mathrm{M}}^{\hat{a},\hat{a}_1\dots \hat{a}_7} &= e^{\Delta_{\mathrm{M}}}\, e^{\hat{a},\hat{a}_1\dots \hat{a}_7}\, ,
							\end{split}
						\end{equation}
					%
				where $\Delta_{\mathrm{M}}$ is the M-theory warp factor and $A$ and $\tilde A$ are the three- and six-form potentials of M-theory. 
				$\hat{e}_{\hat{a}}$ is a frame for $T M_7$, $e_{\hat{a}} $ is the dual one and $e^{\hat{a}_1\ldots \hat{a}_p}= e^{\hat{a}_1}\wedge \cdots \wedge e^{\hat{a}_p}$, and $e^{\hat{a},\hat{a}_1\ldots \hat{a}_7} = e^{\hat{a}}\otimes e^{\hat{a}_1\ldots \hat{a}_7}$. 
				The index $\hat{a}$ goes from 1 to 7 and, not to clutter the notation, we omitted the subscript M on $\hat{e}_{\hat{a}} $ and $e_{\hat{a}}$.

				In reducing to type IIA, we decompose the M-theory potentials as 
					%
						\begin{align}
							A &= C_3 - B \wedge \dd z\, , \\
							\tilde{A} &= \tilde{B} - \tfrac{1}{2}C_5 \wedge C_1 + (C_5 - \tfrac{1}{2}B \wedge C_3) \wedge \dd z\, , 
						\end{align}
					%
				where $z$ denotes again the circle direction along which we are reducing, and $B$, $\tilde{B}$ and $C_k$ are the IIA potentials. 
				As already pointed out, the IIA and M-theory warp factors are related by
					%
						\begin{equation}
							\Delta_{\mathrm{M}} \,=\, \Delta - \phi/3 \ . 
						\end{equation}
					%
				To reduce the split frame~\eqref{eq:geom-basis}, we also need to decompose the seven-dimensional indices as $\hat{a} = (a, z)$ with $a=1, \ldots, 6$ and write the seven-dimensional frames as 
					%
					\begin{align}
								\hat{e}_{{\mathrm{M}}\ \hat{a}} = 	\begin{cases} 
																e^{\phi/3}\left(\hat{e}_a + C_a \partial_z\right) \, , \\
 																e^{-2\phi/3}\partial_z \, ,
															\end{cases}
							& &
								e_{\mathrm{M}}^{\hat{a}} =		\begin{cases} 
																e^{-\phi/3}e^a \, ,\\
																e^{2\phi/3}\left(\dd z - C_1\right)\ ,
															\end{cases}
						\end{align}
					%
				where $\hat{e}_a$ and $e^a$ are basis for the IIA frame bundles and $C_a$ denotes the components of the one-form $C_1$. 
				The reduction gives
					%
						\begin{equation*}
							\{ \hat{E}_A \} = \{ \hat{\mathcal{E}}_a \} \cup \{\mathcal{E}^a\} \cup \{ \mathcal{E}^{a_1 \ldots a_5} \} \cup \{ \mathcal{E}^{a,a_1\ldots a_6} \} \cup \{ \mathcal{E} \} \cup \{ \mathcal{E}^{a_1a_2} \} \cup \{ \mathcal{E}^{a_1\ldots a_4} \} \cup \{ \mathcal{E}^{a_1 \ldots a_6} \} \, ,
						\end{equation*} 
					%
				with
					%
						\begin{equation}\label{splitframe_explicit}
							\begin{split}
								\hat{\mathcal{E}}_a &= e^{\Delta} \left(\hat e_a + \iota_{\hat e_a}B + e^{-B}\wedge\iota_{\hat e_a}(C_1 + C_3 + C_5) + \iota_{\hat e_a}\tilde{B} + j\tilde{B} \wedge \iota_{\hat e_a}B \right. \\
												& \phantom{= e^{\Delta}} - \tfrac{1}{2} C_1 \wedge \iota_{\hat e_a}C_5 +\tfrac{1}{2} C_3 \wedge \iota_{\hat e_a} C_3 -\tfrac{1}{2} C_5 \wedge \iota_{\hat e_a} C_1 - \tfrac{1}{2} j C_5 \wedge \iota_{\hat e_a}C_3 \\
												& \phantom{= e^{\Delta}} \left. + \tfrac{1}{2}j B\wedge C_3 \wedge \iota_{\hat e_a}C_3 -\tfrac{1}{2}j B\wedge C_5 \wedge \iota_{\hat e_a}C_1 -\tfrac{1}{2}j B\wedge C_1 \wedge \iota_{\hat e_a}C_5 \right) \, , \\[2mm]
								\mathcal{E}^a &= e^{\Delta} \big(e^a - e^{-B}\wedge (C_1+ C_3 + C_5) \wedge e^a + j\tilde{B} \wedge e^a - C_3 \wedge C_1 \wedge e^a \\
											& \phantom{=e^{\Delta}} - j B \wedge C_3 \wedge C_1\wedge e^a + \tfrac{1}{2}j C_1 \wedge C_5\wedge e^a - \tfrac{1}{2} j C_3 \wedge C_3\wedge e^a \\
											& \phantom{=e^{\Delta}} + \tfrac{1}{2}j C_5 \wedge C_1\wedge e^a \big)\, , \\[2mm]
								\mathcal{E}^{a_1 \ldots a_5} &= e^{\Delta-2\phi}\left(e^{a_1 \ldots a_5} + jB \wedge e^{a_1 \ldots a_5} \right)\, , \\[2mm]
								\mathcal{E}^{a,a_1\ldots a_6} &= e^{\Delta-2\phi}\left(e^{a,a_1 \ldots a_6} \right)\, , \\[2mm]
								\mathcal{E} &= e^{\Delta-\phi} \left( e^{-B} - C_5 - j B\wedge C_5 \right)\ , \\[2mm]
								\mathcal{E}^{a_1 a_2} &= e^{\Delta-\phi} \left( e^{-B}\wedge e^{a_1 a_2} + C_3\wedge e^{a_1 a_2} - jC_5 \wedge e^{a_1 a_2} + jB \wedge C_3\wedge e^{a_1 a_2} \right)\, , \\[2mm]
								\mathcal{E}^{a_1 \ldots a_4} &= e^{\Delta-\phi}\left(e^{-B}\wedge e^{a_1 \ldots a_4} - C_1 \wedge e^{a_1 \ldots a_4} \right. \\
											 & \phantom{=e^{\Delta}}\left. + jC_3 \wedge e^{a_1 \ldots a_4} -jB\wedge C_1 \wedge e^{a_1\ldots a_4} \right) , \\[2mm]
								\mathcal{E}^{a_1 \ldots a_6} &= e^{\Delta-\phi}\left(e^{a_1 \ldots a_6} - j C_1 \wedge e^{a_1\ldots a_6} \right)\, .
							\end{split}
						\end{equation}
					%
				These expressions can be summarised in the twist given in~\eqref{twist_splitfr}.
			%
			%%%
			%%
			%
		\section{Twisted bundle and gauge transformations}\label{appsec:EGGgauge}
			%
				In this section -- closely following~\cite{oscar1} -- we show how one can derive the patching conditions~\eqref{patchingIIA} of the generalised tangent bundle starting from the supergravity gauge transformations.
				We will refer to (massive) type IIA generalised geometry, however, the line of reasoning holds in general. 
				The key requirement will be that the generalised vector generates the diffeomorphism and gauge transformations that act on the supergravity fields. 
				We include the Romans mass in our computation, the massless case simply follows by setting $m=0$.

				We start imposing that in each chart $U$ covering the manifold $M_6$, a generalised vector $V$ generates a diffeomorphism and gauge transformation of the type IIA supergravity potentials:
					%
						\begin{equation}\label{eq:gauge-trans-by-V-E1}
							\begin{split}
								\delta_{V} B &= \mathcal{L}_v B - \dd \lambda \, , \\
								\delta_{V} C &= \mathcal{L}_v C - e^B \wedge (\dd \omega - m \lambda) \, , \\
								\delta_{V} \tilde{B} &= \mathcal{L}_v \tilde{B} - (\dd \sigma + m \,\omega_6) - \tfrac{1}{2} \big[e^B \wedge (\dd \omega - m \lambda) \wedge s(C) \big]_6 \,,\\
							\end{split}
						\end{equation}
					%
				where all the fields are defined on $U$.
				In these expressions, the diffeomorphism along a generic vector $v$ is generated by the ordinary Lie derivative $\mathcal{L}_v$, while the remaining terms correspond to the supergravity gauge transformation.

				We next require that the generalised diffeomorphism~\eqref{eq:gauge-trans-by-V-E1} be globally well-defined. 
				This means that on the intersection of a patch $U_\alpha$ with another patch $U_\beta$, the new field configuration defined by~\eqref{eq:gauge-trans-by-V-E1} is patched in the same way as the original one, so as to preserve the global structure (which cannot be changed by an infinitesimal transformation). 
				The patching conditions for the gauge potentials on $U_{\alpha} \cap U_{\beta}$ are given by the gauge transformation of the supergravity fields. 
				At the linearised level, these read
					%
						\begin{equation}\label{eq:lin-gauge-E1}
							\begin{split}
								B_{(\alpha)} &= B_{(\beta)} + \dd \Lambda_{(\alpha\beta)} \, ,\\
								C_{(\alpha)} &= C_{(\beta)} + e^{B_{(\beta)} } \wedge (\dd \Omega_{(\alpha\beta)} -m \Lambda_{(\alpha\beta)})\, , \\
								\tilde{B}_{(\alpha)} &= \tilde{B}_{(\beta)} + \dd \tilde \Lambda_{(\alpha\beta)} + m \Omega_{6(\alpha\beta)} + \tfrac{1}{2} \left[e^{B_{(\beta)}}\wedge (\dd \Omega_{(\alpha\beta)} -m\Lambda_{(\alpha\beta)})\wedge s(C_{(\beta)}) \right]_6 \, ,
							\end{split}
						\end{equation}
					%
				where the labels $(\alpha)$ and $(\beta)$ indicate fields on $U_\alpha$ and $U_\beta$, respectively, while $(\alpha\beta)$ denotes a field defined just on $U_{\alpha} \cap U_{\beta}$. 
				Note that these gauge transformations have the opposite signs with respect to those in~\eqref{eq:gauge-trans-by-V-E1}, as that equation describes an active transformation which shifts the field configuration to a physically equivalent one; contrastingly, equation~\eqref{eq:lin-gauge-E1} describes a patching relation needed to define the fields on the whole manifold, similar to coordinate invariance in general relativity, which is a passive transformation.

				From \eqref{eq:lin-gauge-E1} we construct the corresponding finite transformation.
				Its form is not uniquely determined, since it depends on the order one chooses for the exponentiation of the infinitesimal transformations. 
				We choose to exponentiate first the action of the RR transformation with parameter $\Omega$, then the NSNS transformation by $ \Lambda$ and finally the one by $\tilde \Lambda$. 
				This gives:
					%
						\begin{equation}\label{eq:finite-gauge-E1}
							\begin{split}
								B_{(\alpha)} &= B_{(\beta)} + \dd \Lambda_{(\alpha\beta)} \, ,\\
								C_{(\alpha)} &= C_{(\beta)} + e^{B_{(\beta)} + \dd \Lambda_{(\alpha\beta)}} \wedge \dd \Omega_{(\alpha\beta)} - m e^{B_{(\beta)} } \wedge\Lambda_{(\alpha\beta)} \wedge \left( \tfrac{e^{\dd \Lambda}-1}{\dd \Lambda} \right)_{(\alpha\beta)} \, , \\
								\tilde{B}_{(\alpha)} &= \tilde{B}_{(\beta)} + \dd \tilde \Lambda_{(\alpha\beta)} + m \Omega_{6 \ (\alpha\beta)} + \tfrac{1}{2} m \Lambda_{(\alpha\beta)} \wedge \Big[ e^{-B_{(\beta)} } \wedge \left(\tfrac{ e^{-\dd \Lambda}-1}{\dd \Lambda} \right)_{(\alpha\beta)} \wedge C_{(\beta)} \Big]_5 \\
												&\phantom{=} + \tfrac{1}{2} \Big[ \dd \Omega_{(\alpha\beta)} \wedge e^{B_{(\beta)} + \dd \Lambda_{(\alpha\beta)}} \wedge s( C_{(\beta)} ) - m\, \dd \Omega_{(\alpha\beta)} \wedge \Lambda_{(\alpha\beta)} \wedge \left( \tfrac{e^{\dd \Lambda}-1}{\dd \Lambda} \right)_{(\alpha\beta)} \Big]_6 \,,
							\end{split}
						\end{equation}
					%
				where we used the shorthand notation
					%
						\begin{equation}
							\tfrac{e^{\pm\dd \Lambda} - 1}{\dd \Lambda} = \pm 1 + \tfrac{1}{2} \dd \Lambda \pm \tfrac{1}{3!} \dd \Lambda \wedge \dd \Lambda\, + \dots \, .
						\end{equation}
					%
				
				Imposing that the new field configurations \eqref{eq:gauge-trans-by-V-E1} in two overlapping patches $U_\alpha$ and $U_\beta$ are still related in the intersection $U_\alpha \cap U_\beta$ by the transformation \eqref{eq:finite-gauge-E1}, and working to first order in the components of $V$, we obtain
					%
					\begin{equation}\label{eq:well-defined}
						\begin{split}
							\delta_{V_{(\alpha)}} B_{(\alpha)} &= \delta_{V_{(\beta)}} B_{(\beta)}\,, \\
 							\delta_{V_{(\alpha)}} C_{(\alpha)} &= \delta_{V_{(\beta)}} C_{(\beta)} + e^{ B_{(\beta)} +\dd \Lambda} \wedge \delta_{V_{(\beta)}} B_{(\beta)} \wedge \dd \Omega \\
													&\phantom{=} - m e^{ B_{(\beta)} } \wedge \delta_{V_{(\beta)}} B_{(\beta)} \wedge \Lambda \wedge \big( \tfrac{e^{\dd \Lambda}-1}{\dd \Lambda} \big) \, , \\
							\delta_{V_{(\alpha)}} \tilde{B}_{(\alpha)} & = \delta_{V_{(\beta)}} \tilde{B}_{(\beta)} \\
														&\phantom{=}+ \tfrac{1}{2} m \Lambda \wedge \Big[ e^{-B_{(\beta)} } \wedge \big(\tfrac{ e^{-\dd \Lambda}-1}{\dd \Lambda} \big) \wedge ( \delta_{V_{(\beta)}} C_{(\beta)} - \delta_{V_{(\beta)}} B_{(\beta)} \wedge C_{(\beta)}) \Big]_5 \\
															&\phantom{=} + \tfrac{1}{2} \Big[ \dd \Omega \wedge e^{B_{(\beta)} + \dd \Lambda} \wedge\Big( s( \delta_{V_{(\beta)}} C_{(\beta)}) + \delta_{V_{(\beta)}} B_{(\beta)} \wedge s(C_{(\beta)})\Big) \Big]_6 \, ,
						\end{split}
					\end{equation}
					%
				where for ease of notation we are omitting the label $(\alpha\beta)$ on $\Lambda$, $\tilde\Lambda$ and $\Omega$.
				This equation can be solved to give relations between the components of $V_{(\alpha)}$ and $V_{(\beta)}$. Also requiring that these relations are linear in $V_{(\alpha)}$ and $V_{(\beta)}$, we obtain the following patching rules for the generalised vector: 
					%
						\begin{equation*}
							\begin{split}
								v_{(\alpha)}&= v_{(\beta)} \, , \\
								\lambda_{(\alpha)} &= \lambda_{(\beta)} + \iota_{v_{(\beta)}} \dd \Lambda \ , \\
								\sigma_{(\alpha)} &= \sigma_{(\beta)} +\iota_{v_{(\beta)}}(\dd\tilde\Lambda+m\Omega_6) + \dd\Omega_{0} \wedge \dd\Omega_{2} \wedge (\lambda_{(\beta)} + \iota_{ v_{(\beta)}}\dd\Lambda) \\ 
											& \phantom{=} - \big[ s(\dd\Omega)\wedge \big( e^{-\dd\Lambda} \wedge\omega_{(\beta)} + m \big(\tfrac{e^{-\dd\Lambda}-1}{\dd\Lambda}\big)\wedge (\iota_{v_{(\beta)}} \Lambda + \lambda_{(\beta)} \wedge \Lambda)\big)\big]_5 \\
											&\phantom{=} + \big[ \tfrac{1}{2}s(\dd\Omega)\wedge \iota_{v_{(\beta)}}\dd\Omega \big]_5 \\
											&\phantom{=} - \big[m \big(\tfrac{e^{-\dd\Lambda}-1}{\dd\Lambda}\big) \wedge\Lambda\wedge \omega_{(\beta)} + m^2 \big(\tfrac{e^{-\dd\Lambda}-1+ \dd\Lambda}{\dd\Lambda\wedge \dd\Lambda}\big)\wedge \Lambda \wedge \iota_{v_{(\beta)}} \Lambda \big]_5 \ , \\ 
								\omega_{(\alpha)} &= e^{-\dd\Lambda} \wedge\omega_{(\beta)} + \iota_{v_{(\beta)}}\dd\Omega + ( \lambda_{(\beta)} + \iota_{ v_{(\beta)}}\dd\Lambda)\wedge \dd\Omega \\ 
												&\phantom{=} + m \big(\tfrac{e^{-\dd\Lambda}-1}{\dd\Lambda}\big)\wedge (\iota_{v_{(\beta)}} \Lambda + \lambda_{(\beta)} \wedge \Lambda) + m \big(\tfrac{e^{-\dd\Lambda}-1+ \dd\Lambda}{\dd\Lambda\wedge \dd\Lambda}\big)\wedge \Lambda\wedge \iota_{v_{(\beta)}} \dd\Lambda \ .
							\end{split}
						\end{equation*}
					%
				Setting $m=0$, these terms match precisely those following from~\eqref{patchingIIA} for the patching of the twisted generalised tangent space relevant to massless type IIA. 
				Keeping \hbox{$m \neq 0$}, we recover the corresponding terms of equation~\eqref{patching_m}. 

				Note however that by this procedure, one can construct the full twisted bundle $E$ only for compactifications on manifolds $M_d$ of dimension $d \leq 5$. 
 				Indeed, one can directly deduce the patching of the differential form parts of the generalised vector (which form a section of the bundle $E'$ in~\eqref{IIAexten}), but not the dual graviton charge, as there is no known treatment of the (non-linear) gauge transformations of the dual graviton field in an arbitrary background. 
				One can nevertheless infer the transformation of the $\tau$ component of the generalised vector by insisting that the patching is an $\E_{d+1 (d+1)}$ adjoint action. 
				In particular, for $m=0$ this yields:
					%
						\begin{equation*}
							\begin{split}
							\tau_{(\alpha)} =& \tau_{(\beta)} + j \dd\Lambda\wedge \sigma_{(\beta)} + j\dd\tilde\Lambda\wedge (\lambda_{(\beta)} + \iota_{ v_{(\beta)}}\dd\Lambda) \\
										& -js(\dd\Omega)\wedge\big(e^{-\dd\Lambda}\wedge\omega_{(\beta)} + \tfrac{1}{2}\iota_{ v_{(\beta)}}\dd\Omega + \tfrac{1}{2} (\lambda_{(\beta)} + \iota_{ v_{(\beta)}}\dd\Lambda) \wedge \dd\Omega \big)\, .
							\end{split}
						\end{equation*}
					%
	
			%
			%%%
			%%
			%
		\section{\texorpdfstring{Exceptional tangent bundle as extension of $\rmO(d,d)$ generalised geometry}{Exceptional bundle as extension of \rmO(d,d) generalised geometry}}\label{appsec:EGGodd}
			%
			As for the previous section, we stick with type IIA for concreteness, but the way of proceeding can be easily extended to type IIB.
			
			In the formulae for exceptional generalised geometry for (massive) type IIA, one can identify combinations of terms familiar from Hitchin's generalised geometry~\cite{hitch1,gualtphd}.
			We devote this section to showing how the exceptional generalised tangent space can be formulated as an extension of that introduced by Hitchin, by $\rmO(d,d)\times\RR^+$ tensor bundles. 
			This clarifies how exceptional geometry constructions like the Dorfman derivative~\eqref{dorfIIAm}, are built out of objects and operators naturally associated to these $\rmO(d,d)\times\RR^+$ generalised geometric bundles.

			Recall that Hitchin's generalised tangent space~\cite{hitch1,gualtphd}, which we denote by $E'$, has the structure of the extension~\eqref{gentanext},
				%
					\begin{equation}\label{eq:HitchinE}
						\begin{tikzcd}
							0 \arrow{r} &T^*M \arrow{r}{i} & E' \arrow{r}{\pi}& TM \arrow[bend left=50, color = red!60]{l}{B} \arrow{r}& 0 \ .
						\end{tikzcd}
					\end{equation}
				%
			The supergravity $B$-field (red arrow in the~\eqref{eq:HitchinE}) provides a splitting of the sequence and thus an isomorphism
				%
					\begin{equation}\label{eq:E'-isom}
						E' \cong T \oplus T^*\,.
					\end{equation}
				%
			As in~\cite{Coimbra:2011nw}, we will view $E'$ as an $\rmO(d,d)\times\RR^+$ vector bundle with zero $\RR^+$-weight. We normalise the $\RR^+$ weight by fixing the line bundle $L \cong \Lambda^d T^*$ to have unit weight. 
			The spinor bundles associated to $E'$ with weight $\frac{1}{2}$, denoted $S^\pm(E')_{\frac{1}{2}}$, can then be represented as local polyforms
				%
					\begin{equation}\label{eq:SE'-isom}
						S^\pm(E')_{\frac{1}{2}} \cong \Lambda^{\rm even/odd} T^*\,,
					\end{equation}
				%
			while (in six dimensions) there is also an isomorphism
				%
					\begin{equation}\label{eq:E'L-isom}
						E' \otimes L \cong \Lambda^5 T^* \oplus (T^* \otimes \Lambda^6 T^*)\,.
					\end{equation}
				%
			The bundles $S^\pm(E')_{\frac12}$ and $E' \otimes L$ are themselves naturally formed from extensions, and the isomorphisms~\eqref{eq:SE'-isom} and~\eqref{eq:E'L-isom} are also provided by the supergravity $B$ field.

			The (massive) type IIA exceptional generalised tangent space $E$ then fits into the exact sequences
				%
					\begin{equation}\label{eq:Odd-E}
						\begin{tikzcd}
							0 \arrow{r} &S^+(E')_{\frac{1}{2}} \arrow{r} &E'' \arrow{r}{\pi'} & E' \arrow{r} & 0 \, , \\
							0 \arrow{r} & E' \otimes L \arrow{r} &E \arrow{r} & E'' \arrow{r} & 0 \, .
						\end{tikzcd}
					\end{equation}
				%
			These give us a mapping
				%
					\begin{equation}\label{eq:Odd-anchor}
						\begin{aligned}
							\pi' : E &\rightarrow \; E' \\
							V &\mapsto X = v + \lambda\,,
						\end{aligned}
					\end{equation}
				%
			which serves as an analogue of the anchor map when viewing the exceptional generalised tangent space $E$ as an extension of $E'$. 

			Some useful $\rmO(d,d)\times\RR^+$ covariant maps can be defined as follows. 
			First, given a section $\tilde{b} \in L$, one has the mapping
				%
					\begin{equation}\label{eq:density-map}
						\begin{aligned}
							\tilde{b} : E' &\rightarrow \; E' \otimes L \\
							v + \lambda &\mapsto i_v \tilde{b} - \lambda \otimes \tilde{b}
						\end{aligned}
					\end{equation}
				%
			There is also a natural derivative
				%
					\begin{equation}\label{eq:d-sigma-map}
						\begin{tikzcd}
							\mathrm{der} : E' \otimes L \arrow{r}& L \\
							\tilde{X} = \sigma + \tau \arrow[mapsto]{r} & \langle \mathrm{der} , \tilde{X} \rangle = \dd \sigma ,
						\end{tikzcd}
					\end{equation}
				%
			which is the analogue of the (covariant) divergence of a vector density in Riemannian geometry, and a covariant pairing of spinors of opposite chirality
				%
					\begin{equation}\label{eq:Odd-spinor-bilinear}
						\begin{aligned}
							\langle (\dots), \Gamma^{(1)} (\dots) \rangle : S^+ (E')_{\frac{1}{2}} \otimes S^- (E')_{\frac{1}{2}} &\longrightarrow E' \otimes L \\
							\langle \omega , \Gamma^{(1)} \theta \rangle = - [s(\omega) \wedge \theta ]_5 - [j s(\omega)\wedge& \theta ]_{1,6}\,.
						\end{aligned}
					\end{equation}
				%
			The supergravity fields\footnote{%
				In this appendix we use the $A$-basis for the RR fields as we wish for the $B$ field to appear purely in the twisting of the $\rmO(d,d)$ bundles in~\eqref{eq:Odd-E} and not in defining the isomorphism~\eqref{eq:Odd-E-isom}.%
				}
			$A$ and $\tilde{B}$ are naturally collections of local sections of $S^- (E')_{\frac{1}{2}}$ and $L$ respectively, patched by the relevant supergravity gauge transformations. 
			These provide splittings of the sequences~\eqref{eq:Odd-E} and thus an isomorphism 
				%
					\begin{equation}\label{eq:Odd-E-isom}
						\begin{aligned}
							E &\cong E' \oplus S^+(E')_{\frac12} \oplus (E' \otimes L) \\
							V &\mapsto \tilde{X} + \tilde{\omega} + \tilde{\hat{X}}
						\end{aligned}
					\end{equation}
				%
			which is given explicitly in terms of the maps above as
				%
					\begin{equation}\label{eq:Odd-twist}
						\begin{aligned}
							\tilde{X} &= X \\
							\tilde{\omega} &= \omega - X \cdot A \\
							\tilde{\hat{X}} &= \tilde{X} - \tilde{B} \cdot X - \langle \omega - \tfrac12 X \cdot A, \Gamma^{(1)} A \rangle
						\end{aligned}
					\end{equation}
				%
			where $X\cdot A$ is the Clifford product. 

			Let us now show how to write the massless type IIA Dorfman derivative~\eqref{dorfIIA} in terms of natural operations in $\rmO(d,d)\times\RR^+$ generalised geometry. 
			Denote by $X = \pi' (V) = v + \lambda$ and $X' = \pi'(V') = v' + \lambda'$ the projections of the generalised vectors $V$ and $V'$ onto $E'$ using the mapping~\eqref{eq:Odd-anchor}. 
			The vector and one-form parts of~\eqref{dorfIIA} correspond to the $\rmO(d,d)$ Dorfman derivative $L_X X' = \mathcal{L}_v v' + \left(\mathcal{L}_v \lambda' - \iota_{v^\prime} \mathrm{d}\lambda\right)$~\eqref{adjDorf}, so that one has $\pi' (L_V V') = L_{\pi' (V)} \pi' (V')$. 
			This is reminiscent of the situation for the usual anchor map $\pi : E \rightarrow TM$, which satisfies $\pi (L_V V') = \mathcal{L}_{\pi(V)} \pi(V')$ so that the Dorfman derivative descends to the Lie derivative. 
			Here, the mapping $\pi'$ preserves the Dorfman derivative structure. 
			We remark that the map $\pi'$ and the Dorfman derivatives can be viewed as providing a generalisation of the notion of an algebroid, where one replaces the tangent bundle with Hitchin's generalised tangent bundle.

			The poly-forms $\omega$ and $\omega'$ are local sections of the $\rmO(6,6)$ spinor bundle $S^+(E')_{\frac{1}{2}}$, and these are treated in an $\rmO(6,6)$-covariant way in~\eqref{dorfIIA}. 
			Indeed, $L_X \omega' = (\mathcal{L}_v + \dd \lambda \wedge)\omega'$ is a spinorial Lie derivative in $\rmO(d,d)$ generalised geometry, while $(\iota_v + \lambda\wedge)\dd\omega$ is the Clifford action of the $\rmO(6,6)$ generalised vector $X$ on $\dd \omega$. 

			In six dimensions, the last two parts $\sigma$ and $\tau$ form a local section $\tilde{X}$ of $E' \otimes L$ as in~\eqref{eq:E'L-isom}. 
			We see that the $\rmO(d,d)\times\RR^+$ Dorfman derivative $L_X \tilde{X}' = \mathcal{L}_v \sigma' + \mathcal{L}_v \tau' + j \sigma' \wedge \dd \lambda$ accounts for some of the terms involving these in $L_V V'$. 	
			From~\eqref{eq:d-sigma-map}, one can write $\dd \sigma = \langle \mathrm{der} , X \rangle$, a section of $L$, which can act on $X'$ via the map~\eqref{eq:density-map}, to give $\langle \mathrm{der} , X \rangle (X') = i_{v'} \dd \sigma - \lambda'\otimes\dd\sigma$. 
			Finally, the exterior derivative gives the natural $\rmO(d,d)\times\RR^+$ Dirac operator $S^+(E')_{\frac{1}{2}} \rightarrow S^-(E')_{\frac{1}{2}}$ and the pairing between $\omega'$ and $\dd\omega$ in the first and second lines of \eqref{dorfIIA} is the $O(6,6)\times\RR^+$ invariant given in~\eqref{eq:Odd-spinor-bilinear}.

			Putting all of this together, we can write the Dorfman derivative in terms of $\rmO(d,d)\times\RR^+$ objects as
				%
					\begin{equation}\label{eq:Odd-Edd-Dorfman}
							L_V V' = L_X X' + (L_X \omega' - X' \cdot \dd \omega) + (L_X \tilde{X}' - \langle \mathrm{der} , \tilde{X} \rangle (X') - \langle \omega' , \Gamma^{(1)} \dd \omega \rangle)\,.
					\end{equation}
				%
			This can be easily enhanced to include the mass terms in~\eqref{dorfIIAm}. 
			The mass can be viewed as a local section of the spinor bundle $S^+(E')_{\frac12}\cong \Lambda^{\rm even} T^*$ and we can write the massive version of~\eqref{eq:Odd-Edd-Dorfman} as
				%
					\begin{equation}\label{eq:massive-Odd-Edd-Dorfman}
						\begin{split}
							L_V V' &= L_X X' + \Big[ L_X \omega' - X' \cdot (\dd\omega - X\cdot m)\Big] \\
		 						& \phantom{=}+ \Big[ L_X \tilde{X}' - \big(\langle \mathrm{der} , \tilde{X} \rangle + \langle \omega , m \rangle\big)(X') - \langle \omega' , \Gamma^{(1)} (\dd \omega - X\cdot m)\rangle \Big]\,.
						\end{split}
					\end{equation}
				%
			Finally, we remark that the projected generalised metric appearing in~\eqref{eq:GB-metric} is formalised by the construction of this appendix as $\mathcal{H}^{-1} \in S^2(E')$, which is the image of the exceptional generalised metric $G^{-1} \in S^2(E)$ in the anchor-like mapping $\pi' : E \rightarrow E'$ from~\eqref{eq:Odd-anchor}. 
			This is much like the first line of~\eqref{invG_comp_1}, where $e^{2\Delta} g^{-1} \in S^2(TM)$ is seen to be the image of $G^{-1}$ in the anchor map $\pi : E \rightarrow TM$.
			%
			%%%
			%%
			%
		\section{Generalised geometry for type IIB}\label{appsec:EGGIIB}
			%
			We have seen in the main text that the generalised tangent bundle has sections transforming in the fundamental of $\E_{d+1(d+1)} \times \RR^+$.
			On a $d$-dimensional manifold it reads
				%
					\begin{equation*}
						\begin{split}
							E 	& \cong TM\oplus T^{*}M\oplus(T^{*}M\oplus\Lambda^{3}T^{*}M\oplus\Lambda^{5}T^{*}M)\oplus\Lambda^{5}T^{*}M\oplus(T^{*}M\otimes\Lambda^{6}T^{*}M)\\
 								& \cong TM\oplus(T^{*}M\otimes S)\oplus\Lambda^{3}T^{*}M\oplus(\Lambda^{5}T^{*}M\otimes S)\oplus(T^{*}M\otimes\Lambda^{6}T^{*}M)\, ,
						\end{split}
					\end{equation*}
				%
			where $S$ transforms as a doublet of $\SL{2}$.
			Formally it is defined by the series of short exact sequences,
				%
					\begin{equation}\label{extIIB}
						\begin{tikzcd}[row sep=tiny]
							0 \arrow{r} & T^*M \arrow{r} & E''' \arrow{r} &TM \arrow{r} &0 	\, , \\
							0 \arrow{r} &\Lambda^{\mathrm{odd}} T^*M \arrow{r} &E'' \arrow{r} &E''' \arrow{r} &0 	\, , \\
							0 \arrow{r} &\Lambda^5 T^*M \arrow{r} &E' \arrow{r} &E'' \arrow{r} &0 \, . \\
							0 \arrow{r} &T^*M \otimes \Lambda^6 T^*M \arrow{r} &E \arrow{r} &E' \arrow{r} &0 \, .
						\end{tikzcd}
					\end{equation}
				%
			analogously to the type IIA case.
			The sequences are split by maps living in the adjoint bundle,
				%
					\begin{equation}
						\begin{split}
							\adj \tilde{F} = & \RR \oplus(TM\otimes T^{*}M) \oplus(S\otimes S^{*})_0 \oplus(S\otimes\Lambda^{2}TM) \oplus(S\otimes\Lambda^{2}T^{*}M) \\
										& \oplus\Lambda^{4}TM\oplus\Lambda^{4}T^{*}M \oplus(S \otimes\Lambda^{6}TM)\oplus(S\otimes\Lambda^{6}T^{*}M) \, ,
						\end{split}
					\end{equation}
				%
			where the subscript on $(S\otimes S^*)_0$ denotes the traceless part. 
			We write adjoint sections as in~\eqref{eq:IIB-adj},
				%
					\begin{equation}
						R=l+r+a+\beta^{i}+B^{i}+\gamma+C+\tilde{\alpha}^{i}+\tilde{a}^{i} \, .
					\end{equation}
				%
				
			Taking $\{\hat{e}_{a}\}$ to be a basis for $TM$ with a dual basis $\{e^{a}\}$ on $T^{*}M$, one can define a natural $\mathfrak{gl}{d}$ action on tensors.
			
			The $\mathfrak{e}_{d+1(d+1)}$ subalgebra is generated by setting $l=r_{\phantom{a}a}^{a}/(8-d)$. This fixes the weight of generalised tensors under the $\mathbb{R}^+$ factor, so that a scalar of weight $k$ is a section of $(\det T^{*}M)^{k/(8-d)}$
				%
					\begin{equation}\label{RplusIIB}
						\mathbf{1}_k \in \Gamma\Bigl( (\det T^{*}M)^{k/(8-d)} \Bigr).
					\end{equation}
				%
			We define the adjoint action of $R\in\Gamma(\adj \tilde{F})$ on $V\in \Gamma(E)$ to be $V^{\prime} =R \cdot V$. 
			Then, the components of $V^{\prime}$ are
				%
				\begin{equation}\label{eq:IIB_adjoint}
						\begin{split}
							v^{\prime} & =lv+r\cdot v+\gamma\lrcorner\rho+\epsilon_{ij}\beta^{i}\lrcorner\lambda^{j}+\epsilon_{ij}\tilde{\alpha}^{i}\lrcorner\sigma^{j},\\
							\lambda^{\prime i} & =l\lambda^{i}+r\cdot\lambda^{i}+a_{\phantom{i}j}^{i}\lambda^{j}-\gamma\lrcorner\sigma^{i}+v\lrcorner B^{i}+\beta^{i}\lrcorner\rho-\tilde{\alpha}^{i}\lrcorner\tau,\\
							\rho^{\prime} & =l\rho+r\cdot\rho+v\lrcorner C+\epsilon_{ij}\beta^{i}\lrcorner\sigma^{j}+\epsilon_{ij}\lambda^{i}\wedge B^{j}+\gamma\lrcorner\tau,\\
							\sigma^{\prime i} & =l\sigma^{i}+r\cdot\sigma^{i}+a_{\phantom{i}j}^{i}\sigma^{j}-C\wedge\lambda^{i}+\rho\wedge B^{i}+\beta^{i}\lrcorner\tau+v\lrcorner\tilde{a}^{i},\\
							\tau^{\prime} & =l\tau+r\cdot\tau+\epsilon_{ij}j\lambda^{i}\wedge\tilde{a}^{j}-j\rho\wedge C-\epsilon_{ij}j\sigma^{i}\wedge B^{j}.
						\end{split}
					\end{equation}
				%
			In addition, one defines the adjoint action of $R$ on $R'$ to be the commutator $R''=[R,R']$.
			As consequence, the components of $R''$ are
				%
					\begin{equation}\label{eq:ad-ad-IIB}
						\begin{split}
							l^{\prime\prime} 								& =\tfrac{1}{2}(\gamma\lrcorner C^{\prime}-\gamma^{\prime}\lrcorner C)+\tfrac{1}{4}\epsilon_{kl}(\beta^{k}\lrcorner B^{\prime l}-\beta^{\prime k}\lrcorner B^{l})+\tfrac{3}{4}\epsilon_{ij}(\tilde{\alpha}^{i}\lrcorner\tilde{a}^{\prime j}-\tilde{\alpha}^{\prime i}\lrcorner\tilde{a}^{j})\, ,\\
							%
							r^{\prime\prime} 						& =(r\cdot r^{\prime}-r^{\prime}\cdot r)+\epsilon_{ij}(j\beta^{i}\lrcorner jB^{\prime j}-j\beta^{\prime i}\lrcorner jB^{j})-\tfrac{1}{4}\id\epsilon_{kl}(\beta^{k}\lrcorner B^{\prime l}-\beta^{\prime k}\lrcorner B^{l})\\
 											& \phantom{=} +(j\gamma\lrcorner jC^{\prime}-j\gamma^{\prime}\lrcorner jC)-\tfrac{1}{2}\id(\gamma\lrcorner C^{\prime}-\gamma^{\prime}\lrcorner C)\\
 											& \phantom{=} +\epsilon_{ij}(j\tilde{\alpha}^{i}\lrcorner j\tilde{a}^{\prime j}-j\tilde{\alpha}^{\prime i}\lrcorner j\tilde{a}^{j})-\tfrac{3}{4}\epsilon_{ij}(\tilde{\alpha}^{i}\lrcorner\tilde{a}^{\prime j}-\tilde{\alpha}^{\prime i}\lrcorner\tilde{a}^{j})\, ,\\
							%
							a_{\phantom{\prime\prime i}j}^{\prime\prime i} 	& =(a\cdot a^{\prime}-a^{\prime}\cdot a)_{\phantom{i}j}^{i}+\epsilon_{jk}(\beta^{i}\lrcorner B^{\prime k}-\beta^{\prime i}\lrcorner B^{k})-\tfrac{1}{2}\delta_{\phantom{i}j}^{i}\epsilon_{kl}(\beta^{k}\lrcorner B^{\prime l}-\beta^{\prime k}\lrcorner B^{l})\\
											& \phantom{=} +\epsilon_{jk}(\tilde{\alpha}^{i}\lrcorner\tilde{a}^{\prime k}-\tilde{\alpha}^{\prime i}\lrcorner\tilde{a}^{k})-\tfrac{1}{2}\delta_{\phantom{i}j}^{i}\epsilon_{kl}(\tilde{\alpha}^{k}\lrcorner\tilde{a}^{\prime l}-\tilde{\alpha}^{\prime k}\lrcorner\tilde{a}^{l})\, ,\\
							%
							\beta^{\prime\prime i} 					& =(r\cdot\beta^{\prime i}-r^{\prime}\cdot\beta^{i})+(a\cdot\beta^{\prime}-a^{\prime}\cdot\beta)^{i}-(\gamma\lrcorner B^{\prime i}-\gamma^{\prime}\lrcorner B^{i})-(\tilde{\alpha}^{i}\lrcorner C^{\prime}-\tilde{\alpha}^{\prime i}\lrcorner C)\, ,\\
							%
							B^{\prime\prime i} 						& =(r\cdot B^{\prime i}-r^{\prime}\cdot B^{i})+(a\cdot B^{\prime}-a^{\prime}\cdot B)^{i}+(\beta^{i}\lrcorner C^{\prime}-\beta^{\prime i}\lrcorner C)-(\gamma\lrcorner\tilde{a}^{\prime i}-\gamma^{\prime}\lrcorner\tilde{a}^{i})\, ,\\
							%
							\gamma^{\prime\prime} 					& =(r\cdot\gamma^{\prime}-r^{\prime}\cdot\gamma)+\epsilon_{ij}\beta^{i}\wedge\beta^{\prime j}+\epsilon_{ij}(\tilde{\alpha}^{i}\lrcorner B^{\prime j}-\tilde{\alpha}^{\prime i}\lrcorner B^{j})\, ,\\
							%
							C^{\prime\prime} 						& =(r\cdot C^{\prime}-r^{\prime}\cdot C)-\epsilon_{ij}B^{i}\wedge B^{\prime j}+\epsilon_{ij}(\beta^{i}\lrcorner\tilde{a}^{\prime j}-\beta^{\prime i}\lrcorner\tilde{a}^{j})\, ,\\
							%
							\tilde{\alpha}^{\prime\prime i} 				& =(r\cdot\tilde{\alpha}^{\prime i}-r^{\prime}\cdot\tilde{\alpha}^{i})+(a\cdot\tilde{\alpha}^{\prime}-a^{\prime}\cdot\tilde{\alpha})^{i}-(\beta^{i}\wedge\gamma^{\prime}-\beta^{\prime i}\wedge\gamma)\, ,\\
							%
							\tilde{a}^{\prime\prime i} 					& =(r\cdot\tilde{a}^{\prime i}-r^{\prime}\cdot\tilde{a}^{i})+(a\cdot\tilde{a}^{\prime}-a^{\prime}\cdot\tilde{a})^{i}+(B^{i}\wedge C^{\prime}-B^{\prime i}\wedge C) \, .
						\end{split}
					\end{equation}
				%
			
			The generalised Lie derivative is defined in~\eqref{eq:Dorf-def-IIB}. 
			The twisted Dorfman derivative on a generic tensor can be written as
				%	
					\begin{equation}\label{eq:twisted_untwisted_lie_der}
							\mathbb{L}_{\tilde V} \tilde{\alpha} = \mathcal{L}_{\tilde v} \tilde{\alpha} - R_{\mathbb{L}_{\tilde V}} \cdot \tilde{\alpha}\, ,
					\end{equation}
				%
			where $R_{\mathbb{L}_{\tilde V}}$ is the adjoint element 
				%
					\begin{equation}\label{eq:tensor_r}
						R_{\mathbb{L}_{\tilde V}} = \dd \tilde{\lambda}^i -\iota_{\tilde v} F^i + \dd \tilde{\rho} - \iota_{\tilde v} F - \epsilon_{ij} \tilde{\lambda}^i \wedge F^j + \dd \tilde{\sigma}^i +\tilde{\lambda}^i \wedge F - \tilde{\rho} \wedge F^i \, . 
					\end{equation}
				%
			Thus, its action on a generic adjoint element $R$ reads
			 	%
					\begin{equation}\label{eq:IIB_Dorf_adjoint}
						\begin{split}
							L_{V}R = & (\mathcal{L}_{v} l + \tfrac{1}{2}\gamma\lrcorner \dd\rho + \tfrac{1}{4}\epsilon_{kl}\beta^{k}\lrcorner\dd\lambda^{l}+\tfrac{3}{4}\epsilon_{kl}\tilde{\alpha}^{k}\lrcorner\dd\sigma^{l})\\
 								& +(\mathcal{L}_{v}r+j\gamma\lrcorner j\dd\rho-\tfrac{1}{2}\id\gamma\lrcorner\dd\rho+\epsilon_{ij}j\beta^{i}\lrcorner j\dd\lambda^{j}-\tfrac{1}{4}\id\epsilon_{kl}\beta^{k}\lrcorner\dd\lambda^{l}\\
								& \phantom{{}+({}}+\epsilon_{ij}j\tilde{\alpha}^{i}\lrcorner j\dd\sigma^{j}-\tfrac{3}{4}\id\epsilon_{kl}\tilde{\alpha}^{k}\lrcorner\dd\sigma^{l})\\
 								& +(\mathcal{L}_{v}a_{\phantom{i}j}^{i}+\epsilon_{jk}\beta^{i}\lrcorner\dd\lambda^{k}-\tfrac{1}{2}\delta_{\phantom{i}j}^{i}\epsilon_{kl}\beta^{k}\lrcorner\dd\lambda^{l}+\epsilon_{jk}\tilde{\alpha}^{i}\lrcorner\dd\sigma^{k}-\tfrac{1}{2}\delta_{\phantom{i}j}^{i}\epsilon_{kl}\tilde{\alpha}^{k}\lrcorner\dd\sigma^{l})\\
 								& +(\mathcal{L}_{v}\beta^{i}-\gamma\lrcorner\dd\lambda^{i}-\tilde{\alpha}^{i}\lrcorner\dd\rho)\\
 								& +(\mathcal{L}_{v}B^{i}+r\cdot\dd\lambda^{i}+a_{\phantom{i}j}^{i}\dd\lambda^{j}+\beta^{i}\lrcorner\dd\rho-\gamma\lrcorner\dd\sigma^{i})\\
 								& +(\mathcal{L}_{v}\gamma+\epsilon_{ij}\tilde{\alpha}^{i}\lrcorner\dd\lambda^{j})\\
 								& +(\mathcal{L}_{v}C+r\cdot\dd\rho+\epsilon_{ij}\dd\lambda^{i}\wedge B^{j}+\epsilon_{ij}\beta^{i}\lrcorner\dd\sigma^{j})+(\mathcal{L}_{v}\tilde{\alpha}^{i})\\
 								& +(\mathcal{L}_{v}\tilde{a}^{i}+r\cdot\dd\sigma^{i}+a_{\phantom{i}j}^{i}\dd\sigma^{j}-\dd\lambda^{i}\wedge C+B^{i}\wedge\dd\rho)\, .
						\end{split}
					\end{equation}
				%
			
			For $\E_{5(5)}$, we also need the vector bundle transforming in the $\mathbf{10}_{2}$ representation of $\mathrm{Spin}{5,5}\times\RR^{+}$. 
			We define this bundle as
				%
					\begin{equation}
						N\simeq S\oplus\Lambda^{2}T^{*}M\oplus S\otimes\Lambda^{4}T^{*}M.
					\end{equation}
				%
			We write sections of the $N$ bundle as
				%
					\begin{equation}
						Q=m^{i}+n+p^{i},
					\end{equation}
				%
			where $m^{i}\in\Gamma(S)$, $n\in\Gamma(\Lambda^{2}T^{*}M)$ and $p^{i}\in\Gamma(S\otimes\Lambda^{4}T^{*}M)$. 
			We define the adjoint action of $R\in\Gamma(\adj \tilde{F})$ on $Q\in\Gamma(N)$ to be $Q'=R\cdot Q$, with components
				%
					\begin{equation}\label{Q_adj_IIB}
						\begin{split}
							m'^{i} & =2lm^{i}+a_{\phantom{i}j}^{i}m^j+\beta^{i}\lrcorner n-\gamma\lrcorner p^{i},\\
							n' & =2ln+r\cdot n+\epsilon_{ij}\beta^{i}\lrcorner p^{j}+\epsilon_{ij}m^{i}B^{j},\\
							p'^{i} & =2lp^{i}+r\cdot p^{i}+a_{\phantom{i}j}^{i}p^{j}+B^{i}\wedge n-m^{i}C.
						\end{split}
					\end{equation}
				%
			Using $\mathrm{16}^{c} \times \mathrm{10}\rightarrow\mathrm{16}$, we define a projection to $E$ as
				%
					\begin{equation}
						\times_E : E^{*}\otimes N\rightarrow E \, .
					\end{equation}
				%
			Explicitly, as a section of $E$, this allows us to define 
				%
					\begin{equation}\label{e5_proj_IIB}
						\dd Q \coloneqq \partial\times_E Q = \dd m^{i}+\dd n \, .
					\end{equation}
				%
			
			Analogously to the $\rmO(d,d)$ $\eta$ metric, in $\E_{d(d)}$ geometry one has several invariants.
			
			The quadratic invariant for $\E_{5(5)}$ is
				%
					\begin{equation}\label{eq:IIB_quadratic}
						\zeta(Q,Q)=\epsilon_{ij}m^i p^j -\tfrac{1}{2} n\wedge n\, .
					\end{equation}
				%
			The cubic invariant for $\E_{6(6)}$ is
				%
					\begin{equation}\label{eq:IIB_cubic}
						c(V,V,V)=-\tfrac{1}{2}(\imath_{v}\rho\wedge\rho+\epsilon_{ij}\rho\wedge\lambda^{i}\wedge\lambda^{j}-2\epsilon_{ij}\imath_{v}\lambda^{i}\sigma^{j}) \, .
					\end{equation}
				%
			Finally, the symplectic invariant for $\E_{7(7)}$ is
				%
					\begin{equation}\label{eq:IIB_symplectic}
						s(V,V^{\prime})=-\tfrac{1}{4}\bigl((\iota_{v}\tau^{\prime}-\iota_{v^{\prime}}\tau)+\epsilon_{ij}(\lambda^{i}\wedge\sigma^{\prime j}-\lambda^{\prime i}\wedge\sigma^{j})-\rho\wedge\rho^{\prime}\bigr) \, .	
					\end{equation}
				%
			The $\mathfrak{e}_{d+1(d+1)}$ Killing form is given by the trace operator
				%
					\begin{equation}\label{eq:IIB_Killing}
						\begin{split}
							\tr(R,R^{\prime}) =& \tfrac{1}{2}\Bigl(\tfrac{1}{8-d}\tr(r)\tr(r^{\prime})+\tr(rr^{\prime})+\tr(aa^{\prime})+\gamma\lrcorner C^{\prime}+\gamma^{\prime}\lrcorner C \\
 											& \phantom{\frac{1}{2}\biggl[} +\epsilon_{ij}(\beta^{i}\lrcorner B^{\prime j}+\beta^{\prime i}\lrcorner B^{j}) +\epsilon_{ij}(\tilde{\alpha}^{i}\lrcorner\tilde{a}^{\prime j}+\tilde{\alpha}^{\prime i}\lrcorner\tilde{a}^{j})\Bigr)\, .
						\end{split}
					\end{equation}
				%

		\section{Type IIB parallelisation on $S^3$}\label{IIBonS3}

In this appendix, we present a parallelisation of the type IIB generalised tangent bundle on $S^3$ which satisfies an $\SO(4)$ gauge algebra.
 A consistent truncation of type IIB supergravity on $S^3$ down to $\SO(4)$ maximal supergravity in seven dimensions has recently been worked out in~\cite{Malek:2015hma} adopting an exceptional field theory approach. This was related to the $S^3$ reduction of massless type IIA by an external automorphism of $\SL(5)$ exchanging the ${\bf 10}\subset{\bf 15}$ and the ${\bf 10'}\subset {\bf 40'}$ representations. 
 Here we show that this type IIB truncation can also be understood in terms of generalised parallelisations.


The type IIB generalised tangent bundle $E$ on a three-dimensional manifold $M_3$ is
\begin{equation}
\label{gb}
E \cong T \oplus T^* \oplus T^* \oplus \Lambda^3 T^*\, ,
\end{equation}
and has structure group $\E_{4(4)} \times \RR^+\cong \SL\left(5,\mathbb{R}\right)\times \RR^+$.
A generalised vector transforms in the $\mathbf{10}_1$ representation and can be written as
\begin{equation*}
V = v + \lambda + \rho + \zeta\ ,
\end{equation*}
where $v\in T$, $\lambda \in T^*$, $\rho \in T^*$, and $\zeta\in \Lambda^3 T^*$.
The relevant Dorfman derivative can be obtained by truncating to three dimensions the type IIB, five-dimensional Dorfman derivative given in~\cite{spheres,AshmoreECY}. This yields
\begin{equation}
\label{dorfb3}
\begin{split}
L_V V' =& \mathcal{L}_v v' + \left(\mathcal{L}_v \lambda' - \iota_{v^\prime} \mathrm{d}\lambda \right) \\
& + \left(\mathcal{L}_v \rho' - \iota_{v^\prime} \mathrm{d}\rho \right) + \left(\mathcal{L}_v \zeta' + \mathrm{d}\lambda \wedge \rho'\right)\, .
\end{split}
\end{equation}

As in the type IIA example discussed in section~\ref{S3and7dsugra}, we decompose the generalised frame $\hat{E}_{IJ}$, $I,J =1,\ldots, 5$, under $\SL\left(4,\mathbb{R}\right)$ as $\{E_{ij} , E_{i5}\}$, with  $i=1,\ldots,4$.
Then we define a generalised parallelisation on $S^3$ as
\begin{equation}
\label{glpb}
\hat{E}_{IJ} = \begin{cases}\  \hat{E}_{ij} = v_{ij} + \rho_{ij} + \iota_{v_{ij}}\!\! \rg{B}\, ,\\
\hat{E}_{i5} = R \dd y_i + y_i \rg{\vol}_3 + R \dd y_i \wedge \rg{B} \, , \end{cases}
\end{equation}
with
\begin{equation}
\rho_{ij} \,=\, \rg{*}(R^2 \dd y_i \wedge\dd y_j)   \,=\, R\, \epsilon_{ijkl}\, y^k\mathrm{d}y^l \, .
\end{equation}
Here, $\hat{E}_{ij}$ parallelises $T \oplus T^*$, that is the part of the generalised tangent bundle common to type IIA, while $\hat{E}_i$ is a parallelisation on the complement $T^* \oplus \Lambda^3 T^*$.
As in section~\ref{S3and7dsugra}, the background two-form potential $\rg{B}$ is chosen such that 
\begin{equation}
\rg{H} \ =\, \dd\! \rg{B}\, \ = \, \frac{2}{R}\rg{\vol}_3\ 
\end{equation}
(we could also have twisted by a background RR two-form potential $\rg{C_2}$).

Evaluating the Dorfman derivative on the frame~\eqref{glpb}, we obtain
\begin{equation}
\begin{split}
\hat{E}_{kl} &= 2R^{-1}\big(\delta_{i[k}\hat{E}_{l]j} -  \delta_{j[k}\hat{E}_{l]i} \big)\ ,\\[1mm]
\hat{E}_{k5}&= -2R^{-1}\delta_{k[i}\hat{E}_{j]5} \ , \\[1mm]
\hat{E}_{kl} &=0 \, , \\[1mm]
\hat{E}_{k5}&= 0 \, ,
\end{split}
\end{equation}
which corresponds to an $SO(4)$ frame algebra.\footnote{This is the same algebra satisfied by the massive IIA generalised parallelisation on $S^3$ discussed in section~\ref{massive_algebras} (cf.~eq.~\eqref{massive_algebra_S3}) -- however in that case the parallelisation fails to be an $\SL(5)$ frame.} This is consistent with the $\SO(4)$ gauging of $D=7$ maximal supergravity originally discussed in~\cite{Samtleben:2005bp}. To see this, it is convenient to dualise $\hat E_{ij}$ to $\widetilde{E}_{ij} = \frac{1}{2} \epsilon_{ij}{}^{kl}\hat{E}_{kl}$. Also renaming $\widetilde{E}_{i5}= \hat{E}_{i5}$, the frame algebra now reads
\begin{equation}
L_{\widetilde{E}_{II'}} \widetilde{E}_{JJ'} \, =\, -X_{[II'][JJ']}{}^{[KK']} \widetilde{E}_{KK'}\ ,
\end{equation}
with
\begin{align}\label{strconst_m_S3}
& & X_{[II'][JJ']}{}^{[KK']} = -4 \, \epsilon_{5II'L[J}w^{L[K}\delta_{J']}^{K']}\ ,& &
w^{IJ} = \frac{1}{2R} \mathrm{diag}\big(1,1,1,1,0\big)\ , & & 
\end{align}
which matches the embedding tensor given in~\cite{Samtleben:2005bp} for the $\SO(4)$ gauging. 
		%
\end{document}