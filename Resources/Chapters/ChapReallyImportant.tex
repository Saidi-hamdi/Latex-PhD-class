\documentclass[debug]{phd}

\begin{document}
%%
\ensurepagenumbering{arabic}
%%
	%
	\chapter{Flux Compactifications}
	\label{chapSugra}
		%
		\section{Introduction and motivations}
			%
						
			This thesis is devoted to the study of supersymmetric compactifications with non-trivial fluxes.
			We will see in the first part of this chapter how requiring some amount of supersymmetry on the lower dimensional theory constrains the geometry of the internal manifold $M$, such that it must admit geometrical structures like the ones we described in~\cref{chap1}.
			For the well-known case of fluxless compactification of a $10$-dimensional type II supergravity to a minimal supergravity in $4$ dimensions, the constraints on the internal manifold requires it to be a Calabi-Yau three-fold~\cite{CYcomp}.
			When we allow fluxes to be turned on, the supersymmetry conditions can be cast in a compact and elegant form using \emph{Generalised Geometry} and generalised structures we will introduce in~\cref{chapEGG}.
			%
		\section{Supergravity theories}
			%
			Supergravity theories are theories combining general relativity with supersymmetry (making this a local symmetry).
			These can be seen as low-energy effective theories of the different string theories.
			There exists also an eleven-dimensional maximally supersymmetric supergravity, which is not connected (as low-energy limit) to any string theory.
			This has been interpreted to have its higher dimensional origin in $M$-theory.
			
			The aim of this section is to describe the main feature of type II and eleven-dimensional supergravity theories, with their effective actions and the gauge symmetries of their potentials.
			%
			\subsection{Eleven-dimensional supergravity}
				%
				This section is devoted to the description of eleven-dimensional supergravity, \emph{i.e.} the low energy effective theory of M-theory. 
				This is not meant to be an exhaustive treatment and we refer to~\cite{polchinski, BeckerBeckerSchw} for further details.

				The bosonic degrees of freedom of eleven dimensions supergavity consist of the metric $g$, a three-form potential $A$ and its dual.
				Although it does not transport independent degrees of freedom, one often introduces the dual seven-form $\tilde G$, whose six-form potential is conventionally denoted by $\tilde A$.			
	
				The theory is invariant under diffeomorphisms and the gauge transformations			
					%
					 	\begin{equation}
							A_3 \longrightarrow A_3 + \dd \Lambda_2 \, ,
						\end{equation}
					%
				where $\Lambda_2$ is a two-form. 
				The gauge invariant field strength is $G_4 = \dd A_3$.
				
				The bosonic action of the eleven-dimensional supergravity is
					%
						\begin{equation}
							\begin{split}
								S_{11} =& \frac{1}{2\kappa^2} \int \dd^{11} x \ \sqrt{g} \left[ \left(R - \frac{1}{2} \lvert G \rvert^2 \right)\right] - \\
								& - \underbrace{\frac{1}{6} G_4 \wedge G_4 \wedge A}_{S_{CS}} \, , 
							\end{split}
						\end{equation}
					%
				
The equation of motion and Bianchi identity can be written (in a sourceless case) as
					%
						\begin{equation}
							\begin{split}
								\dd &\star G + \frac{1}{2} G \wedge G = 0 \, , \\
								\dd &G = 0 \, .
							\end{split}
						\end{equation}
					%
The theory is supersymmetric, with $\mathcal{N}=1$ supersymmetry. Notice that this is the maximal possible supersymmetry in eleven dimensions.
The fermionic degrees of freedom are completely captured by the gravitino $\Psi$. 
				
				
				In addition, the equation of motion for the metric $g$, \emph{i.e.} the Einstein equation, can be written as follows,
					%
						\begin{align}
							R_{MN} - \dfrac{1}{12}\left( G_{MPQR}G_{N}^{\phantom{N}PQR} - \dfrac{1}{12}g_{MN} G^2 \right) &= 0 \, .
						\end{align}
					%
				%%	
				%
			\subsection{Type II theories}\label{sec:IIsugra}
				%
Type II supergravities are the ten-dimensional effective theories for massless fields type II string theories. There are two such theories that differ in the chirality of
the fermionic fields and the rank of the form potentials. 

The bosonic sector consists of two sets of fields: the Neveu-Schwarz Neveu-Schwarz (NSNS) and the Ramond-Ramond (RR).
					%
						\begin{table}[h!]
						\centering
							\begin{tabular}{c c c c}
									\toprule
%														&											&										\\
%									\midrule
														& 			$g$								&	metric (graviton)							\\[1.2mm]
									NSNS 				&			$B$								&	Kalb-Ramond $2$-form						\\[1.2mm]
											 			&			$\phi$							&	dilaton									\\[1.6mm]
									RR		 			&			$A_p$							&	\begin{tabular}{@{\ }l@{}}
 																											$p$ odd for type IIA \\ 
																											$p$ even for type IIB 
 																										\end{tabular}								\\[2mm]
									\midrule	
														&	$\psi_M^{\alpha, +}$, $\psi_M^{\alpha, \mp}$		&	Gravitinos									\\[1.2mm]
														&	$\lambda_\alpha^{-}$, $\lambda_\alpha^{\pm}$		&	Dilatinos									\\[1.2mm]
									\bottomrule
								\end{tabular}
							\caption{Type II supergravities spectrum in ten dimensions. 
							The different chiralities of spinors define the two theories. 
							Upper signs refer to type IIA, while lower ones to type IIB.}
							\label{tabspectr}
						\end{table}
					%			
				
				As one can see from~\cref{tabspectr}, the NSNS sector is the same for both type II theories.
				It contains the metric $g$, the dilaton $\phi$ and the NSNS two-form $B$.
				The latter is a $\U(1)$ gauge potential with field strength $H = \dd B$.
				
				The RR sector depends on the theory.
				Type IIA contains odd forms, while for type IIB has even ones.
				These are also $\U(1)$ gauge potentials. For applications to generalised geometry is it convenient to use the \emph{democratic formulation}~\cite{DemSugra},
of supergravity. 		This formulation considers RR potentials of all ranks $C_p$, with $p = 1, 3, \ldots, 9$ for type IIA and $p = 0, 2, \ldots, 8$ for IIB. 
				These are not all independent since their field strengths have to satisfy duality relations with respect to the Hodge dual.
				The field strength\footnote{%
					There exists another common choice for the RR potential, the so-called $A$-basis, which is related to the $C$-basis we use as $A = e^{-B} \wedge C$. 
					In this basis the field strength~\eqref{Fform} reads $F = e^B \wedge (\dd A + m)$.%
					}
				are defined by,
					%
						\begin{equation}\label{Fform}
							F_p = \dd C_{p-1} + H \wedge C_{p-3} + e^B F_0 \, ,
						\end{equation}
					%
				where $F_0 = m$ is the Romans mass, which can be added only in type IIA~\cite{RomansMass}, and 
				the duality relations,
					%
						\begin{equation}\label{Fdual}
							F_p = (-1)^{\left[\frac{p+3}{2}\right]} \star F_{10-p} \, .
						\end{equation}
					%

				 
				 The fermionic sector of the two theories consists of two Majorana-Weyl spinors of spin $3/2$, the gravitinos $\psi_M^\alpha$, and two Majorana-Weyl spin 1/2 spinors $\lambda^\alpha$, the dilatinos.
				 Gravitinos and dilatinos have opposite chirality.
				 In type IIA the gravitinos have opposite chirality, while in type IIB they have the same chirality (chosen positive by convention).
				 As a consequence, type IIB dilatinos will both have negative chirality. 
				 This is the difference between type IIA and IIB, the former is a non-chiral theory, while the latter is chiral.
				 Nevertheless, they are both maximal supersymmetry in ten dimensions, \emph{i.e.} they are $\mathcal{N}=2$.
				 An important fact is that type IIA supergravity can be obtained by the eleven-dimensional one by a compactifiation on a circle.
				 We will analyse this reduction in a while.
				 
				 The string frame\footnote{%
				 	Einstein frame and string frame metric are related by a dilaton rescaling, \emph{i.e.}
						%
							\begin{equation*}
								g = e^{\phi/2} g^E \, .
							\end{equation*}
						%
					} action for the bosonic fields of type IIA is (we follow the conventions of~\cite{DemSugra}) 
				 	%
						\begin{equation*}
							\begin{split}
								S_{IIA} = \frac{1}{2\kappa^2}& \int \dd^{10} x \ \underbrace{\sqrt{g} \left[ e^{-2\phi} \left(R + 4 \nabla \phi ^2 - \frac{1}{2} \lvert H \rvert^2 \right)\right]}_{S_{NS}} - \underbrace{\frac{\sqrt{g}}{2} \sum_{k=0}^2 \lvert F_{2k} \rvert^2}_{S_{R}} \\
								&- \underbrace{\frac{1}{2} B \wedge \mathcal{F}_4 \wedge \mathcal{F}_4}_{S_{CS}} \, ,
							\end{split}		
						\end{equation*}
					%
				 where $\mathcal{F}_4 = \dd C_3$, while $F_p$ are the field strength defined above in~\eqref{Fform}.

				The bosonic action for type IIB reads 
					%
						\begin{equation*}
							\begin{split}
							S_{IIB} = \frac{1}{2\kappa^2}& \int \dd^{10} x \ \underbrace{\sqrt{g} \left[ e^{-2\phi} \left(R + 4 \nabla \phi ^2 - \frac{1}{2} \lvert H \rvert^2 \right)\right]}_{S_{NS}} - \underbrace{\frac{\sqrt{g}}{2} \sum_{k=0}^2 \frac{1}{k!}\lvert F_{2k+1} \rvert^2}_{S_{R}} \\
															& - \underbrace{\frac{1}{2} C_4 \wedge H_3 \wedge \mathcal{F}_3}_{S_{CS}} \, .
							\end{split}							
						\end{equation*}
					%
				Analogously to the type IIA case, we introduced $\mathcal{F}_n = \dd C_{n-1}$.
				This action has a constant shift symmetry $C_0 \rightarrow C_0 + c$, where $c$ is a constant. 
				Hence, it is referred to as an \emph{axion}~\cite{BeckerBeckerSchw, Weinberg:1977ma}.
				Furthermore, the five-form field strength $F_5$ satisfies the self-duality condition
					%
						\begin{equation}\label{F5dual}
							F_5 = \star F_5 \, ,
						\end{equation}
					%
				which has to be imposed as a further constraint together with the equations of motion.
				
					
				It is be useful to collect all the RR field strengths and potentials into a single polyform,
					%
						\begin{equation*}
							\begin{array}{lr}
								C = \sum_p C_p & p\ \text{odd/even for type IIA/IIB}\, , \\[3mm]
								F = \sum_p F_p & p\ \text{even/odd for type IIA/IIB}\, .								 
							\end{array}
						\end{equation*}
					%
				In this notation, the~\eqref{Fform} and~\eqref{Fdual} take the following form,
					%
						\begin{align*}
							F &= \dd_H C + e^B F_0\, , \\
							F &= \star s (F) \, ,
						\end{align*}
					%
				where we introduced the differential operator $\dd_h := \dd - H \wedge$ acting on polyforms, called \emph{$H$-twisted exterior derivative}, and the \emph{index reversal operator} $s$,
					%
						\begin{equation}
							s(A_p) = (-1)^{\left[p/2\right]} A_p \, .
						\end{equation}
					%
				The field strengths defined above are invariant under gauge transformations of potentials,
					%
						\begin{equation}\label{gaugetrans}
							\begin{split}
								\delta B &= - \dd \lambda \, , \\
								\delta C &= - e^B \wedge \left(\dd \omega - m \lambda \right) \, , \\
							\end{split}
						\end{equation}
					%
				where $\lambda$ is a one-form, $\omega$ is a polyform made of even/odd forms for type IIA/IIB and the term proportional to the Romans mass $m$ is there only in the type IIA case.			
				
				The RR field strengths have the following equations of motions and Bianchi identities (when there are no sources, like $\mathrm{D}_p$ branes).
				Bosonic fields equations for type II appear as,
					%
						\begin{align}
							(\dd e^{-2\phi} \star H) \pm \frac{1}{2} F \wedge \star F = 0 \, , \label{eom1}
							(\dd + H) \star F &= 0 \, , 
						\end{align}
					%
				where $\pm$ sign is referred to type IIA/B respectively, and
					%
						\begin{equation}
							\dd F = H \wedge F \, .
						\end{equation}
					%
				
				Notice that for type IIA the~\eqref{eom1},
					%
						\begin{equation}\label{Hstar}
							\dd (e^{-2\phi} \star H) + \frac{1}{2} \left[ F \wedge \star F \right]_8 = 0 \, ,
						\end{equation}
					%
				can be interpreted as the Bianchi identity for the dual seven-form field strength,
					%
						\begin{equation}
							\tilde{H} = e^{-2\phi} \star H \, .
						\end{equation}
					%
				We denoted by $[\ldots]_k$ the rank $k$ form of the polyform in the bracket.

				Making use of the self-duality relation for $F$~\eqref{Fdual}, we can rewrite the~\eqref{Hstar} as,
					%
						\begin{equation}
							\dd \left( \tilde H + \frac{1}{2} \left[ s(F) \wedge C + m e^{-B} \wedge C \right]_7 \right) = 0\, ,
						\end{equation}
					%
				which is solved by,
					%
						\begin{equation}
							\tilde H = \dd \tilde{B} - \frac{1}{2} \left[ s(F) \wedge C + m e^{-B} \wedge C \right]_7 \, .
						\end{equation}
					%
				Thus we introduce a new potential $\tilde{B}$~\cite{Bergshoeff:1997ak, Bergshoeff:2006qw}, whose (linearised) gauge transformations are fixed by requiring the invariance of its field strength,
					%
						\begin{equation}
							\delta \tilde{B} = -(\dd \sigma + m \omega_6) - \frac{1}{2} \left[ e^{B} \wedge (\dd \omega - m \lambda ) \wedge s(C) \right]_6 \, ,
						\end{equation}
					%
				where $\sigma$ is a five-form, while $\omega$ and $\lambda$ are the parameters of the gauge transformations~\eqref{gaugetrans}.
				
				An interesting point to notice about the massive IIA theory~\cite{RomansMass} is that one can obtain it from the non-massive one by shifting the gauge parameters as,
					%
						\begin{equation}\label{gaugeshifts}
							\begin{split}
								\dd \omega_0 &\longrightarrow \dd \omega_0 - m \lambda \, , \\
								\dd \sigma & \longrightarrow \dd \sigma + m \omega_6 \, .
							\end{split}
						\end{equation}
					%
				These relations will be the key of the construction of the exceptional generalised geometry for massive type IIA~\cite{oscar1}.
				
				Type IIB theory exhibits a non-compact global symmetry $\SL(2, \RR)$.
				This is not evident in the formulation we gave above, so we want to make it explicit.
				The two two-form potentials $B$ and $C_2$ can be organised into a doublet of $\SL(2, \RR)$,
					%
						\begin{equation}
							B^i := \begin{pmatrix}
									B \\
									C_2
							\end{pmatrix}^i \, .
						\end{equation}
					%
				Similarly, we introduce $F^i = \dd B^i$.
				Under an $\SL(2,\RR)$ transformations the $B$ fields transform linearly,
					%
						\begin{align}
							& &	B^i \longrightarrow \Lambda_{ij} B^j \, ,	& &	\Lambda_{ij} = \begin{pmatrix}
																					a & b \\
																					c & d					
																				\end{pmatrix} \in \SL(2, \RR) \, . 
						\end{align}
					%
				One can also define a complex scalar field $\tau$ which is the complex combination of the axion and the dilaton field, for this reason this is called \emph{axion-dilaton field}.
				This is useful since it transforms nicely under $\SL(2,\RR)$,
					%
						\begin{equation}
							\tau \longrightarrow \frac{a \tau + b}{c \tau + d} \, .
						\end{equation}
					%
				Then, type IIB action $S_{IIB}$ can be re-written in terms of $\SL(2,\RR)$ representations, like the symmetric matrix $h$,
					%
						\begin{equation}
							h_{ij} = \begin{pmatrix}
								\lvert \tau \rvert^2 & -C_0 \\
								-C_0 & 1
							\end{pmatrix}_{ij} \, ,
						\end{equation} 
					%
				transforming under $\SL(2, \RR)$ as
					%
						\begin{equation}
							h_{ij} \longrightarrow \Lambda^{ik} h_{kl} \Lambda^{lj} \, .
						\end{equation}
					%
				Then the action $S_{IIB}$ in terms of $\SL(2, \RR)$ covariant objects can be recast as,
					%
						\begin{equation}
							\begin{split}
								S_{IIB} = \frac{1}{2\kappa^2}& \int \dd^{10} x \ \sqrt{g} \left[e^{-2\phi} \left(R - \frac{1}{12}F^i h_{ij} F^j + \frac{1}{4} \partial h_{ij} \partial h^{ji} \right)\right] \\
											& - \frac{1}{8 \kappa^2}\int \dd^{10} x \ \left[ \sqrt{g} \lvert F_{5} \rvert^2 - \epsilon_{ij} C_4 \wedge F^i \wedge F^j \right] \, .
							\end{split}
						\end{equation}
					%
				The self duality condition on the $5$-form field strength~\eqref{F5dual} (which is a constraint in this formalism) is also $\SL(2,\RR)$ invariant.
				Moreover, one can re-write its definition in an $\SL(2,\RR)$ invariant form,
					%
						\begin{equation}
							F_5 = \dd C_4 + \frac{1}{2} \epsilon_{ij} B^i \wedge H^j \, .
						\end{equation}
					%
				
				%%%
					
				%%
				%
	\section{Supersymmetric backgrounds and compactifications}
				%
				After describing the actions and the equations of motion of eleven- and ten-dimensional supergravities, we are interested in solutions.
	 Since we want to study compactifications, we look for solutions that are warped products 
		
			\begin{equation*}
					\mathcal{M}_{10} = \mathcal{X} \times M_d\, ,
				\end{equation*}
				
				of a maximally symmetric external spacetime (Minkowski, Anti-de Sitter, de Sitter) and an internal space $M_d$.
				In order to preserve Poincar\'e invariance in the external spacetime, we must set all fermionic fields to zero so the background is purely bosonic.
				
				Then, the metric ansatz reads
					%
						\begin{equation}\label{metrsplit}
							\dd s^2_{\mathcal{M}} = e^{2A}\dd s^2_{\mathcal{X}} + \dd s^2_{M} \, ,
						\end{equation}
					%
				where $A$ is a real function of the coordinates on $M_d$, the \emph{warp factor}.
					
				We look for supersymmetric solutions. 			
				A background is supersymmetric if all the supergravity fields (and hence the solutions) are invariant under supersymmetry transformations.
				Choosing $\epsilon$ as the quantity parameterising supersymmetry variations, one is allowed to write (schematically)
					%
						\begin{align}\label{susyvar}
							& &	\delta(\text{boson}) = \epsilon (\text{fermion})\, ,	& &	\delta(\text{fermion}) = \epsilon (\text{boson})\, .	& &
						\end{align}
					%
				The variations of the bosonic fields always contain a fermionic field, and since we have set these to zero the variations automatically vanish.
				On the other hand, we get non-trivial conditions from the variations of the fermionic fields. 
				Then, supersymmetry of the background is equivalent to the existence of a non-vanishing spinor $\varepsilon$ for which the supersymmetry variations vanish.
				These can be recast into differential and algebraic equations, known as \emph{Killing spinor equations}.
				The spinor $\epsilon$ solving these is then called \emph{Killing spinor}.
				A background is supersymmetric if it admits Killing spinors. The Killing spinor equation for M-theory is 
					%
						\begin{equation}
							\nabla_M \epsilon + \frac{1}{288} \left[ \Gamma_M^{\phantom{M}NPQR} - 8 \delta_M^{\phantom{M}N} \Gamma^{PQR} \right] G_{NPQR}\ \epsilon =0 \, ,
						\end{equation}
					%
				where $M,N, \ldots = 0,1, \dots, 10$, $\epsilon$ is a Majorana spinor and the Gamma matrices are the Clifford algebra elements in $11$ dimensions. 
				
				Type IIA Killing spinor equation can be derived by the previous one by a compactification, so we do not give it explicitly.
Finally, for type IIB the Killing spinor equations, given in terms of ten-dimensional Gamma matrices are
					%
						\begin{subequations}
							\begin{align}
								\nabla_M \epsilon - \frac{1}{96} \left[ \Gamma_M^{\phantom{M}PQR} - 9 \Gamma^{PQ} \right] G_{MPQ}\ \epsilon^c + \frac{1}{192}\Gamma^{PQRS}F_{MPQRS} \epsilon =0 \, , \\
								i \Gamma^M P_M \epsilon^c + \frac{i}{24} \Gamma^{PQR} G_{PQR} \epsilon = 0 \, .
							\end{align}
						\end{subequations}
					%
				where, following~\cite{Gauntlett:2005ww}, we defined $P = \tfrac{i}{2}e^\phi \dd C_0 + \tfrac{1}{2}\dd \phi$ and $G = i e^{\phi/2}(\tau \dd B - \dd C_2)$, and here $M,N = 0,1, \dots, 9$.
				%
					
The Killing spinor equations are central in the study of supersymmetric string backgrounds. This is due to the fact that for backgrounds of the 	\eqref{metrsplit} one can show that
the supersymmetry variations plus the Bianchi identities for the NS and RR fields imply all other equations of motion. Thus a solution of the 			
Killing spinor equations is automatically a solution of the supergravity equations of motions. 				
	
	On backgrounds of the type~\eqref{metrsplit} the supersymmetry parameters factorise accordingly					%
						\begin{equation}
							\epsilon = \sum_{i=N} \varepsilon_i \otimes \chi \, ,
						\end{equation}
					%
				where $\varepsilon_i$ are anticommuting spinors on the external space and $\chi$ is a generic commuting spinor on the internal manifold.
				The number $N$ of spinors $\varepsilon_i$ detemine the number of supersymmetries preserved by the background. 
				This splitting induces also a splitting of the Killing spinor equations into distinct conditions for $\varepsilon$ and for $\chi$.
				The existence of Killing spinors (and the differential conditions they have to satisfy) on the internal manifold $M$ puts several constraints on the geometry of the manifold. 
				Investigating how this happens and how this allows compactifications with fluxes is the goal of next sections.
						
			\subsection{Calabi-Yau backgrounds in type II}
				%
				Let us start by a famous example of compactifications to four dimensions. We consider a purely geometric solutions where the only non-trivial field is the metric.
				We are going to see how supersymmetry conditions constrain the internal geometry to be Calabi-Yau.
				
				
				As discussed above, to find solutions it is enough to solve the supersymmetry variations for the spinors~\eqref{susyvar}.
				In type II theory, in absence of fluxes they reduce to 
					%
						\begin{equation}
							\begin{array}{lcr}
								\delta \lambda_1 = \partial_M \phi \Gamma^M \epsilon_1 = 0 \, , & & \delta \lambda_2 = \partial_M \phi \Gamma^M \epsilon_2 = 0 \, ,
							\end{array}
						\end{equation}
					%
				for the dilatino variations, while the gravitino variations reduce to the requirement that the supersymmetry parameters must be covariantly constant
					%
						\begin{equation}\label{grav0}
							\begin{array}{lcr}
								\delta \psi^1_{M} = \nabla_M \epsilon_1 = 0 \, , & & \delta \psi^2_{M} = \nabla_M \epsilon_2 = 0 \, .
							\end{array}
						\end{equation}
					%
				The supersymmetry parameters decompose as 
					%	
						\begin{align}\label{spinansatz}
							& \epsilon_1 = \zeta_1 \otimes \eta_1 + \text{c.c.} \, , \\
							& \epsilon_2 = \zeta_2 \otimes \eta_2 + \text{c.c.} \, .
						\end{align}
					%
				Here $\zeta$ is a four-dimensional chiral spinor $(\gamma_5\zeta=\zeta)$ and $\eta_{1,2}$ are six-dimensional chiral spinors, of opposite chirality in IIA and same chirality in IIB
					%
						\begin{align}
							& &	\gamma_7\eta_{1}=\eta_{1} & & \gamma_7\eta_{2}=\mp\eta_2 & & \mbox{in IIA/IIB} \, .
						\end{align}					
					%
					
				Using the decomposition ansatz for the metric~\eqref{metrsplit}, and supersymmetry parameters~\eqref{spinansatz}, we obtain the six-dimensonal equations
\begin{equation}\label{intdilCY}
\slashed{\partial} \phi\,\eta_{1,2}=0\ ,
\end{equation}
where $\slashed{\partial}\phi=\gamma^m\partial_m\phi$. This implies that the dilaton must be constant, $\partial_m\phi=0$, since $||\slashed{\partial}\phi\,\eta_{1,2}||^2=(\partial\phi)^2||\eta_{1,2}||^2$.


The gravitino variations reduce to 
\begin{align}
& \nabla_\mu \rightarrow \nabla_\mu\otimes \id 
+ \frac{1}{2}e^A (\gamma_\mu \gamma_5 \otimes \slashed\partial A )\ , \\
& \nabla_m \rightarrow \id\otimes \nabla_m\ ,
\end{align}
where, on the right hand side, $\nabla_\mu$ and $\nabla_m$ are the covariant derivatives with respect to the external four-dimensional unwarped metric and the internal six-dimensional metric, respectively. Using again~\eqref{spinansatz}, we can decompose~\eqref{grav0} into
an external (four-dimensional) and an internal (six-dimensional) part as
\begin{align}
\label{extgravsplit} 
& \nabla_\mu\zeta_1 \otimes \eta_1 - \frac{1}{2}e^A (\gamma_\mu \zeta^* \otimes \slashed\partial A\, \eta_1^*)+\text{c.c.}=0 \, , \\
\label{intgravsplit}
& \zeta_1 \otimes \nabla_m \eta_1+\text{c.c.}=0 \ .
\end{align}
An identical equation holds for $\eta_2$.
The external gravitino equations~\eqref{extgravsplit} imply that $\slashed\partial A\, \eta_{1,2}^*$ should be proportional to $\eta_{1,2}$, which is impossible since $\eta_{1,2}^\dagger\gamma_m\eta_{1,2}=0$. It follows that the warping must be constant. Taking this into account,~\eqref{extgravsplit} further reduces to
\begin{equation}
\label{extgravsplit2}
\nabla_\mu\zeta_{1,2}=0 \, .
\end{equation}
The commutator of two external covariant derivatives gives
 \begin{equation}\label{extint}
[\nabla_\mu, \nabla_\nu]= \frac{1}{4\!} R_{\mu \nu \rho \sigma} \gamma^{\rho \sigma}= \frac{\Lambda}{6} \gamma_{\mu \nu} \, ,
\end{equation}
where we used the expression for the curvature tensor for a maximally symmetric (unwarped) four-dimensional metric:
$R_{\mu \nu \rho \sigma} = \frac{1}{3} \Lambda (g_{\mu \rho} g_{\nu \sigma}
- g_{\mu \sigma} g_{\nu \rho})$. 
%This is a solution of the Einstein equations containing {\em cosmological constant} $\Lambda$:
%\eq{
%R_{\mu\nu} - \frac{1}{2} g_{\mu\nu} R + \Lambda g_{\mu\nu} = 0 \,.
%}
From~\eqref{extgravsplit2} and~\eqref{extint}, it then follows $\Lambda\gamma_{\mu \nu}\zeta=0$
which implies the vanishing of the cosmological constant
\begin{equation}
\label{lapl}
\Lambda = 0 \, .
\end{equation}
Then, the external gravitino equations require that the warp factor $A$ must be constant, 
and the four-dimensional space must be Minkowski ($\Lambda =0$).

Let us now turn to the internal gravitino equations,~\eqref{intgravsplit}, which reduce to
\begin{align}
\label{intgrav2}
& & \nabla_m\eta_1=0\, , & & \nabla_m\eta_2=0\ . & &
\end{align}
Applying the same argument as below~\eqref{extint}, we see that the internal metric must be Ricci flat.

Moreover, a covariantly constant spinor implies a reduction of the holonomy group of a Riemannian manifold.
From~\eqref{intgrav2} it follows that the internal metric must have at most holonomy SU(3). A Ricci flat manifold of SU(3) holonomy is a Calabi-Yau.



If the internal metric has strict SU(3)-holonomy, then $\eta^1_+$ and $\eta^2_+$ must be proportional. Without loss of generality we can set $\eta^1_+=\eta^2_+=\eta$ with $\eta^\dagger\eta=1$. We can express 

We can construct the forms $\omega$ and $\Omega$ as spinors bilinears of the covariantly constant spinor $\eta$. Then 
we can rewrite the Calabi-Yau condition $\nabla_m\eta=0$ in the alternative form
					%
						\begin{equation}
							\begin{array}{lcr}
								\dd \omega = 0\, , & & \dd \Omega = 0 \, .
							\end{array}
						\end{equation}
					%
				
				Thus, we got an important result: in absence of fluxes, looking for a supersymmetric vacuum requires to consider a Calabi-Yau three-fold as internal manifold.
	
The properties of Calabi-Yau's manifolds are such that one can explicitely derive the four-dimensional effective action (see~\cite{CeresoleN2, Bodner:1990zm} for details).		
			This is an $\mathcal{N} = 2$ supergravity theory in four-dimensions, that is characterised by the presence of massless uncostrained scalars, the moduli. 
			In supersymmetric theories massless scalar fields are not a problem, the trouble is if some of them stay massless after SUSY breaking: massless scalar fields would provide long range interactions that are not observed in nature. 
			One solutions to the moduli problem is to find ways to generate potential terms for some or all such scalars. 
			One way is to consider compactifications admitting non-trivial fluxes.
			%	
		\subsection{Backgrounds with fluxes in type II}
			%		
			We now turn to the study of more general solutions of type II supergravity where some of the fluxes have non-zero values. 
			The presence of fluxes drastically changes the properties of the solutions. This can be seen both from the equations of motion and the
supersymmetry variations.		
Indeed, from the Einstein equation, which reads schematically
\begin{equation}
R_{MN} \sim H_{MPQ}H_N{}^{PQ}+\sum_p F_{M Q_{1} \ldots Q_{p}} F_N{}^{\, Q_{1} \ldots Q_{p}} \, ,
\end{equation}
we see that the fluxes back-react on the metric, which generically cannot be Ricci-flat (and thus Calabi-Yau) anymore. 
%Therefore the geometry is modified with respect to the fluxless case. 
Another generic feature is a non trivial warp factor in the ten-dimensional metric. 

The supersymmetry variations are also modified. For example, from~\eqref{susyvar} one can see that in the presence of RR fluxes the supersymmetry conditions relate $\epsilon^{1}$ and $\epsilon^2$ so that the four-dimensional components $\zeta_{1,2}$ cannot be chosen independently anymore, as in~\eqref{spinansatz}. Therefore, in the presence of RR-fluxes one generically obtains $\mathcal{N}=1$ in four dimensions. 
	

Repeating the strategy used in the fluxless case, we decompose the supersymmetry conditions according to the compactification
ansatz. We do not give all details here, but one can easily see that the internal gravitino variations become
%
\begin{equation}\label{intgravsplitagain}
\begin{split}
(\nabla_m+\frac14\slashed{H}_m)&\eta_1+\frac{1}{8} e^\phi\slashed{F}\gamma_m\gamma_7\eta_2 = 0\ , \\
(\nabla_m-\frac14\slashed{H}_m)&\eta_2-\frac{1}{8} e^\phi\slashed{F}^\dagger\gamma_m\gamma_7\eta_1 = 0\ , 
\end{split}
\end{equation}
%
from which we see that, generically, the internal manifold is no longer Ricci flat and hence no longer Calabi-Yau, since
\begin{equation}\label{susyintegr}
	[\nabla_m , \nabla_n ] \eta_{1,2} = \frac{1}{4} {R}_{mn}{}^{ pq} \gamma_{pq} \eta_{1,2} \neq 0 \, .
\end{equation}
%This is in agreement with the argument based on the equations of motion at the beginning of this section.

It is therefore natural to wonder whether it is still possible
to say something about the geometry of the internal manifold.
Generalized Complex Geometry provides a general framework to describe flux backgrounds in string theory.

								%
\end{document}