\documentclass[draft]{phd}

\begin{document}
	%
	\remerciements
		%
			Une thèse c'est un travail long et complexe, et donc on peut comprendre qu'il soit impossible de le complêter sans l'aide incommensurable de beaucoup de personnes.
			Peut-être que j'oublierai quelqu'un, mais si leurs noms ne sont pas dans ma tête, ils sont dans mon coeur.
			
			Tout d'abord, je tiens à remercier vivement ma directrice de thèse, Michela, pour avoir accepté de me prendre en thèse, quand je ne savais presque rien de la théorie des cordes. 
			J'ai beaucoup appris sous sa tutelle, en physique et en general d'un point de vue humain. 
			Je la remercie pour toutes les discussions sans fin et pour les réponses à mes questions souvent bêtes.
			Je souhaite lui exprimer ma gratitude pour ça et pour avoir été un point de référence et un exemple incroyable de rigueur scientifique et personnelle.
			
			L'amitié avec mes collègues thésards a été précieuse. Merci vraiment à tous.
			En particulier, sans les pauses café avec Enrico, Ruben, Sophie, Thomas, Charles, Johannés, Alessandro et Constantin, je n'aurais jamais survécu à la derniere année. 
			Sans l'aide de ces pauvres diables que sont Hugo, Matthieu et Thomas je n'aurai jamais appris le français et jamais eu l'opportunité de devenir si fort à SportsHead.
			Enfin, merci à Enrico pour sa force tranquille et à Ruben pour sa folle sagesse.
			
			Merci à tous les membres du \textsc{LPTHE} pour leur accueil chaleureux et pour avoir été une communauté de laquelle je me suis senti toujours partie.
			En particulier, je suis vraiment reconnaissant envers Isabelle et Françoise pour leur aide avec la bureaucratie qui a été pour moi un obstacle parfois plus grand que la théorie des cordes.
			
			Je veux remercier aussi Dan Waldram, qui m'a introduit à la geometrie généralisée, et m'a donné confiance quand je n'étais rien que un étudiant effrayé, et Charles Strickland-Constable pour les discussions précieuses sur la théorie des groups et des représentations. 
			Une mention speciale à Davide Cassani, qui n'a pas seulement été une source des profondes connaissances de la théorie des cordes, mais aussi un support pendant mes moments de difficulté.
			
			Ma vie à Paris n'aurais pas été la même sans l'amitié de Andrea, Davide, Lara, et tous les autres ``monas'', merci de tout mon coeur.
			Merci à mes amis loins, qui sont proches dans l'esprit, parce que quand on se voit c'est comme si on avait toujours été ensemble.
			
			Je souhaite aussi souligner que je ne serais pas ici aujourd'hui sans ma famille. 
			Merci à ma mère et à mon père pour être toujours là, sans être intrusifs, pour m'avoir enseigné à penser de façon indépendante et libre.
			Merci à ma soeur Sara, pour me donner toujours une comparaison et pour me soutenir quand je me sens perdu.
			Merci à ma soeur Camilla, pour être joyeuse et 	joviale, sans jamais être superficielle, pour être sage et gentille avec tous.
			
			Finalement, je remercie chaleureusement les membres du jury pour avoir accepté d'être présents à la soutenance de cette thèse. 
			Merci à Anna Ceresole, Mariana Graña, Henning Samtleben, Jan Troost et Alberto Zaffaroni.
		%
	%
\end{document}