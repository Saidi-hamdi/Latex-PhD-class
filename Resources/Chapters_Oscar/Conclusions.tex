\documentclass[debug]{phd}

\begin{document}
%%
\ensurepagenumbering{arabic}
%%	
	%
	\chapter*{Conclusions}
	
		The major aim of this thesis was to study flux backgrounds.
		In the first part we focused on the problem of seeking for a systematic way to find consistent truncations in the presence of fluxes.
		In the second part, we were more concerned in reformulating the supersymmetry conditions for brane probes in AdS backgrounds, in terms of integrability of exceptional structures in exceptional geometry.
	
		This thesis contains the construction of a generalised geometric description of type IIA flux backgrounds.
		
		We made a large use of the formalism of $G$-structures and generalised geometry.
		As in conventional geometry, integrability is defined as the existence of a generalised torsion-free connection that is compatible with the structure, or equivalently as the vanishing of the generalised intrinsic torsion.
		In the case analysed, \emph{i.e.} truncations of massive type IIA, the integrability conditions correspond to the Leibnitz algebra for parallelisations.
		These enquires the trucation preserves maximal supersymmetry in lower dimension~\cite{spheres}.
		Moreover, the notion of \emph{generalised Leibniz parallelisation} is the key ingredient to construct a consistent truncation ansatz.
		We showed in chapter~\ref{chapComp} how to build consistent truncation ansatze, by the so-called~\emph{Generalised Scherk-Scwharz reductions}, making use of generalised Leibniz parallelisations.
		We constructed various examples of massive type IIA spheres truncations.
		In the case of the truncation on a six-dimensional sphere, we obtained a generalised parallelisation on $S^6$ satisfying the $\ISO(7)$ algebra, and spelled out the corresponding truncation ansatz as obtained from the generalised Scherk--Schwarz prescription. 
		As recently described in \cite{Guarino:2015jca,Guarino:2015vca}, the Romans mass introduces a magnetic gauging of the $\ISO(7)$ translations in the truncated four-dimensional theory, yielding a symplectic deformation~\cite{Dall'Agata:2014ita} of the type first found in~\cite{Dall'Agata:2012bb} for the $\SO(8)$ gauging.
		We found the same phenomenon for type IIA supergravity on the six-dimensional hyperboloids $H^{p,7-p}$: on these spaces one can define a consistent truncation down to $\ISO(p,7-p)$ supergravity in four-dimensions; switching the Romans mass on leads to the symplectically-deformed $\ISO(p,7-p)$ gauging described in~\cite{Dall'Agata:2014ita}.
		We also obtained generalised Leibniz parallelisations on $S^4$, $S^3$ and $S^2$ for vanishing Romans mass, reproducing the Leibniz algebra of known consistent truncations of massless type IIA supergravity on these manifolds. 
		When the Romans mass is switched on, these parallelisations no more satisfy a Leibniz algebra. 
		We offered an explanation of why this is the case by showing that the frame lies in the stabiliser group of the Romans mass only for the parallelisation on $S^6$. 
		For massive type IIA on $S^3$ we presented a no-go result indicating that a consistent truncation including the $\SO(4)$ algebra does not exist. 
		It would be interesting to see whether similar no-go theorems can be proved for the $S^4$ and $S^2$ cases.
		
		As said, in order to construct apply the generalised Scherk-Schwarz prescription, we built the adapted version of generalised geometry for massive type IIA. 
		The principal issue in this construction is accommodating the flux due to the Romans mass. 
		This is achieved by deforming the generalised Lie derivative such that the deformed one generates the gauge transformations of type IIA supergravity with a non-zero $m$-flux.

		An interesting fact is the existence of an alternative massive type IIA~\cite{Howe:1997qt}. 
		This can be obtained from eleven-dimensional supergravity by gauging a combination of the $\GL(1)$ global symmetry and the trombone symmetry of the equations of motion.
		It is not known a description of such a theory through a Lagrangian.
		One may wonder whether other massive extensions of type IIA can exist.
		However, in~\cite{Tsimpis:2005vu}, making use of superspace arguments, it was discussed how this and the Romans mass are the only possible extensions.
		It is natural to ask how this deformation appears in our formalism.
		We want deformation parameters to be diffeomorphism invariant then we require them to appear as $\GL(6)$ singlets with zero $\RR^+$ weight. 
		There are precisely two such singlets in the $\mathbf{912}_{-1}$ representation of $\E_{7(7)}\times\RR^+$, one of which we have already identified as the Romans mass deformation.
		There is also a singlet in the $\mathbf{56}_{-1}$ representation, which is another part of the generalised torsion~\cite{Coimbra:2011ky}, and which could also be used to deform the Dorfman derivative. 
		When performing generalised Scherk-Schwarz reductions, this additional $\mathbf{56}_{-1}$ part of the embedding tensor is generated by gauging the trombone symmetry~\cite{LeDiffon:2008sh}, and the resulting theory does not have an action. 
		It is natural to conjecture that deforming the Dorfman derivative by switching on a combination of the second singlet in $\mathbf{912}_{-1}$ and the singlet in $\mathbf{56}_{-1}$ would give the relevant gauge algebra for the theory described in~\cite{Howe:1997qt}.
		The result of~\cite{Tsimpis:2005vu} can be verified in this case, since, by considering the closure of the gauge algebra, one can argue that there are no others deformations, as singlets of $\GL(6)$ in the torsion representation bundle.
		
		In the last chapter of the thesis, we focused on brane calibrations.
		The tools used are again generalised $G$-structures.
		In particular, we concentrated our attention on AdS backgrounds with eight supercharges.
		These have an elegant description in generalised geometry in terms of exceptional Sasaki-Einstein structures~\cite{AshmoreESE, Grana_Ntokos}.
		In~\cref{chapbrane}, we studied the relation between the Exceptional Sasaki-Einstein structures and generalised brane calibrations in $\mathrm{AdS}_5 \times M_5$ backgrounds in type IIB and in $\mathrm{AdS}_5 \times M_6$ and $\mathrm{AdS}_4 \times M_7$ compactifications in M-theory.
		We focussed on the calibrations forms associated to branes wrapping cycles in the internal manifolds and that are point-like in the AdS space. 
		We showed that for these configurations the general expression for the calibration forms that can be constructed using $\kappa$-symmetry can be expressed in terms of the generalised Killing vector $K$ defining the Exceptional Sasaki-Einstein structure and that the closure of the calibration forms is given by the integrability (more precisely the $L_K$ condition) of the ESE structure. 
		The results of the chapters prove the conjecture appeared in~\cite{AshmoreESE} that the (form part of the) generalised Killing vector is a generalised calibration.
		The motivation of this conjecture can be found in holography~\cite{Maldacena:1997re}.
		One can observe that the generalised killing vector $K$ generates the global $R$-symmetry of the filed theory dual to the AdS background in supergravity.
		It is made by a combination of the vector part (generating diffeomorphism) and $p$-forms (parametrising gauge transformations), under which the generalised metric is invariant.
		Thus there is a sign that $K$ is an object related to $R$-symmetry in a non-trivial way, since it encodes informations about the gauge transformations of flux potentials.
		In AdS/CFT correspondence, BPS branes have volume associated to the conformal dimension of chiral operator in the dual SCFT, thus finding a calibration (that is a supersymmetric circle on which branes wrap) is equivalent to find the conformal dimensions of the dual operators.
		
		We have seen how to construct these calibrations in generalised geometry and how BPS conditions correspond to the integrability of the structures.
		
		We also partially discussed other brane configurations that are calibrated by the vector $K$.
		However we did not perform a complete analysis, leaving the discussion for a future work.
		
	%
	\section*{Future works}
		%
		There are many other directions for future study.
		
		For intance, so far, there is not a description of supergravity flux backgrounds in terms of generalised geometry for any amount of supersymmetry.
		As stated, supersymmetric background preserving $\mathcal{N}$ supersymmetries are given by integrable $G$-structures, where $G$ is the stabiliser group of the $\mathcal{N}$ Killing spinors~\cite{AndCharSpecHol}.
		Maximally supersymmetric backgrounds, as seen are described by parallelisations~\cite{spheres, Baguet:2015sma}.
		Half-maximal supergravity truncations have been recently described in exceptional field theory~\cite{Malek:2017cjn}, but a generalised geometry formulation is not known at the moment.
		The $\mathcal{N}=2$ backgrounds, as seen in~\cite{AshmoreESE, AshmoreECY, Grana:2009im, Malek:2016bpu}, is a rich field for generalised $G$-structure applications.
		An $\mathcal{N}=1$ formalism to describe vacua is known in $\rmO(d,d)$ generalised geometry~\cite{petrini2, Grana:2005sn, Grana:2006kf} and recently an exceptional picture has been found in~\cite{CoimbraN1}. 
		However, the analysis is far to be complete.
		Hence, obvious extension is to consider backgrounds with different amounts of supersymmetry, which will be described by new geometric structures within generalised geometry.
		In~\cite{oscar5}, there is a work in progress development of the structures to describe supersymmetric backgrounds with sixteen supercharges.
		Furthermore, the hope is not just to find a new descriptions of such backgrounds, but that this would lead to discover new examples of truncations.
		In addition, an ambitious project would be a complete classification of such structures analysing the mathematical constraints on the internal geometries and finding a coherent structure describing them.
		
		Moreover, all these techniques are may also be applied in different contexts from consistent truncations, like for example, studies of holography effects, marginal deformations of the dual field theories, etc.
		
		More specifically, the formalism developed in the first part of this thesis about massive type IIA may also be applied to investigate marginal deformations of conformal field theories holographically dual to (massive) type IIA AdS backgrounds.
		Recently this theories have been identified with a class of Chern-Simons-matter theories~\cite{Guarino:2015jca}, and in~\cite{oscar4}, we aim to describe exactly marginal deformations of such conformal theories by constructing (and studying their deformations) the exceptional Sasaki-Einstein structures for massive type IIA, describing the AdS background in~\cite{Varela:2015uca} preserving eight supercharges.
		The work of~\cite{AshmoreDef} is the first example in this sense.
		There, the authors study the exceptional Sasaki-Einstein structures describing AdS$_5$ background in both type IIB and eleven-dimensional supergravity to analyse marginal deformations of the dual $\mathcal{N}=1$ CFT in four dimensions.
		A famous result in gauge theory~\cite{Green:2010da} states that marginal deformations are determined by imposing $F$-term conditions on operators of conformal dimension three and then quotienting by the complexified global symmetry group.
		In~\cite{AshmoreDef}, it was shown that this result has a geometrical interpretation: the marginal deformations are obtained as solutions of moment maps for the generalised diffeomorphism group that have the correct charge under the Reeb vector.
		Indeed, the Reeb generates the $U(1)_R$ symmetry group.
		In the case this is the only symmetry of the background, then all the marginal deformations are exactly marginal.
		When there are other global symmetries, the field theory result predicts one has to quotient out these symmetries.
		On the supergravity side, this can be read as fixed points of the moment maps, being an obstruction for marginal deformations to be exactly marginal.
		
		This analysis holds for any kind of internal geometry in an AdS background, however, so far the examples to which this has been applied are all Sasaki-Einstein geometries.
		So, it would be interesting to apply it to one of the few examples of non-Sasaki–Einstein backgrounds, such as the Pilch–Warner solution~\cite{PilchWarner}.
		This would give the marginal deformations of the dual SCFT, the so-calle Leigh-Strassler theory~\cite{LSexact}.
		We aim to study this example in~\cite{oscar3}.
		
		To conclude, generalised geometry gives tools to better analyse several aspects of supergravity and string theory.
		It has also applications in holography and field theory.
		Furthermore, it is interesting also in pure mathematics, since it is related to various areas beyond differential geometry, like algebraic topology, algebraic geometry and group theory.
		Thus it provides an example of a topic that lies at the frontier between mathematics and physics, on the one hand, receiving deep insights from both fields, but on the other hand it could also give some useful tools to understand and answer questions in both areas.
		For these reasons, generalised geometry seems to be worth of further efforts and studies, since its U-duality covariant approach may reveal something hidden so far and perhaps help us to understand the geometrical nature of dualities in string theory.
		
\end{document}